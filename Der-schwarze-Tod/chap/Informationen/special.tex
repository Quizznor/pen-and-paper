\section*{Abenteuerspezifische Regeln}

Ein wichtiges Element dieses Abenteuers ist die Tatsache, dass Spielcharaktere im Verlauf der Handlung mit der Pest in Berührung treten. Sind sie dabei fahrlässig kann es passieren, dass sie selbst an der Pest erkranken. Hierzu wurde die \gqm{Pestilenz} implementiert, die den Fortschritt einer Pestinfektion anzeigt.

\begin{itemize}
  \item \violet{\textbf{Moral}}:
  Alle Stellen, an denen die Spieler mit moralischen Fragen konfrontiert werden, sind \violet{lila} gekennzeichnet. Diese markieren Entscheidungen, die sich auf den Ausgang der Geschichte auswirken können.

  \item \brown{\textbf{Pestilenz}}:
  Alle Stellen, an denen die Spieler in Kontakt mit der Pest kommen und gegebenenfalls Pestilenz anhäufen können, sind \brown{braun} markiert.
\end{itemize}

Die Spieler wissen davon jedoch nichts, da dieser Wert nur dem Spielleiter bekannt sein sollte, jedoch äußert sich die Pestilenz in Symptomen die zufällig auftreten und wie folgt verschlüsselt sind:

\begin{tabular*}{\textwidth}{@{\extracolsep{\fill}} lr}
  \toprule
  Anzahl & Auswirkung \\
  \midrule
  1-2 & Schwindel, Gleichgewichtsstörungen \\
  3-4 & Schüttelfrost, Fieber \\
  5-6 & Kopfschmerzen \\
  7-8 & akutes Fieber, Gliederschmerzen \\
  9-10 & eitrige Beulen in der Leistengegend / unter den Achseln / am Hals \\
  >10 & strenge Bettlägerigkeit, nach kurzer Zeit verendet er / sie \\
  \bottomrule
\end{tabular*}
