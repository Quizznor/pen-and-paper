%!TEX root = ../main.tex

\brown{\textbf{Pestilenz}: Jeder, der das Armenviertel betritt, erhält +1 Pestilenz.}

\red{\textbf{Szene}}:

Ihr kommt also in Hammerbrook an, einem der Stadtviertel in denen sich die Ärmsten der Stadt zusammenpferchen in der Hoffnung vom Reichtum zu profitieren den der Handel in Hamburgs Kassen spült. Dicht gedrängt leben hier Hafenarbeiter, einfache Leute und anderes Gesindel in ärmlichen Hütten und Häusern. Man merkt, dass hier andere Regeln zu gelten scheinen als im Rest der Stadt, die Straßen sind gesäumt von Toten und Kranken, und in kaum einem Haus brennt Licht. Der Tod geht um, und tagtäglich rechnen hier viele mit dem schlimmsten.

\subsection*{Auf den Straßen}
\label{strasse}

\red{\textbf{Szene}}:

Die Straßen sind verschneit und verlassen. Nur vereinzelt ziehen vermummte Gestalten umher und werfen unserer Gruppe argwöhnische Blicke zu. Ihr scheint hier nicht willkommen zu sein.

\red{\textbf{Information für den Spielleiter}: Sollte einer der Spieler in auffallend feinen Kleidern durch das Armenviertel wandern wird die Gruppe später überfallen. (gehe zu} \blue{\ref{kampf2}}\red{).}

Plötlich kommen mehrere verhärmte Kinder unter lautem Geschrei auf euch zu. Sie haben einen unförmigen Lederklumpen dabei und scheinen ganz offensichtlich Fußball zu spielen. Als sie auf eurer Höhe sind rempeln einige der größeren Kinder euch an und murmeln eine Entschuldigung, bevor sie wieder in einer Seitengasse verschwinden.

\red{\textbf{Probe auf Gassenwissen o.ä.}: Alle würfeln eine Probe. Wer diese erfolgreich besteht bemerkt, dass ihm ein Gegenstand geklaut wurde.}

Während die Spieler das bemerken (oder auch nicht!) biegen die Kinder gerade aus ihrem Sichtfeld in eine kleine Seitengasse ein. Die Spieler können die Kinder jedoch noch versuchen mit einer Probe auf Rennen (um 10 erschwert) die Kinder zu verfolgen.

\begin{itemize}
  \item Erfolg: Holen sie die Kinder ein (gehe zu \blue{\ref{eingeholt}}).
  \item Misserfolg: Die Gegenstände sind verloren (gehe zu \blue{\ref{neingeholt}})
\end{itemize}

\subsection*{Option 1 - Beim Lumpensammler}
\label{eingeholt}

\red{\textbf{Szene}}:

Ihr verfolgt also die Kinder bis diese die Tür zu einem Laden aufstoßen und darin verschwinden. Wenige Momente später betretet auch ihr außer Atem den Laden. Von den Kindern ist keine Spur zu sehen. Jedoch begrüßt der Lumpensammler Luis (\blue{\ref{Luis}}) überschwänglich die neue \gqm{Kundschaft}.

Auf Nachfrage stellt sich heraus, dass der Lumpensammler die Gegenstände der Spieler hat, aber nicht ohne weiteres herausgeben will. Nun hat die Gruppe folgende Optionen:

\begin{itemize}
  \item Probe auf Einschüchtern, um 30 erschwert (gehe zu \blue{\ref{fertig}})
  \item Die Gruppe geht wieder (gehe zu \blue{\ref{neingeholt}})
  \item Der Lumpensammler möchte mit den Spielern wetten (gehe zu \blue{\ref{wette}})
\end{itemize}

\subsubsection{Das Rätsel}
\label{wette}

Lumpensammler: \gqm{\textit{Ich hab es! Wir schließen eine Wette ab, ich stelle euch ein Rätsel das es zu lösen gilt. Wenn ihr es löst gebe ich euch eure Sachen wieder, schließlich können so piekfeine Schnösel wie ihr es seid bestimmt gut ein Rätsel lösen?}}

Voller Hochmut und Vorfreude trägt der Lumpensammler das Rätsel vor:

Lumpensammler: \gqm{\textit{Einst wurde ein Pirat gefasst der dem König sein Gold gestohlen hatte. Der König tobte und verlangte, dass der Pirat unverzüglich am Galgen aufgeknüpft werde. Es war aber üblich, den zum Tode verurteilten Dieben eine letzte Chance zu geben und Gott über sie richten zu lassen. Daher mussten sie aus einem schwarzen Säckchen einen Stein ziehen. Im Säckchen befanden sich immer genau ein weißer und ein schwarzer Stein. Zog der Dieb den weißen Stein, wurde ihm die Freiheit geschenkt. Zog er hingegen den schwarzen Stein, so baumelte er.  Eines Tages kam nun jener Pirat der einst das Gold des Königs geraubt hatte vor den Scharfrichter und wartete auf sein Gottesurteil.}}

Lumpensammler: \gqm{\textit{Der König aber wollte sichergehen, dass der Halunke hängt und hat dem Henker am Abend zuvor im Heimlichen befohlen zwei schwarze Steine in das Säckchen zu legen.
So ging der König am nächsten Tage also voller Zuversicht also zum Richtplatz, wo überall weiße und schwarze Steine herumlagen. Als es Zeit wurde für den Verurteilten sein Urteil zu erhalten bückte sich der Henker und nahm zwei schwarze Steine vom Boden auf, die er im Säckchen ablegte. Der Pirat sah dies jedoch und wusste somit, dass er kein gerechtes Urteil erhalten würde. Er glaubte die Schlinge schon um seinen Hals als ihm die rettende Idee kam. \\
Er zog einmal und musste freigelassen werden. \\ Was war es, dass dem Piraten das Leben rettete?}}

\begin{itemize}
  \item Erfolg: Die Lösung des Rätsels besteht darin, dass der Dieb einen Stein zieht und ihn sogleich wegwirft. Da der verbleibende Stein im Sack schwarz ist, muss der gezogene Stein scheinbar weiß gewesen sein. Beantwortet die
  \item Misserfolg: Können sie das Rätsel nicht lösen oder liegen falsch wird der Lumpensammler ihnen ihre Gegenstände nicht zurückgeben, er genießt seinen Triumph viel zu sehr!
\end{itemize}

\subsubsection{Abschließen der Aufgabe}
\label{fertig}

Ungeachtet der vorigen Ereignisse wird der Lumpensammler ihnen erzählen, dass sie nicht die ersten wohlbetuchten Personen sind, die in den letzten Tagen in seinem Laden waren. Es ist noch nicht allzu lange her, da sei einer in einem ganz komischen Aufzug hier hereingeschneit. Er habe nach abstrusen Pflanzen gefragt. Chillies und Zitronen seien darunter gewesen. Als ob man sowas hier bekomme. Außerdem habe er noch einen Blasebalg gewollt und etwas grobes Leinen, Lumpen und Teer. Komischer Typ.

Er kann den Abenteurern jedoch die grobe Richtung beschreiben in die der Kunde verschwunden ist. Nachdem er ihnen das erzählt hat drängt Luis die Gruppe zu gehen, er hat noch andere Kundschaft um die er sich kümmern muss. Wieder auf der Straße stehend können die Spieler nach der Person fragen die der Lumpensammler ihnen beschrieben hat oder dem vorgeschlagenen Weg folgen. Nach einem kurzen Fußmarsch kommen sie dann an einem Lagerhaus an (gehe zu \blue{\ref{Lagerhaus}})

\subsection*{Das Treffen mit Sigrun}
\label{neingeholt}

\red{\textbf{Szene}}:

Ihr probiert noch den Kindern nachzueilen, doch als ihr um die Ecke biegt hinter der ihr sie eben noch verschwinden sehen habt ist keine Spur mehr zu finden, sie scheinen sich in Luft aufgelöst zu haben.

Jedoch schaut euch eine dicklichere Frau verwundert an.

Frau: \gqm{\textit{Ihr wurdet wohl von den Kindern hier ausgeraubt? Findet euch lieber schnell mit dem Verlust ab, in den Gassen hier habt ihr keine Chance sie noch irgendwo aufzufinden.}}

Im weiteren Verlauf des Gesprächs stellt sie sich als die Bäckerin Sigrun (\blue{\ref{Sigrun}}) vor. Sie folgt dem Ruf von Pater Salus und ist gerade auf dem Weg nach St. Petri. Dort sind unzählige Menschen und keiner von ihnen ist krank. Ein sicherer Aufenthalt, so scheint es. Allerdings müsse sie erst in Erfahrung bringen, ob auch ihre Kinder dort Willkommen seien, da sie mit Fieber zu Hause im Bett liegen. Sigrun bittet die Gruppe, bei ihr zu Hause vorbeizuschauen und nach den Kindern zu sehen während sie weg ist.

\violet{\textbf{Moral}: Wie wird die Gruppe mit der Bitte von Sigrun umgehen?}

\begin{itemize}
  \item Die Gruppe hilft den Kindern nicht, Sigrun zieht niedergeschlagen davon, und die Gruppe bleibt auf sich gestellt.
  \item Sie helfen den Kindern, Sigrun weist ihnen den Weg zu ihrem Haus und verrät, wo der Zweitschlüssel hinterlegt ist.
\end{itemize}

\subsubsection{Bei Sigrun Zuhause}

\red{\textbf{Szene}}:

Nach einem kurzen Fußmarsch steht ihr vor Sigruns Haus. Vor der Türe des Hauses steht ein Gestell, auf dem Brote zum Verkauf angeboten werden können, momentan ist es jedoch leer. Ansonsten gibt es nicht viel zu sehen. Ein unangenehmer Geruch liegt in der Luft. Öffnen die Spieler die Türe wird der Geruch stärker.

\brown{\textbf{Pestilenz}: Alle die das Haus betreten erhalten +2 Pestilenz.}

In einem der Zimmer findet ihr zwei schweissnasse Kinder zugedeckt in einem Bett liegen. Eines der Kinder weint lautstark, das andere regt sich nicht.

\red{\textbf{Probe auf Medizin o.ä}: Den Kindern ist nicht mehr zu helfen, zumindest nicht mit euren Fähigkeiten.}

\brown{\textbf{Pestilenz}: Kümmern sich die Spieler weiter um die Kinder erhalten sie +1 Pestilenz.}

\subsubsection{Das Treffen mit Hagen}
\label{klopf}

\red{\textbf{Szene}:}

Plötzlich klopft es an der Türe. Ein blonder, kräftig gebauter staht auf der Straße und fragt die Gruppe wo Sigrun sei und was sie hier zu suchen hätten. Nach einer Erklärung stellt er sich als Hagen (\blue{\ref{Hagen}}) vor und läd sie zu sich nach Hause ein, er habe ihnen etwas spannendes zu erzählen! Folgt ihm die Gruppe gelangen sie nebenan zu Hagens Haus.

\textbf{Raumbeschreibung}: Hagens Haus ist aufgeräumt. Nirgendwo liegen Waren oder ähnliches herum und auch sonst ist es nur kärglich eingerichtet Einige Kerzen spenden ein wenig Helligkeit und werfen lange Schatten auf einen Holztisch in der Mitte des Hauses. Hagen fordert die Gruppe auf sich um den Tisch zu setzen.

\red{\textbf{Information für den Spielleiter}: Hat Hagen die Gruppe nicht zu sich eingeladen (falls die Gruppe ihn beispielsweise von selbst aufsucht), dann verlangt er im Gegenzug für Informationen einen Aufseherposten innerhalb der Hanse. Die Gruppe kann zustimmen oder ablehnen. Hauptsache sie kommen an die Infos.}

Hagen berichtet, dass er seit geraumer Zeit heimlich Waren abzweige. Er sei nicht stolz darauf, aber man müsse eben sehen, wo man bleibt. Jedenfalls komme er nachts am Nikolaifleet an diese Waren. Dort arbeite er. Als immer mehr seiner Kollegen krank wurden – das startete bereits vor vier Wochen, also vor allen anderen Ausbrüchen – wurde das sogar noch leichter. Eines Nachts jedenfalls schlich er sich wieder ins Lager, als ihm ein eigenartiger Mann begegnete. Er trug eine lange Maske, die beinahe wie der Schnabel eines Vogels aussah, ein weites Gewand und einen Stock bei sich. Außerdem roch es nach... Parfüm. Der Mann floh, als er Hagen sah, und ließ nichts außer einer Rattenfalle zurück. Diese war aber leer. Auf Nachfrage kann Hagen ihnen eine Wegbeschreibung zum Lagerhaus am Nikolaifleet geben. Unter keinen Umständen will er die Gruppe jedoch begleiten. Auch mit Gewalt oder Drohungen kann man ihn nicht überzeugen.

Ende des relevanten Hammerbrook-Plots.

Weiter mit:

Hafen (gehe zu \blue{\ref{Hafen}}) \\
Eeksdurf (gehe zu \blue{\ref{xd}}) \\
Die Kirche St. Petri (gehe zu \blue{\ref{Petri}}) \\
Der Nikolaifleet (gehe zu \blue{\ref{Fleet}}) \\

\subsection*{Kampf gegen Räuber}
\label{kampf2}

Handelt es sich um einen Überfall und die Räuber können sich unbemerkt anschleichen und einen Angriff ohne Verteidigungsmöglichkeit auswürfeln, diese verursacht zudem den doppelten Schaden eines normalen Angriffes.

\begin{center}
  \begin{tabular}{lc}
    \toprule
    Fähigkeit & Punkte \\
    \midrule
    Leben & 80 \\
    Fäuste & 70 \\
    Schaden & 15 \\
    Parieren & 5 \\
    \bottomrule
  \end{tabular}
\end{center}

Sinkt das Leben der Räuber unter 30 Punkte werden sie versuchen zu fliehen (Würfle 1W100, bei einer Augenzahl von 60 oder weniger gelingt den Räubern die Flucht).
Sind die Räuber besiegt so können die Spieler ein Dolch bei einem der beiden finden. Ansonsten findet sich nichts von Wert.

\subsection*{Didrichs Lagerhaus}
\label{Lagerhaus}

\subsubsection{Vor dem Lagerhaus}

\red{\textbf{Szene}}:

Das Lagerhaus, das Ruth euch beschrieben hatte, steht recht nahe der Stadtmauer, gerade vor den Toren der wohlhabenden Stadt. Es sieht hier noch erbärmlicher aus als sonst wo im Armenviertel. Durch die Ritzen zwischen der Holzverkleidung strahlt jedoch ein wenig Helligkeit, im Inneren scheint Licht zu brennen.

Bei genauerem Betrachten entdecken die Spieler ein Loch im Dach des Lagerhauses

\red{\textbf{Interaktionen}:}

Die Gruppe kann nun auf zwei Arten versuchen in das Lager einzudringen:

\begin{itemize}
  \item \textbf{Tür aufbrechen (Probe auf Körperkraft o.ä)}: Wenn sie einbrechen, tauchen zwei kräftige Männer auf, die sehr skeptisch sind. Ablenkung ist gefragt! Schafft die Gruppe nicht das Misstrauen der beiden zu vertreiben werden sie die Stadtwache holen oder gegen die Spieler kämpfen (gehe zu \blue{\ref{kampf2}})
  \item \textbf{Durch das Loch im Dach klettern (Probe auf Klettern, erschwert)}: Vom Dach müssen die Spieler ins Innere springen, bestehen sie dabei eine Probe auf Geschick o.ä nicht nehmen sie 10 Schaden.
\end{itemize}

\subsubsection{Im Inneren}

\textbf{Raumbeschreibung}: Im Lagerhaus ist es eiskalt. Allerdings brennt eine riesige Öllampe, die auf einem Tisch steht, auf dem allerlei Dinge liegen, unter anderem eine tote Ratte, eine Zitronenschale und etwas von einem roten Pulver, das einen scharfen Geruch verströmt (Chili).

Außerdem führen drei Türen zu weiteren Räumen. Bei allen sind jegliche Ritze und Schlitze säuberlich mit Lumpen und Lappen verstopft und verteert.

\begin{itemize}
  \item \textbf{Linke Türe}: Die linke Türe ist nur angelehnt und scheint offen zu stehen. Öffnen sie die Türe finden sie üppige Kornvorröte. Auf den Säcken steht \gqm{De groote Buur} geschrieben.
  \item \textbf{Mittlere Türe}: Die mittlere Türe ist geschlossen. Öffnen die Spieler diese erhalten sie \brown{+2 Pestilenz}. Hinter der Türe liegt ein Raum der bis auf einen großen Tisch in der Mitte leer ist. Auf dem Tisch liegt nackt ein Toter mit aufgeschnittenem Bauch. Auch in den Ecken liegen mehrere Tote, die meisten von ihnen von der Pest gezeichnet.
  \item \textbf{Rechte Türe}: Auch die rechte Türe ist verschlossen. Öffnen sie diese strömen Ratten heraus. Sie sind ausgehungert und aggresiv. Sie greifen die Gruppe an! (gehe zu \blue{\ref{kampf3}})
\end{itemize}

\subsubsection{Kampf gegen Ratten}
\label{kampf3}

Der Kampf gegen die Ratten läuft wie ein normaler Kampf gegen jeden anderen Gegener ab. Nun besteht der Gegner jedoch aus 10 Ratten mit jeweils 10 Lebenspunkten. Für Angriffe der Ratten würfle 1W10, das Ergebnis ist die Anzahl an Ratten, die die Spieler angreifen und pro Biss jeweils 4 Schaden austeilen. Werden die Ratten angegriffen so versuchen sie auszuweichen (Ausweichen: 10).

\brown{\textbf{Pestilenz}: Alle Spieler die gebissen werden erhalten einmalig +1 Pestilenz.}

Sind alle Ratten besiegt können die Spieler in den rechten Raum eintreten. Auch hier finden sie Getreidesäcke auf denen \gqm{De groote Buur} geschrieben steht.

Ende des relevanten Hammerbrook-Plots.

Weiter mit:

Hafen (gehe zu \blue{\ref{Hafen}}) \\
Eeksdurf (gehe zu \blue{\ref{xd}}) \\
Die Kirche St. Petri (gehe zu \blue{\ref{Petri}}) \\
Der Nikolaifleet (gehe zu \blue{\ref{Fleet}}) \\
