%!TEX root = ../main.tex

\red{\textbf{Szene}}:

Nach einem kurzen Fußmarsch steht ihr nun also inmitten des Hafens. Es herrscht geschäftiges Treiben. Allerlei Matrosen (\blue{\ref{Matrosen}}) und Hafenarbeiter (\blue{\ref{Hafenarbeiter}}) gehen ihren Aufgaben nach. An den Kais stehen einige Aufseher (\blue{\ref{Aufseher}}) und löschen die Ladungen der vor Anker liegenden Handelsschiffen.

\textbf{Ortsbeschreibung}: Allerhand Gesindel und unzählige Hafenarbeiter treiben sich herum. Prostituierte (\blue{\ref{Prostituierte}}) bieten ihre Dienste an, und Kinder betteln um ein wenig Brot. Hin und wieder sieht man Menschen, die sich vermummen oder gar ihr ganzes Gesicht hinter Tüchern verbergen.

\red{\textbf{Interaktionen}}:

Die Gruppe kann sich nun zum „Gelockten Hund“ aufmachen oder sich zunächst ein wenig umsehen. An den Docks treffen sie auf allerhand Charaktere, die ihnen etwas zum Hafen erzählen können.
Diese werden ihnen erzählen, dass immer mehr Arbeiter ausfallen, die mit der Lagerung von Lebensmitteln und anderen verderblichen Waren zu tun haben.

\subsection{Vor dem \gqm{Gelockte Hund}}
\label{vorhund}

\red{\textbf{Szene}}:

Ihr haltet vor einem kleinen Wirtshaus dessen besten Tage bereits weit in der Vergangenheit liegen. Es sieht - wie seine Kundschaft - ein wenig heruntergekommen aus. Vor der Türe des Gebäudes halten sich drei finstere Gestalten, bei denen es sich wohl um Seeräuber handelt, auf.

Ereignis: Schon vor der Tür wird die Gruppe unangenehm begrüßt. Die Seeräuber behaupten es würde Eintritt kosten, in den Laden zu kommen.

Es gibt nun verschiedene Möglichkeiten an den Seeräubern vorbei in das Gasthaus zu gelangen:

\begin{itemize}
  \item \red{\textbf{Probe auf Menschenkenntnis (erleichtert):} Dass der Eintritt etwas kostet, ist eine Lüge.} \\
  \item \textbf{\red{Kampf:}} Betrunkener Pirat \\
\begin{center}
  \begin{tabular}{lc}
    \toprule
    Fähigkeit & Punkte \\
    \midrule
    Leben & 80 \\
    Fäuste & 70 \\
    Schaden & 15 \\
    Parieren & 5 \\
    \bottomrule
  \end{tabular}
\end{center}

Werden die Piraten besiegt können die Spieler ein verziertes Kreuz aus Silber in der Tasche eines Piraten finden.

\red{\textbf{Probe auf christliche Kultur o.ä.}: Es ist ein Kreuz, wie es sonst nur Priester tragen würden.
Auf der Rückseite ist Apostel Petrus eingraviert.}
\end{itemize}

\subsection{Im \gqm{Gelockten Hund}}
\label{imhund}

\red{\textbf{Szene}}:

Der Eindruck, dass es sich hier um eine finstere Absteige handelt bestätigt sich, als ihr eintretet. Auch das Innere dementsprechend heruntergekommen aus. Der Großteil des Publikums ist Gesindel, das Fremden gegenüber nicht sonderlich wohlgesonnen sein dürfte.

\textbf{Raumbeschreibung}: Es herrscht ausgelassene Stimmung. In der Kaschemme selbst ist erst mal aber nichts besonders ungewöhnlich.

\red{\textbf{Interaktionen}}:

Die Charaktere können sich umhören, ob jemand Gorich kennt. Sollten sie nach ihm fragen wird niemand bis auf den Wirt Gert (\blue{\ref{Gert}}) ihnen antworten.

Gert: \gqm{\textit{Edle Herren, ich würde euch gerne Auskunft geben, doch seht was hier los ist! Heute morgen ist meine Wirtin nicht zur Arbeit erschienen. Auch meine Frau, die das Essen zubereitete ist letzte Woche der Pest zum Opfer gefallen. Wenn ihr mir... ein wenig unter die Arme greifen könntet? Danach werde ich euch freilich gerne helfen!}}

Entscheidet sich die Gruppe dem Wirt Gert zu helfen, dann muss sie ihn nun bei Aufgaben in der Kneipe unterstützen. Diese Aufgaben sind:

\begin{itemize}
  \item Eintopf kochen (Probe auf Kochen o.Ä)
  \item Gäste bedienen (Probe auf Menschenkenntnis (erleichtert))
  \item Einen Streit schlichten (Probe auf Beruhigen o.Ä)
  \item Ein wenig Musizieren (Probe auf passendes Talent)
\end{itemize}

Für einen Erfolg müssen mindestens zwei der Aufgaben erfolgreich bestanden werden. Sind mehr Aufgaben erledigt kann der Wirt je nach Ermessen des Spielleiters den Abenteurern einen Schilling für ihre Dienste geben.

Schafft die Gruppe es nicht zwei der vier Aufgaben zu bewältigen gibt Gorich sich nicht zu erkennen. Er kann nur noch mit Gewalt oder Tricks dazu gebracht werden, sich zu offenbaren. Dies ist aber stark erschwert.

\red{\textbf{Szene}}:

Ein großer, bärbeißiger Mann mit schmalem Gesicht und noch schmaleren Augen tritt an euch heran. Er stellt sich als Gorich (\blue{\ref{Gorich}}) vor. Er erzählt den Abenteurern er und seine Leute haben nichts mit der ganzen Sache zu tun. Er zeigt der Gruppe sogar, dass seine eigenen Kinder im Hinterzimmer liegen... krank. Sie hatten schon früh von der Seuche gehört und auch erfahren, dass es irgendwas mit Lebensmitteln zu tun haben könnte. Denn es waren zuallererst die Bauern und Karrenlenker krank geworden, die im Umland lebten. Also kaufte er all seine Nahrung nur noch aus Einfuhr. Das schien aber auch nichts zu helfen. Denn seine Frau ist bereits gestorben, und auch seinen Kindern gehe es immer schlechter. Ein gewisser Hagen habe es ihm verkauft. Dieser lebe in Hammerbrook, direkt vor den Toren der Stadt. Gekauft habe er die Güter direkt bei einem Lagerarbeiter. Er wisse, dass auch andere, die bei ihm gekauft haben, krank wurden.

Bevor die Abenteurer aufbrechen wird Gorich sie bitten seine Kinder und ihn nicht zu verraten. Das würde den Untergang seines Geschäfts bedeuten, und dann könne er sich erst recht nicht mehr um sie kümmern.

\red{\textbf{Interaktionen}}:

\red{\textbf{Probe auf Menschenkenntnis}: Gorich scheint die Wahrheit zu sagen.}

\brown{\textbf{Pestilenz}: Jeder, der sich den Kindern von Gorich nähert, erhält Pestilenz +2.}

\green{\textbf{Moral}: Was tut die Gruppe also mit Gorich und seinen Kinder?}

Ende des relevanten Hafen-Plots. Weiter mit:

Eeksdurf (gehe zu \blue{\ref{xd}}) \\
Hammerbrook (gehe zu \blue{\ref{arm}}) \\
Die Kirche St. Petri (gehe zu \blue{\ref{Petri}}) \\
Der Nikolaifleet (gehe zu \blue{\ref{Fleet}}) \\
