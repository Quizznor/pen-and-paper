\red{\textbf{Szene}}:

Erst seit wenigen Jahren wächst am Nikolaifleet ein gewaltiges Lagerzentrum für die Waren aus aller Welt. Der neue Reichtum trägt hier Früchte, und in direkter Nähe zu Tee, Gewürzen und Tulpen siedeln sich die betuchteren Bürger Hamburgs in prächtigen Villen an. Alles scheint makellos und vom Chaos der übrigen Stadt unberührt. Fast schon zu makellos. Weißer Schnee fällt auf frisch gepflasterte Straßen. Niemand ist zu sehen, obwohl hier reges Treiben herrschen sollte. Ohne große Schwierigkeiten macht ihr das Lagerhaus auffindig.


\subsection*{Grote Buur}
\label{"grote buur"}

\red{\textbf{Szene}}:

„Grote Buur“ steht in eisernen Lettern über dem Eingang eines mehrstöckigen Lagerhauses. Es besticht gegenüber den anderen Lagerhäusern durch sein Aussehen und scheint erst kürzlich weiß getüncht worden zu sein. Eine große Eichentür versperrt den Zugang ins Innere.

\red{\textbf{Interaktion}}:

Die Gruppe kann probieren die Türe des Lagerhauses zu öffnen, überraschenderweise ist diese nur angelehnt. Auch kompliziertere Methoden (Klettern, Tür eintreten, etc.) führen zum Ziel, erregen aber unter Umständen die Aufmerksamkeit der Anwohner...

\textbf{Raumbeschreibung}: Es handelt sich wohl um ein typisches Lagerhaus. Auf der Rückseite befinden sich die Tore, die zum Löschen der Handelsboote genutzt werden. Momentan sind diese jedoch verschlossen. Das Lagerhaus hat gleich mehrere Etagen, und es gibt diverse Kräne und Seilzüge, um Waren auf den Ebenen zu transportieren. Bei näherer Betrachtung erkennt man, dass sich hier einige Ratten niedergelassen haben. Bei genauerer Betrachtung entdecken die Spieler Korn, Tücher und andere Handelswaren, allerdings nicht so viele, wie ein Lagerhaus von dieser Größe vermuten lassen würde.

\red{\textbf{Ereignis}}: Das Lagerhaus scheint alles in allem verlassen zu sein. Doch dann hören die Spieler ein Rumpeln, das aus den Untergeschossen stammt. Dort angekommen finden Sie eine Luke die mit einem Metallschloss befestigt ist.

\red{\textbf{Probe auf Kraft oder Schlösser knacken o.ä.}: Das Schloss lässt sich von der Luke entfernen, die sich jetzt problemlos öffnen lässt.}

Darunter sitzt ein gefesselter Mann. Er hat überall am Körper Wunden, die nicht nur von der Pest verursacht wurden.



Interaktionen:

Begegnung mit dem Mann:

Dem Mann fällt das Sprechen sehr schwer, aber er kann noch „Sinnfeld“ und „gegenüber“ stammeln.

Die Gruppe kann ihn retten, erhält dann aber Pestilenz +3.

Andernfalls stirbt er vor ihren Augen.
Körper untersuchen: Die Wunden an seinem Körper sind unter anderem Rattenbisse. Außerdem ist er übersäht mit Flöhen.

Das Handelsregister
\red{\textbf{Szene}}:

Das Handelsregister ist der Ort, an dem alles Wissen der Stadt um jeden Geschäftsmann und seine Machenschaften zusammenkommt.

Ortsbeschreibung: Das Gebäude ist modern, fein verziert und kann sich sehen lassen.

Die Gruppe kann hierhin gehen, um Auskunft über den Eigentümer des „Grote Buur“ und Didrich von Sinnfeld einzuholen.

Beim Eintreten sehen sie eine Beamtin, die offensichtlich das Sagen hat, und ein paar um sie herum wuselnde Mitarbeiter.

Interaktionen:

Auf Anfrage kann die Gruppe hier ein Dokument erhalten, das belegt, dass Didrich von Sinnfeld der Eigentümer des „Grote Buur“ ist und gegenüber wohnt.

Das Haus gegenüber
\red{\textbf{Szene}}:

Gleich gegenüber des „Grote Buur“ steht ein ansehnliches Bürgerhaus.

Ortsbeschreibung: Es ist gepflegt, sauber und von außen in hervorragendem Zustand.

Ereignis: Die Gruppe kann an die Tür und anklopfen. Als sie klopfen öffnet ihnen...

Wurf: Wer öffnet?
Ein Mann (0 bis 49)
Eine Frau (50 bis 99)
Option 1 – Ein Mann
\red{\textbf{Szene}}:

Ereignis: Die Tür öffnet sich einen Spalt, und eine Stimme ertönt.

Man sieht aber niemanden. „Wer da?“

Die Person reagiert erschrocken auf die Stimmen unserer Charaktere und stürmt nach hinten ins Haus davon. Verfolgen sie den Mann, sehen sie diesen um eine Ecke rennen und hören dann ein gewaltiges Klirren und Krachen.

Sie kommen in einen düsteren Raum voller medizinischer Instrumente und anderer abstruser Gerätschaften. Überall im Raum sind Notizen und Tagebücher verteilt. Auf dem Tisch in der Mitte des Raumes liegt ein kleiner Junge. Er ist tot und gezeichnet von der Pest. Sein Brustkorb und sein Kopf sind geöffnet. Seine Organe liegen auf Tellern neben seinem Körper. Sein Gesicht ist in Panik und Schmerz erstarrt. Vor dem Tisch auf dem Boden liegt der Körper eines bewusstlosen Mannes, der eben noch vor ihnen geflohen war. Er hat das Blut des Jungen an den Händen.

Interaktionen:

Dursuchen oder Lesen der Notizen und Tagebücher:

Die Notizen im Raum verraten, dass der Mann an etwas rund um die Pest geforscht hat und das Ganze etwas mit Ratten zu tun gehabt haben dürfte. Jedenfalls tauchen diese überall auf. Auch Beschreibungen zum Bau von Geräten (Masken und Kleidung), die eine Ansteckung verhindern, gibt es.

Probe auf Medizin:

Außerdem finden sie unzählige Zeichnungen sezierter Körper sowie Berichte von Obduktionen lebender Patienten und Krankheitsverläufen bei Gefangenen. Harter Toback. Aber auch wichtige Erkenntnisse. Diesen Unterlagen zufolge hat die Krankheit ihren Ursprung in der Rattenpopulation, und hält man sich von Kranken und Ratten fern, so kann man eine Ansteckung verhindern.
Moral:

Was macht die Gruppe mit Didrich und den Informationen?

Option 2 – Eine Frau
\red{\textbf{Szene}}:

Ereignis: Klopfen sie an, öffnet die Magd Traudel.

Gespräch mit Traudel: Im Haus wohne ein Herr von Sinnfeld, Didrich von Sinnfeld. Ja, er sei zu Hause. Die Gruppe möge eintreten!

Die Magd führt sie in den Salon. Sie werden gebeten, zu warten.

Ereignis: Nach einer Weile hören sie einen Schrei. Dann rennt das Hausmädchen an ihnen vorbei zur Tür hinaus.

Geht die Gruppe dem nach, finden sie Didrich in seinem Arbeitszimmer. Dieser beugt sich gerade über den leblosen Körper eines kleinen Jungen und ist im Begriff, ihn zu sezieren.

Gespräch mit Didrich:

„Ein Kleingeist, die gute Traudel. Viel zu leicht zu erschrecken. Und nun?! Habe ich nichts als Ärger am Hals.

Denn die werten Herrschaften, so nehme ich an, stehen meinem Treiben hier ebenso wenig wohlgesonnen gegenüber wie andere Vertreter ihrer Stände.

Sehe ich das richtig?”

Didrich schildert ihnen, dass er an der Krankheit forsche. Er sei kurz vor einem Durchbruch. Es wisse nun mit Sicherheit, dass es etwas mit Ratten zu tun habe. Nur der genaue Ablauf der Ansteckung sei ihm noch schleierhaft. Aber ein Fehlen von Ratten in einer Stadt ginge unmittelbar mit einem Abhandensein der Seuche einher, obwohl durch Kontakt zu Kranken ebenfalls eine Seuche zustande kommen könne. Seine Forschungen seien nicht immer ganz lupenrein gewesen. Er brauchte tote Körper, später lebendige, das gebe er zu. Zum Glück gäbe es genug Kranke direkt hier am Fleet. Gleich gegenüber fielen die Arbeiter reihenweise um. Aber dann blieben sie zu Haus. Ein Besuch auf Hammerbrook sei also unumgänglich geworden. Dann noch einer. Und noch einer. Jedenfalls könne man solche Forschungen in einer kleingeistigen Stadt wie Hamburg nicht ohne Weiteres betreiben. Daher die Heimlichtuerei. Was nun? Soll er seine Forschungen fortsetzen?

Ihm ist durchaus bewusst, weshalb die Gruppe hier ist.


Moral:

Was tun sie mit Didrich und seinen Forschungen?

Ende des relevanten Nikolaifleet-Plots.

Weiter mit:

Hafen (gehe zu \blue{\ref{Hafen}}) \\
Eeksdurf (gehe zu \blue{\ref{xd}}) \\
Hammerbrook (gehe zu \blue{\ref{arm}}) \\
Die Kirche St. Petri (gehe zu \blue{\ref{Petri}}) \\
