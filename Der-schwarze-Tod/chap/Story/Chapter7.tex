\red{\textbf{Szene}}:

Erst seit wenigen Jahren wächst am Nikolaifleet ein gewaltiges Lagerzentrum für die Waren aus aller Welt. Der neue Reichtum trägt hier Früchte, und in direkter Nähe zu Tee, Gewürzen und Tulpen siedeln sich die betuchteren Bürger Hamburgs in prächtigen Villen an. Alles scheint makellos und vom Chaos der übrigen Stadt unberührt. Fast schon zu makellos. Weißer Schnee fällt auf frisch gepflasterte Straßen. Niemand ist zu sehen, obwohl hier reges Treiben herrschen sollte.

\subsection*{Der Handelsregister}
\label{handelsregister}

\red{\textbf{Szene}}:
Das Handelsregister ist ein fein herausgeputztes Gebäude. Die Fassade ist gesäumt von in Metall gefassten Glasfenstern. Eine geöffnete Türe führt ins Innere des Gebäudes, in die hin und wieder Kaufläute ein und aus gehen.

Nach dem Eintreten führt ein kurzer Gang in einen Raum, in dessen Mitte an einem imposanten Schreibtisch eine Beamtin sitzt.
Sie reagiert forsch auf Fragen der Gruppe, kann jedoch mit einem passenden Talent überredet werden Informationen über die Häuslichkeiten am Nikolaifleet preiszugeben.

\red{\textbf{Probe auf passendes Talent:} Die Beamtin erzählt, dass die Geschäfte und das Leben am Nikolaifleet seit dem Ausbruch des schwarzen Todes stark zurückgegangen sind. Nur ein Herr Didrich von Sinnfeld lebe noch in einem Haus gleich am Fleet.}

Das Handelsregister ist ein fein herausgeputztes Gebäude. Die Fassade ist gesäumt von in Metall gefassten Glasfenstern. Eine geöffnete Türe führt ins Innere des Gebäudes. Im Inneren sitzt an einem
massiven Eichentisch eine Beamtin. Die Gruppe kann sie um Auskunft fragen über den Eigentümer des „Grote Buur“ und Didrich von Sinnfeld einzuholen.

\subsection*{Grote Buur}
\label{"grote buur"}

\red{\textbf{Szene}}:

„Grote Buur“ steht in eisernen Lettern über dem Eingang eines mehrstöckigen Lagerhauses. Es besticht gegenüber den anderen Lagerhäusern durch sein Aussehen und scheint erst kürzlich weiß getüncht worden zu sein. Eine große Eichentür versperrt den Blick ins Innere.

\red{\textbf{Interaktion}}:

Die Gruppe kann probieren die Türe des Lagerhauses zu öffnen, überraschenderweise ist diese nur angelehnt. Auch kompliziertere Methoden (Klettern, Tür eintreten, etc.) führen zum Ziel, erregen aber unter Umständen die Aufmerksamkeit der Anwohner...

\textbf{Raumbeschreibung}: Es handelt sich wohl um ein typisches Lagerhaus. Auf der Rückseite befinden sich die Tore, die zum Löschen der Handelsboote genutzt werden. Momentan sind diese jedoch verschlossen. Das Lagerhaus hat gleich mehrere Etagen, und es gibt diverse Kräne und Seilzüge, um Waren auf den Ebenen zu transportieren. Bei näherer Betrachtung erkennt man, dass sich hier einige Ratten niedergelassen haben. Bei genauerer Betrachtung entdecken die Spieler Korn, Tücher und andere Handelswaren, allerdings nicht so viele, wie ein Lagerhaus von dieser Größe vermuten lassen würde.

\red{\textbf{Ereignis}}: Das Lagerhaus scheint alles in allem verlassen zu sein. Doch dann hören die Spieler ein Rumpeln, das aus den Untergeschossen stammt. Dort angekommen finden Sie eine Luke die mit einem Metallschloss befestigt ist.

\red{\textbf{Probe auf Kraft oder Schlösser knacken o.ä.}: Das Schloss lässt sich von der Luke entfernen, die sich jetzt problemlos öffnen lässt.}

Unter der Luke sitzt ein gefesselter Mann. Er hat überall am Körper Wunden, die nicht nur von der Pest zu stammen scheinen. Ihm fällt das Sprechen sehr schwer, aber es sind schwach die Worte „Sinnfeld“ und „gegenüber“ zu hören, bevor der Mann in Ohnmacht fällt.

\brown{\textbf{Pestilenz:} Die Gruppe kann probieren ihn zu retten bzw. untersuchen, erhält dann aber +1 Pestilenz}.

Eine nähere Inspektion legt offenbar, dass die Wunden an seinem Körper unter anderem von Rattenbissen stammen, außerdem entdeckt die Gruppe einige Schnittwunden entlang der schwarzen Beulen. Einige Flöhe krabbeln auf dem Boden herum. Abgesehen von dem Mann gibt es sonst nichts zu entdecken in dem Lagerhaus.

\subsection*{Sinnfelds Haus}
\label{sinnfelds haus}
\red{\textbf{Szene}}:

Dem "Groote Buur" gegenüber steht ein ansehliches Herrenhaus, dem es in Architektur und Größe ähnelt. Im zweiten Stock befinden sich große Fenster in eingefassten Metallrahmen. Das Haus wirkt alles in allem gepflegt.

\red{\textbf{Ereignis}: Die Gruppe kann an die Tür und anklopfen. Als sie klopfen wird ihnen nach kurzer Zeit geöffnet.}

\subsubsection{Eine Frau öffnet}
\label{frau öffnet}
\red{\textbf{Ereignis}: Klopfen sie an, öffnet die Magd Traudel.}

Ein Gespräch mit Traudel legt offen, dass im Haus ein Herr von Sinnfeld wohne, Didrich von Sinnfeld. Ja, er sei zu Hause. Die Gruppe möge eintreten! Die Magd führt sie in den Salon, wo Sie gebeten werden, zu warten.

\red{\textbf{Ereignis}: Nach einer Weile hören sie einen Schrei. Dann rennt das Hausmädchen an ihnen vorbei zur Tür hinaus.}

Geht die Gruppe dem nach, finden sie Didrich in seinem Arbeitszimmer. Dieser beugt sich gerade über den leblosen Körper eines kleinen Jungen und ist im Begriff, ihn zu sezieren.
Als er aufblickt und die Gruppe ihn entdeckt entwickelt sich ein Gespräch:

Didrich: \gqm{\textit{Ein Kleingeist, die gute Traudel. Viel zu leicht zu erschrecken. Und nun?! Habe ich nichts als Ärger am Hals. Denn die werten Herrschaften, so nehme ich an,
stehen meinem Treiben hier ebenso wenig wohlgesonnen gegenüber wie andere Vertreter ihrer Stände. Sehe ich das richtig?}}

Didrich schildert ihnen, dass er an der Krankheit forsche. Er sei kurz vor einem Durchbruch. Er wisse nun mit Sicherheit, dass es etwas mit Ratten zu tun habe. Nur der genaue Ablauf der Ansteckung sei ihm noch schleierhaft. Aber ein Fehlen von Ratten in einer Stadt ginge unmittelbar mit einem Abhandensein der Seuche einher, obwohl durch Kontakt zu Kranken ebenfalls eine Seuche zustande kommen könne. Seine Forschungen seien nicht immer ganz lupenrein gewesen. Er brauchte tote Körper, später lebendige, das gebe er zu. Zum Glück gäbe es genug Kranke direkt hier am Fleet. Gleich gegenüber fielen die Arbeiter reihenweise um. Aber dann blieben sie zu Haus. Ein Besuch auf Hammerbrook sei also unumgänglich geworden. Dann noch einer. Und noch einer. Jedenfalls könne man solche Forschungen in einer kleingeistigen Stadt wie Hamburg nicht ohne Weiteres betreiben. Daher die Heimlichtuerei. Was nun? Soll er seine Forschungen fortsetzen?

\violet{\textbf{Moral}: Was tun sie mit Didrich und seinen Forschungen?}

Ende des relevanten Nikolaifleet-Plots.

Weiter mit:

Hafen (gehe zu \blue{\ref{Hafen}}) \\
Eeksdurf (gehe zu \blue{\ref{xd}}) \\
Hammerbrook (gehe zu \blue{\ref{arm}}) \\
Die Kirche St. Petri (gehe zu \blue{\ref{Petri}}) \\
