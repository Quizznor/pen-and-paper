%!TEX root = ../main.tex

\red{\textbf{Szene}}:

Die Kirche St. Petri steht ganz im Zeichen des neuen Hamburger Wohlstands. Noch immer wird gebaut, aber man sieht ihr schon jetzt an, dass sie eines der zukünftigen Wahrzeichen der Handelsmetropole sein wird. Ihr steht vor der gewaltigen Pforte die ins Innere des Baus führt. Viele Menschen halten sich hier auf, die meisten tragen Lumpenkleidung und scheinen hier auf Beistand und Sicherheit vor der Krankheit und Armut zu suchen.

\red{\textbf{Ereignis}}: Sobald jemand an die Tür klopft werden sie von Helfern begrüßt und hereingelassen.

\red{\textbf{Probe auf Wahrnehmung, christliche Kultur o.ä.}: Die Helfer tragen Mönchsgewänder, scheinen aber keinem (den Spielern bekannten Orden) zu gehören.}

\red{\textbf{Information für den Spielleiter}: Auch der tote Priester in Richtung Eeksdurf trug solche Gewänder, bei ihm handelt es sich um Pater Salus (\blue{\ref{Salus}}.}

Fragen die Spieler nach einem der Leiter der Kirche wird ihnen der Helfer von Pater Salus berichten, und dass dieser aufgebrochen sei, um Gerüchten auf den Grund zu gehen.
Nun ist von schwarzer Magie die Rede, böse Mächte seien in der Stadt am Werk. Die Gruppe ist aber herzlich eingeladen zu bleiben, sofern sie gesund sind und dies beschwören.

\subsection{Im Hauptraum}
\label{hauptraum}

\red{\textbf{Szene}}:

Ihr werdet in das große Schiff der Kirche geführt. Hier sitzen überall Menschen mit besorgten Gesichtern herum. Andere wiederum blicken voller Zuversicht und lauschen in Grüppchen gebannt Gesängen und Predigten, welche Mönche überall im Raum zum Besten geben. Außerdem gehen verdächtig viele Klingelbeutel herum. Es führen einige Türen aus dem Hauptraum. Alle sind momentan geschlossen. Man kann erkennen, dass im oberen Drittel des Hauptraumes noch gebaut wird, und die Wände teils mit Gerüsten gesäumt sind.

\red{\textbf{Interaktionen}}:

Die Gruppe kann nun mit den Menschen in der Halle interagieren.

\begin{itemize}
  \item Den Mönchen zuhören \\
  Gesellen sich die Spieler zu den Gruppen die um die Mönche stehen können sie hören:

  Mönch: \gqm{\textit{Ja! Das jüngste Gericht ist über uns gekommen. Mit eurem wolllüstigen Leben habt ihr euren Teil dazu beigetragen die Pest in die Mauern Hamburgs zu tragen! Doch Gott, geheiligt sei sein Name, ist ein gnädiger Herr. Selbst den sündigsten Menschen vergibt er ihre Taten wenn sie ihn in sein Herz einlassen. Der Weg in Gottes Paradies ist gezeichnet durch Reue und Frömmigkeit. Tragt also Rechenschaft für eure Buße und entledigt den gerechten Anteil eures Lohnes an die Kirche aufdass der Herr über euer Seelenheil nach dem Tode wacht!...}}

  In diesem Moment geht ein Klingelbeutel umher. Sollten die Spieler kein Geld spenden werden sie nach kurzer Empörung gebeten die Kirche zu verlassen. Eine Probe auf ein passendes Talent kann dies verhindern.

  \item Mit den Leuten hier sprechen \\
  Einige der Leute sitzen alleine in der großen Halle. Spricht die Gruppe mit ihnen erzählen die Menschen, dass sie mit Angehörigen hier waren, diese wurden aber durch die Türen in private Gemächer gebeten, wohl, weil sie besonders fromm waren.
\end{itemize}

Versucht die Gruppe durch die Türen zu gelangen kommt ein Mönch zu ihnen und weist sie darauf hin, dass dies die privaten Gemächer der Priester seien.

\red{\textbf{Probe auf Menschenkenntnis}: Der Mönch lügt.}

Jedenfalls will man sie hier nicht hindurchlassen. Die Gruppe kann natürlich versuchen sich Zugang zu verschaffen.

\subsection{Im Hinterzimmer}
\label{hinten}

\red{\textbf{Szene}}:

Schaffen sie es irgendwie hier hinein, geht es erstmal leicht abwärts durch ein paar Gänge. Fenster, sofern sie in den Gängen vorhanden sind sind mit Stoff verhangen, es ist dunkel und ein modriger Geruch bedrängt die Spieler sobald sie die Türe hinter sich schließen.

\brown{\textbf{Pestilenz}: Alle erhalten +1 Pestilenz.}

Nach einer Weile öffnet sich der Gang zu einer Art Gewölbe. Es stink erbärmlich nach Blut, Schweiß und Exkrementen. Hinter einem massiven Eisengitter liegen unzählige Menschen wie in einer Zelle. Manche der Eingesperrten sind schwer von der Pest gezeichnet, andere sehen relativ gesund aus. Auch ettliche Tote liegen zwischen den Kranken. Vereinzelt durchbricht ein heftiges Husten die Stille des Ortes.

\brown{\textbf{Pestilenz}: Gehen sie näher an die Menschen heran, erhalten sie Pestilenz +2.}

Verwickeln sie die Kranke in Gespräche erfahren sie, dass alle hier überzeugt davon sind, dass sie das hier als Sünder verdient haben.

\textbf{Ortsbeschreibung}: Am hinteren Ende des Raumes gibt es eine Tür. Davor am Gitter steht eine junge Frau, Gundel (\blue{\ref{}}). Sie ist bereits von der Pest gezeichnet, winkt die Gruppe aber energisch zu sich.

Gundel bittet die Gruppe, ihr dabei zu helfen zu entkommen. Sie müsse zu diesem Arzt, diesem Didrich. Der wisse, was zu tun ist. Da ist sie sich sicher! Gundel arbeitet am Nikolaifleet. Dort kocht sie für die Männer. Nach und nach wurden sie alle krank.

Plötzlich tauchte der Arzt auf, Didrich von Sinnfeld. Er versprach ihr, wenn sie ihm ein paar Ratten aus der Küche fange, würde er sie fürstlich bezahlen. Sie lehnte ab. Das schien ihr doch sehr merkwürdig, und die Bezahlung sei nicht üppig gewesen. Als er aber sagte, dass er an etwas arbeite, um den Schwarzen Tod zu besiegen, lenkte sie ein. Sie gab ihm die Ratten. Doch dann wurden immer mehr Leute krank. Sie bekam es mit der Angst zu tun und kam hierher. Doch es war zu spät. Man erleichterte sie um ihr Geld und sperrte sie dann hier zum Sterben hinein.

\violet{\textbf{Moral}: Lassen sie Gundel und die anderen raus und gefährden damit die Stadt? Oder gehen sie und überlassen die Kranken ihrem unumgänglichen Schicksal?}

Öffnen die Spieler die zweite Türe stehen sie vor einer nach oben führenden Treppe. Am oberen Ende ist eine weitere Türe, unter deren Türschwelle helles Tageslicht hindurchdringt. Sie führt die Gruppe wieder ins Freie. Hier treffen sie noch einmal auf den Totensammler Hanno (\blue{\ref{Hanno}}), dieser schaut sie vorwurfsvoll an und zieht dann wortlos weiter um Leichen zu sammeln.


Ende des relevanten St.Petri-Plots.

Weiter mit:

Hafen (gehe zu \blue{\ref{Hafen}}) \\
Eeksdurf (gehe zu \blue{\ref{xd}}) \\
Hammerbrook (gehe zu \blue{\ref{arm}}) \\
Der Nikolaifleet (gehe zu \blue{\ref{Fleet}}) \\
