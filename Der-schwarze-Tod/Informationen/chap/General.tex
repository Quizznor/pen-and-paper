%!TEX root = ../main.tex

\textbf{Wo?}            -  mittelalterliches Hamburg \\
\textbf{Wann?}          -  1350 n. Chr. \\
\textbf{Spielerzahl?}   -  3 - 5 \\
\textbf{Schwierigkeit?} -  einfach - mittel \\
\textbf{Spieldauer?}    -  3-4 Stunden

\section{Anmerkungen für Spielleiter}

Ein wichtiges Element dieses Abenteuers ist die Tatsache, dass Spielcharaktere im Verlauf der Handlung mit der Pest in Berührung treten. Sind sie dabei fahrlässig kann es passieren, dass sie selbst an der Pest erkranken. Hierzu wurde die \gqm{Pestilenz} implementiert, die den Fortschritt einer Pestinfektion anzeigt. Die Spieler wissen davon jedoch nichts, da dieser Wert nur dem Spielleiter bekannt sein sollte, jedoch äußert sich die Pestilenz in Symptomen die zufällig auftreten und wie folgt verschlüsselt sind:

\begin{tabular}{cc}
  \toprule
  Anzahl & Auswirkung \\
  \midrule
  1-2 & Dem Charakter wird von Zeit zu Zeit schwindelig \\
  3-4 & Plötzlich wird dem Charakter ganz heiß/kalt \\
  5-6 & Der Charakter klagt über Kopfschmerzen \\
  7-8 & Der Charakter bekommt hohes Fieber, wird von Gliederschmerzen befallen \\
  9-10 & In den Leisten, unter den Armen, am Hals bilden sich schwarze, eitrige Beulen \\
  >10 & Der Charakter ist streng bettlägerig, nach einiger Zeit stirbt er \\
  \bottomrule
\end{tabular}


Die Formatierung ist nicht zufällig gewählt. Im Verlauf des Abenteuers werden spezielle Informationen wie folgt verdeutlicht:

\begin{itemize}
  \item \textit{Kursive Texte}:
  Alles, was \textit{kursiv} geschrieben ist, kann wörtlich vorgetragen werden. Dabei handelt es sich meistens um die Einleitungen der einzelnen Abschnitten oder um wörtliche Rede in Gesprächen.

  \item \textbf{Raumbeschreibungen/Ortsbeschreibungen}:
  Diese Beschreibungen verweisen auf die Einrichtung eines Raums oder die Beschaffenheit eines Ortes und sind \textbf{fett} gedruckt. Raumbeschreibungen beschreiben meistens alles, was innerhalb eines Gebäudes zu sehen ist, Ortsbeschreibungen hingegen beschreiben, was draußen ist.

  \item \red{\textbf{Szenen und Interaktionen}}:
  Die Abschnitte sind in Szenen und Interaktionen unterteilt. Damit es einfacher ist, dorthin zu navigieren, sind diese \red{rot} markiert. Szenen geben eine Handlung vor, die sich den Spielern offenbart, wenn sie sich in einem Abschnitt befinden. Interaktionen ermöglichen optionale Handlungsstränge, die den Spielern entgehen können, wenn sie nicht die entsprechenden Aktionen durchführen oder sich für die entsprechende Option entscheiden.

  \item \green{\textbf{Moral}}:
  Alle Stellen, an denen die Spieler mit moralischen Fragen konfrontiert werden, sind \green{grün} gekennzeichnet. Diese markieren Entscheidungen, die sich auf den Ausgang der Geschichte auswirken können.

  \item \purple{\textbf{Pestilenz}}:
  Alle Stellen, an denen die Spieler in Kontakt mit der Pest kommen und gegebenenfalls Pestilenz anhäufen können, sind \purple{lila} markiert.

  \item \blue{\textbf{Referenzen}}:
  Zur einfachen Navigation sind an relevanten Stellen Referenzen eingebaut, mithilfe derer sich einfach und schnell zwischen verschiedenen Textpassagen hin und her springen lässt. Solche Referenzen lassen sich anklichen und sind \blue{blau} markiert.
\end{itemize}
