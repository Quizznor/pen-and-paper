%!TEX root = ../main.tex

\section{Allgemeines}
\textbf{Wo?}            -  ISV Venture Star (interstellares Raumschiff) \\
\textbf{Wann?}          -  2100 n. Chr. \\
\textbf{Spielerzahl?}   -  4 Spieler \\
\textbf{Schwierigkeit?} -  Mittel \\
\textbf{Spieldauer?}    -  3-6 Stunden \\

\section{Vorwort für Spielleiter}

\section{Fraktionen}

\subsection{Blue Ridge Corp.}



\subsection{The Reborn Order}



\newpage

\section{Charaktere}

\subsection{}

\subsection{}

\newpage

\section{Prolog}

Man sagt, es hat knapp 16 000 Jahre gedauert, bis der anatomisch moderne Mensch das Rad erfunden hat.
Noch 5 000 Jahre vergingen, bis er das erste Automobil erfand, die Pferdekutschen als primäres Fortbewegungsmittel ablösten.
Kaum weitere 100 Jahre vergingen, und er landete auf dem Mond.

Wir schreiben das Jahr 2100.
Neue Entdeckungen in der Medizin treiben die durchschnittliche Lebenserwartung eines gesunden Menschen weit über hundert Jahre.
Viele wissen von einst scheinbar unheilbare Krankheiten wie Krebs überhaupt nur noch, wenn sie zufällig im Extranet danach googlen.
Dank des technischen Fortschritts erfreut man sich einer hervorragenden Lebensqualität.
Roboter und künstliche Intelligenzen werden überall dort eingesetzt,
wo während der ersten industriellen Revolution noch Männer und Frauen für einen Hungerlohn schweißtreibende Arbeiten verrichteten.

Natürlich ist dieser bisher ungeahnte Lebensstandard nicht kostenlos. Künstlich angelegte Monokulturen haben den südamerikanischen Regenwald in kleine Nationalparks zurückgedrängt.
Riesige Müllhalden prägen die Landschaften um die Millionenstädte. Obwohl zur Jahrtausendwende noch überwiegend zentralafrikanische Länder zu den "Verlierern" der Globalisierung gezählt wurden
ist mittlerweile offensichtlich, dass die Leidtragenden diejenigen sind, die sich nicht an diese rasant ändernde Welt anpassen können.
Wissenschaftler sprechen von dem sechsten großen Massenaussterben, das den Beginn des Anthropozäns mit einem Paukenschlag verkündet.

Während exotische Tierarten um ihr Überleben kämpfen sieht sich die Menschheit anderen Problemen gegenübergestellt. Überpopulation. 


\section{Einführung}
