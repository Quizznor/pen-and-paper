\thispagestyle{fancy-info}
\section*{Generelle Informationen}

\begin{itemize}
  \item \textbf{\textit{Kursive Texte}} \\
  Alles, was \textbf{\textit{kursiv}} geschrieben ist, soll wörtlich vorgetragen werden. Dabei handelt es sich in der Regel
  um wörtliche Rede in Gesprächen, in der sich unter Umständen wichtige Informationen verstecken können, oder eine
  stimmungsvolle Einleitung zu verschiedenen Abschnitten.

  \say{Beispiel für ein Zitat}{Jemand, der das gesagt hat}

  \item \textcolor{RoyalRed}{\textbf{Interaktionen}} \\
  Pen and Paper lebt von der \textcolor{RoyalRed}{Interaktion} der Spieler mit ihrer Umwelt. Wann immer eine Probe auf ein Talent
  möglich ist ist diese Information \textcolor{RoyalRed}{dunkelrot} markiert. Am Beispiel unten sieht man dabei auf welches Talent
  gewürfelt werden kann (Lesen), und ob dieser Wurf beeinflusst ist. Die Zahl in Klammern ($-10$) wird dabei immer zum Talentwert
  des Spielers dazuaddiert, ein negativer Wert bedeutet also, dass der Wurf erleichtert ist.

  \begin{probe}{Lesen}{-10}
  Bei erfolgreicher Probe geschieht, was hier steht.
  \end{probe}
  \vspace{-5pt}

  \item \textcolor{RoyalBlue}{\textbf{Referenzen}} \\
  Zur einfachen Navigation sind an relevanten Stellen \textcolor{RoyalBlue}{Referenzen} eingebaut, mithilfe derer sich einfach und
  schnell zwischen verschiedenen Textpassagen hin und her springen lässt. Solche Referenzen lassen sich (in der pdf
  Version) anklichen und sind \textcolor{RoyalBlue}{blau} markiert.

  \vspace{-5pt}
  \begin{refbox}{\chaptername$\;$\thechapter}
      Gehe zur Story auf Seite \textcolor{RoyalBlue}{\pageref{adv-start}}. \\
      Gehe zu den Charakteren auf Seite \textcolor{RoyalBlue}{\pageref{char-start}}.
  \end{refbox}
  \vspace{-5pt}

  \item \textcolor{violet}{\textbf{Charaktere}} \\
  Während der Handlung treffen die Spieler auf verschiedene \textcolor{violet}{\textbf{Charaktere}}. Referenzen zu Informationen über
  die jeweiligen Charaktere sind \textcolor{violet}{\textbf{violett}} hinterlegt. In den jeweiligen Abschnitten befinden sich
  Hintergrundinformationen wie Motivationen und Verhaltensweisen, etc. die der Person Leben einhauchen sollen. Hier ein
  Beispiel:

  \begin{centering}
    \noindent\makebox[\textwidth][c]{%
    \begin{minipage}{80mm}
      \charakter{Beispielcharakter} sieht aus dem Fenster.
    \end{minipage}}
  \end{centering}

\end{itemize}
