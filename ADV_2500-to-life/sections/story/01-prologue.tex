\chapter[Prolog]{Eine kurze Geschichte der Menschheit}

Komprimiert man die Geschichte der Erde - insgesamt knapp 5 Millarden Jahre - auf nur einen einzigen Tag, so entwickeln sich Mikroben und Einzeller gegen
18 Uhr. Die Gattung Homo tritt etwa um 23:59 Uhr auf den Tagesplan. Der anatomisch moderne Mensch etwickelt sich vier Sekunden vor Mitternacht.

In diesen umgerechnet immerhin dreihunderttausend Jahren kann der Homo sapiens sapiens auf eine erstaunliche Entwicklung zurückblicken. Die Entdeckung der
Töpferei $20\,000$ v.Chr. ist hier nur einer der ersten Schritte. Weitere fünfzehntausend Jahre dauerte es bis zur Erfindung des Rades. Mit dem wachsenden
Wissen um die Mathematik und Naturwissenschaften machte sich der Mensch die Natur immer weiter zu Untertan, bis die Gebrüder Wright mit einem Flugzeug vom
Boden abhoben. Kaum 70 Jahre später landete der Mensch anno 1969 auf dem Mond.

Wir schreiben das Jahr 2500 n.Chr.

Fortschritte in der Medizin strecken die natürliche Lebensspanne eines jeden, der es sich leisten kann bis deutlich über 100 Jahre. Über 30 Millarden
Menschen leben mittlerweile auf der Erde. Die genaue Zahl kennt eigentlich niemand. Höchstens die weltumspannenden Ultrakonzerne haben noch einen groben
Überblick darüber, was vor sich geht. Fest steht jedenfalls, die Erde ist überbevölkert. Zwar gibt es florierende Kolonien auf dem Mond und Mars, doch auch
diese stoßen mittlerweile an ihre Populationsgrenzen. Die einzige Lösung scheint das Erschließen neuer Sonnensysteme.

\todo{continue introducing MegaCorp}
