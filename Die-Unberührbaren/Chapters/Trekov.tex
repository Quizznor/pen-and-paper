%!TEX root = ../main.tex

Die Spieler können sich nun in Trekov frei bewegen. Sie können die Autowracks als auch die Häuser durchsuchen. Beim Durchsuchen des ersten Autos finden sie einen Bolzenschneider. Beim weiteren durchsuchen wird ihnen auffallen, dass das komplette Dorf schon mehr als einmal geplündert wurde. Beim Durchsuchen der Häuser Würfel 1W10. Zeigt der Würfel eine 1-6 ist das Haus geplündert worden.
\\Zeigt der Würfel eine 4-10 kommen sie an das Haus der Familie Belkin. Nachdem das Event abgeschlossen ist finden die Spieler nur noch bereits geplünderte Häuser. Bewegen sich die Spieler weiter in Richtung des Gebirges kommen sie am Bunker an.

Verlassen die Spieler Trekov in Richtung Westen lies weiter bei Novabarsk. Entscheiden sie sich nach Norden zu gehen treffen sie auf Juggernauts.

\section{Haus}

Ihr steht vor einem in Efeuranken gehüllten, einstöckigem Haus. Zwischen dem hochgewachsenen Gras im Vorgarten macht ihr einen Pfad aus, der hinter das Haus führt. Die Eingangstür ist aus massiven Holz gefertigt, in deren Mitte ein eiserner Türklopfer prankt, sie scheint noch intakt zu sein. Rechts neben der Türe könnt ihr auf einem verrostetem Namensschild \TODO{Namen einfügen} entziffern. Auf der anderen Seite der Türe befindet sich ein Briefkasten

\textbf{Im Schlafzimmer des Hauses befindet sich ein Zombie. Der Zombie kann mit einem -20 erschwerten Wurf auf Wahrnehmung o.ä. ausgemacht werden. Der Spieler hört dann ein leises Wimmern. Der Schwierigkeitsgrad kann auf +/- 0 gesenkt werden wenn die Spieler einmal um das Haus herum laufen und sich nach Gefahren umsehen bzw. hören}

\begin{figure}[t]
	\begin{center}
		\includegraphics[scale=0.65]{./fig/Trekov_Haus.png}
		\caption{Grundriss des Hauses}
	\end{center}
\end{figure}

\subsection{Vorgarten}

Die Eingangstür ist abgeschlossen. Sollten die Spieler den Türklopfer betätigen oder sich gewaltsam Zutritt zu dem Haus verschaffen wird der Zombie auf die Spieler aufmerksam und wird versuchen diese anzugreifen nachdem sie das Haus betreten. Durch den Schlitz im Briefkasten können die Spieler einen Zettel erkennen, um ihn allerdings an sich zu nehmen benötigen die Spieler entweder einen Schlüssel um den Briefkasten zu öffnen oder sie versuchen es über eine Fähigkeit wie Fingerfertig, Schlangenmensch o.ä. (mit einem Malus von -30) den Zettel aus dem Briefkasten zu holen. Die Spieler können den Briefkasten auch zerstören allerdings wird der Untote dann auf sie aufmerksam.

\newpage
Auf dem Zettel steht in krakeliger Schrift:

\begin{itshape}
	Liebste Mutter,
	\\mittlerweile ist die Kommunikation mit Moskau abgebrochen.
	\\Ihr müsst Trekov verlassen, Ihr seit dort nicht sicher.
	\\Sie sagen im Norden befindet sich ein Stützpunkt von Überlebenden.
	\\Ich flehe euch an, bitte nimm Jakub mit und verlass Trekov.

	\\Natascha

\end{itshape}

\subsection{Garten}

Folgen die Spieler dem Pfad der hinter das Haus führt kommen sie in den Garten. Dort können sie eine Schaufel, alte Gummihandschuhe und eine Gartenhacke (würfel bei jedem Gebrauch 1w6. Ist das Ergebnis \gqm{6} zerbricht die Hacke, SBW 25) finden. Der Hintereingang zum Haus scheint offen zu sein, da die Tür nur angelehnt ist. Durch die Hintertür gelangen die Spieler in das Wohnzimmer.

\subsection{Wohnzimmer}

Das Wohnzimmer ist ärmlich eingerichtet. Hinter einem Esstisch erstrecken sich Sofas und Sessel die einem alten Röhrenfernseher zugewandt sind. In der äußeren Ecke des Raumes befindet sich ein geschmückter Weihnachtsbaum, dessen verweklte Nadeln noch immer auf dem Boden liegen.
\\Auf dem Esszimmer ist befindet sich ein Buch (1984, George Orwell) sowie ein zerknitterter Zettel.
\\Auf dem Zettel steht in einer fein säuberlichen Handschrift:

\begin{itshape}
Liebste Mutter,
\\ich hoffe es geht dir gut. Die Nachrichten sind furchtbar. Ich bin zum Bunker berufen worden. Irgendein biologischer Notfall. Ich hoffe diese Krankheit breitet sich nicht auch noch hier aus. Es wurden wirklich viele Wissenschaftler gerufen. Biochemiker, Genetikexperten und Ärzte. Ich hoffe sie erlauben es, dass wir unsere Familien nachholen dürfen.

Gib Jakub von mir einen Kuss.
\\Haltet durch! Ich liebe euch.
\\Natascha
\end{itshape}

Geradeaus führt ein Flur zur vorderen Haustür, links daneben  befindet sich zwei Türen, die zur Abstellkammer sowie dem Bad führen. Direkt rechts neben dem Esstisch ist eine Küchenzeile in die Wand eingelassen.

\subsection{Küche}

Die Küche wirkt ungewöhnlich gut in Schuss bis auf die Tatsache, dass neben der Spüle ein toter Hund in seinem Korb liegt. Auf dem Herd stehen leere Töpfe, in den Schränken befinden sich einige vergammelte Lebensmittel.
\\Durchsuchen die Spieler den Schrank neben dem Kühlschrank finden sie ein Küchenmesser (SBW 20). Sind sie hierbei auffällig laut wird der Untote versuchen an sie heranzuschleichen.

\subsection{Flur}

Im Flur sind weitere Familien- und Kinderportraits aufgehängt. Vor der Eingangstüre sind mehrere Paare an Frauenschuhen abgestellt. Auf einem Abstelltisch neben der Eingangstüre liegt ein kleiner Schlüssel mit dem sich der Brifkasten öffnen lässt.
Unterhalten sich die Abenteurer auf dem Flur wird der Zombie versuchen sich an die Abeenteurer anzuschleichen

\subsection{Badezimmer}

Die Badezimmer sind mit blauen Fliesen ausgelegt. Abgesehen von Hygieneprodukten lassen sich hier Handtücher finden. Das Wasser in den Wasserhähnen bzw. der Klospülung und der Dusche funktionieren nicht.

\subsection{Abstellkammer}

In der Abstellkammer liegen unter einem Fenster zwei extrem verweste Leichen. Es handelt sich um ein Kind, dass in den Armen einer älteren Frau liegt. Untersuchen die Spieler die Leichen näher erkennen sie, dass beide Einschusslöcher im Kopf vorweisen.
\\Will ein Spieler die Leichen auf Wertsachen untersuchen, so findet er einen Revolver mit 4 Schuss, erhält aufgrund der verwerflichen Tat jedoch einen Malus von -10 auf die restlichen Würfe des Tages.

\subsection{Schlafzimmer}

Sobald die Spieler die Tür öffnen führt der Zombie einen Angriff aus, töten sie den Untoten, so können sie in seinen zerfetzten Jeanshosen ein Taschenmesser sowie Zigaretten und ein Feuerzeug finden.
Unter dem Ehebett befindet sich des weiteren eine Taschenlampe ohne Batterien.

Der Untote hat 70 Lebenspunkte. Weitere Stats finden sich in der unteren Tabelle

\begin{center}
  \input{./tab/zombie_trekov_house.tex}
\end{center}

\subsection{Kinderzimmer}

Ihr betretet einen mit blauem Teppichboden ausgelegten Raum. Die Wände sind in schlichtem Weiß gehalten. Auffälligerweise steht der Raum völlig leer. Bei genauerer Betrachtung ist erkennbar,
dass im Teppichboden Dellen sind, die von Möbeln herrühren könnten. Außerdem weisen Wände auffällige Verfärbungen auf. Fast so, als hätten dort vor nicht allzu langer Zeit Bilder gehangen.


\section{Marktplatz}

Gehen die Spieler weiter in das Dorf hinein gelangen sie an eine offene, gepflasterte Fläche, auf der verstreut Unrat herumliegt.
In der Mitte des Platzes steht ein mittlerweile vertrockneter Brunnen. Davor befindet sich an einer Anschlagtafel eine Karte von Trekov und Umgebung.
Besitzt einer der Spieler eine besondere Auffassungsgabe oder vergleichbare Talente (Oder Spieler würfeln eine um +10 erleichterte Probe) kann er erkennen, dass jemand einen Pfeil und \gqm{HILFE} nahe der Straße zum Uralgebirge in die Tafel geritzt hat.

\begin{figure}[ht]
	\begin{center}
		\includegraphics[scale=0.4]{./fig/Trekov_Karte.png}
		\caption{Karte von Trekov und Umgebung}
	\end{center}
\end{figure}

\section{Kirche}

Im Inneren der Kirche ist es dunkel. Vereinzelt fällt Licht durch zerbrochene Ornamente. Die wenigsten Kirchenbänke stehen noch an ihren angedachten Plätzen, die meisten sind umgekippt oder gleich völlig zerstört. Im Raum verteilt liegen hin und wieder reglose Körper von Untoten und Menschen gleichermaßen.
Eine Probe auf Sinneswahrnehmung o.ä. kann offenbaren, dass aus einer dunklen Ecke ein gequältes Röcheln hervordringt.

Hierbei handelt es sich um Petrov, einem Schwerverletztem, der sie um Wasser bitten wird. Folgt die Gruppe seiner Aufforderung wird er ihnen von einer missglückten Suche nach den Sentinel erzählen.
\\Seine Gruppe habe die ganze Umgebung um Trekov abgesucht, doch keinen geheimen Unterschlupf gefunden, in dem sich Wissenschaftler verschanzt hätten. Der einzige Ort an dem sie noch nicht gesucht hätten sei das Gelände im Osten (zum Ural hin).

Verwehrt die Gruppe das Trinken wird Petrov ihnen nicht von seiner Expedition erzählen, kann jedoch mit einem Wurf auf überreden umgestimmt werden. Bevor die Gruppe weitergeht wird Petrov sie bitten ihn zu töten, damit er nicht als Zombie verendet.
Durchsuchen die Spieler daraufhin sein Hab und Gut werden sie einige Bandagen und Schmerzmittel in einem Rucksack finden.

\subsection{Friedhof}
Ironischerweise ist auf dem Friedhof die Abwesenheit des Menschen am deutlichsten. Hohes Gras bedeckt die meisten Gräber, auf denen noch immer welke Pflanzen in Tontöpfen stehen. Auffallend ist, dass neben den älteren Gräbern die meist über einen Grabstein samt Namen verfügen auch viele Erdlöcher mit schlichten Holzkreuzen drapiert sind.

Ansonsten findet sich hier - abgesehen von evtl. Grabkerzen - nichts an Wert.

\section{nördliches Ortsende}

Kurz hinter dem nördlichen Ortsausgang begegnen die Spieler auf dem Weg nach Kiringrad drei schwer bewaffneten Männern, die einen Außenposten der Juggernaut halten.
Sobald die Spieler sich nähern werden die Juggernaut diese mit gezogenen Waffen zum Umkehren bewegen wollen.
