%!TEX root = ../adventure.tex

\section*{Allgemeines}

\begin{tabular*}{\textwidth}{@{\extracolsep{\fill}} lr}

  \textbf{Wo?} & \place \\
  \textbf{Wann?} & \storytime \\
  \textbf{Spielerzahl?} & \playercount \\
  \textbf{Schwierigkeit?} & \difficulty \\
  \textbf{Spieldauer?} & \duration \\

\end{tabular*}

\section*{Anmerkungen für Spielleiter}

Die Formatierung ist nicht zufällig gewählt. Im Verlauf des Abenteuers werden spezielle Informationen wie folgt
verdeutlicht:

\begin{itemize}
  \item \textit{Kursive Texte}:
  Alles, was \textit{kursiv} geschrieben ist, kann wörtlich vorgetragen werden. Dabei handelt es sich meistens um die Einleitungen der einzelnen Abschnitten oder um wörtliche Rede in Gesprächen.

  \item \textbf{Raumbeschreibungen/Ortsbeschreibungen}:
  Diese Beschreibungen verweisen auf die Einrichtung eines Raums oder die Beschaffenheit eines Ortes und sind \textbf{fett} gedruckt. Raumbeschreibungen beschreiben meistens alles, was innerhalb eines Gebäudes zu sehen ist, Ortsbeschreibungen hingegen beschreiben, was draußen ist.

  \item \red{\textbf{Szenen und Interaktionen}}:
  Die Abschnitte sind in Szenen und Interaktionen unterteilt. Damit es einfacher ist, dorthin zu navigieren, sind diese \red{rot} markiert. Szenen geben eine Handlung vor, die sich den Spielern offenbart, wenn sie sich in einem Abschnitt befinden. Interaktionen ermöglichen optionale Handlungsstränge, die den Spielern entgehen können, wenn sie nicht die entsprechenden Aktionen durchführen oder sich für die entsprechende Option entscheiden.

  \item \blue{\textbf{Referenzen}}:
  Zur einfachen Navigation sind an relevanten Stellen Referenzen eingebaut, mithilfe derer sich einfach und schnell zwischen verschiedenen Textpassagen hin und her springen lässt. Solche Referenzen lassen sich anklichen und sind \blue{blau} markiert.
\end{itemize}
