%!TEX root = ../adventure.tex

\chapter{Prolog: Das Gespräch mit dem Rat}
\label{adventure}

\begin{advquote}
\large Wir schreiben das Jahr 1350. Hamburg wächst dank des erstarkenden Seehandels
stetig und die Hanse trägt ihren Teil dazu bei. Täglich gehen am Rheinhafen Schiffe
aus aller Herren Länder vor Anker, während vom Rathaus an der Troßtbrücke der Rat die
Geschicke der Stadt lenkt. Es ist eine Zeit des Aufbruchs, aber auch eine Zeit der
Angst. Denn neben Piraten und anderen finsteren Gestalten, die zunehmend in den
Gassen der Stadt umherstreifen, greift etwas noch viel gefährlicheres um sich. Die
Leute hatten bereits von einem „Schwarzen Tod" gehört, der in anderen Städten gewütet
haben soll, und nun scheint es so, als sei die Plage auch in Hamburgs Gassen
eingekehrt. Der düstere Geruch des Todes zieht durch die Docks und Armenviertel,
während der hohe Rat in der Hansestadt darüber entscheidet, was zu tun ist.

Und was auch immer das sein mag, es muss schnell geschehen.
\end{advquote}

\section*{Das Wartezimmer}
\label{sec:wartezimmer}

Ihr seit also aus ebendiesem Grund vom Hamburger Rat zu sich gerufen worden. Was
genau von euch erwartet wird um dem schwarzen Tod vorzubeugen wird sich alsbald
zeigen. So begibt es sich, dass ihr geduldig vor einer gewaltigen Holztür im Ratshaus
an der Troßtbrücke sitzt. Hinter der Tür tagt im Moment der Hamburger Rat, zu den
man euch nun in wenigen Augenblicken rufen wird. Ob ihr euch bereits kennt dürft ihr
entscheiden und ausspielen wie ihr möchtet. Noch besteht jedoch ein wenig Zeit sich
zu unterhalten und umzuschauen. Was tut ihr also?

\begin{place-box}{Wartezimmer}

Ihr befindet euch in einer Art Warteraum, der für damalige Verhältnisse sehr üppig
eingerichtet ist. Man erkennt deutlich, dass Hamburgs Reichtum hier sein Übriges
getan hat. An den Holzwänden hängen mehrere Bilder von großen Hamburger 
Persönlichkeiten. Außerdem stehen kleine Teller mit allerlei Obst und Brot bereit, 
und auch Getränke werden angeboten. In den Ecken des Raumes verteilt stehen 
Ratsdiener die für die Erfrischungen sorgen. An der Wand hängt, und das ist für euch
vermutlich am interessantesten, eine Karte der Stadt (\then \page{fig:map})

\end{place-box}

\section*{Das Gespräch mit dem Rat}
\label{sec:ratsgespräch}

Als Ihr euch also unterhaltet öffnet sich plötzlich die Holztür und ein junger
Ratsdiener bittet euch vor den Rat zu treten.

\begin{place-box}{Hamburger Rat}

Ihr tretet in einen großen Raum ein, der von einer U-förmigen Tischreihe dominiert
wird an dem 18 Männer in erhöhter Position sitzen. Auch wenn der Raum sonst nur
spärlich eingerichtet ist, ist offensichtlich dass sich hier sonst niemand aufhält,
der wenig Geld hat. Aus den Wänden sind kunstvoll Löwenköpfe und andere Muster
herausgearbeitet. Eine Wand ist von hohen Fenstern gesäumt, die den Raum hell
erleuchten. Im Hintergrund huschen Ratsdiener mit Papieren umher oder gehen anderen
Aufgaben nach.

\end{place-box}

\begin{info-box}

Die 18 Männer setzen sich aus neun Rechtskundigen, sieben Kaufleuten und zwei
Vertretern der Kirche zusammen. Je nach Gesprächsverlauf können diese zu Wort kommen.

\end{info-box}

Nachdem ihr von den Ratsherren nähergewunken werdet spricht der Vorsitzende.

\begin{say-box}{Ratsvorsitzender}

Der Hamburger Rat dankt euch für euer Erscheinen. Bevor wir uns jedoch mit den euch
betrauten Aufträgen befassen werden, sagt, was wisst ihr über die Pestilenz?

\end{say-box}

\begin{probe-box}{Medizin}{10}

Menschen die von der Pestilenz befallen sind leiden unter Schwindelgefühlen und
Schüttelfrost. Auch klagen manche über starke Kopfschmerzen und erbrechen sich
häufig. Ein hohes Fieber sowie schwarze Pestbeulen an den Lymphknoten treten im
Verlaufe der Krankheit auf und sind für den Patienten meist ein sicheres
Todeszeichen.

\end{probe-box}

\begin{say-box}{Ratsvorsitzender}

Nun ist dem Rat zu Ohren gekommen, dass auch vor den Mauern Hamburgs die Pestilenz
ihr Unwesen treibt. Diese Krankheit muss so kurz vor dem christlichsten aller
Feiertage unter allen Umständen aus der Stadt verdrängt werden. Zu diesem Zwecke
werdet ihr in einen Beirat gerufen. Der Rat fordert euch auf in den nächsten vier
Tagen die Fälle von Pestilenz zu untersuchen und dem Rat am fünften Tag eine
Empfehlung auszusprechen, wie die Pestilenz zu bewältigen ist. Werdet ihr eure
Aufgaben gewissenhaft erfüllen soll es euer Schaden nicht sein.
Außerdem werdet ihr während eurer Mitgliedschaft im Beirat in einer Kapelle nahe
der St. Michaelis Kirche untergebracht werden, die fortan als eure
Operationszentrale dienen wird, alle weiteren Informationen wird man dort für
euch bereitlegen.

\end{say-box}
