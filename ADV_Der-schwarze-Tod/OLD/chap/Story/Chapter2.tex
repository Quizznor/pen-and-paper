%!TEX root = ../main.tex

\red{\textbf{Szene}}:

Nachdem ihr aus dem Rat entlassen wurdet macht ihr euch also auf den Weg zum Sitz des Beirats. In der unmittelbaren Umgebung des Ratshauses spürt man den Aufstieg Hamburgs als Handelsmetropole am deutlichsten. Ihr schlendert durch breite Straßen die von hohen Häusern gesäumt werden. Dienstboten eilen über den Pflasterstein und auch sonst herrscht geschäftiges treiben, als ihr unweit der St. Michaelis Kirche vor eine Kapelle tretet, die euch der Rat als eure Operationszentrale genannt hat

\textbf{Raumbeschreibung}: Als ihr an der kleinen Kapelle ankommt, erkennt ihr, dass es sich dabei um ein durchaus anschauliches Gebäude handelt, dass erst kürzlich einen neuen Anstrich mit weißer Farbe erhalten hat. Im Inneren stehen allerhand Tische und Stühle herum. Außerdem gibt es Schlafmöglichkeiten und so ziemlich alles, was man zum Leben so gebrauchen kann. Selbst ausgewählte Speisen stehen bereits zum Verzehr bereit.

\textbf{Ereignis}: Ein Gespräch der Spieler wird plötzlich von einem Klopfen unterbrochen ...

\begin{tcolorbox}
  Wurf: Wer oder was unterbricht das Gespräch unserer Gruppe? \\
  1 bis 33: Das Militär (gehe zu \blue{\ref{militär}}) \\
  34 bis 66: Der Tod (gehe zu \blue{\ref{tot}}) \\
  67 bis 99: Ein Kind (gehe zu \blue{\ref{kind}}) \\
  Bei 100: Würfle erneut.
\end{tcolorbox}

\subsection*{Der militärische Besuch}
\label{militär}

\red{\textbf{Szene}}:

Durch eine Luke in der Tür könnt ihr einen älteren Mann erspähen. Er ist in eine Mantel des Militärs gehüllt, an dem zahlreiche Orden hängen. Seine breite Nase ist von einer Narbe entstellt. Generell macht eher einen sehr befehlsgewohnten und grobschlächtigen Eindruck. Außerdem ist er in Begleitung zweier Soldaten.

Fremder Mann:
\gqm{\textit{Im Namen Karl des IV, öffnet eure Türe!}}

Nach dem der Mann eintritt stellt er sich den Spielern als General Brügge (\blue{\ref{Brügge}}) vor. Er berichtet ihnen in vehementem Ton, dass die ganze Krise ein Werk der Dithmarscher sei. Diese hätten sich jahrelang an Hamburgs Handelsschiffen gütlich getan. Nun, da es einen Vertrag gibt, der das verhindert, versuchen einige von ihnen die Stadt zu schwächen, um davon zu profitieren oder sie gar ganz an sich zu reißen. Die Dithmarscher operieren von ihrem Versteck aus, das sich in einer Hafenkaschemme namens „Beim Gelockten Hund“ befinden soll. Ein gewisser Gorich (\blue{\ref{Gorich}}) leite das Ganze. Dort sollten sie mit ihren Recherchen beginnen.


\red{\textbf{Interaktionen}}:

\red{\textbf{Probe auf Menschenkenntnis}: General Brügge scheint die Wahrheit zu sagen, zumindest glaubt er das. Seine Ansichten sind allerdings durch den langen Kampf gegen Piraten alles andere als neutral.}

Gehe weiter zu \blue{\ref{weiter}}.

\subsection*{Der Tod klopft an}
\label{tot}

\red{\textbf{Szene}}:

Vor der Türe steht ein buckliger alter Mann, der in einer Hand eine Öllaterne hält. Hinter ihm steht ein Handkarren, auf dem mehrere Leichen aufgebahrt sind.

Fremder Mann:
\gqm{\textit{Habt ihr Tote im Hause?}}

Es handelt sich um Hanno - den Totensammler (\blue{\ref{Hanno}}). Er meint im Armenviertel sei es am Schlimmsten. Die Leichen könne er kaum mehr entsorgen. Gegen Bestechung verrät er, dass er jemandem Leichen verkaufe, die an der Pest gestorben sind und derer er sich ohnehin nirgends entledigen könne. Dazu müsse er sie allerdings recht weit fortbringen, nämlich in einen kleinen Ort namens Eeksdurf vor den Toren der Stadt, auch dort habe die Pest besonders schlimm gewütet. Er hinterlege die leblosen Körper in einem Lagerhaus, wo bereits seine Bezahlung auf ihn warte. Die Absprache habe er dereinst mit einer jungen rothaarigen Frau getroffen, die ihn angesprochen hatte als er in Eeksdurf Tote abgeholt hätte.

Gehe weiter zu \blue{\ref{weiter}}.

\subsection*{Eine verlorene Seele}
\label{kind}

\red{\textbf{Szene}}:

Eine junge Frau (\blue{\ref{MutterKind}}) klopft energisch an die Türe. Sie ist in Lumpen gehüllt und das bisschen Haut, das sich in der Dunkelheit erahnen lässt ist schwer von Krankheit gezeichnet. Ein abgemagertes Kind klammert sich an ihre Hand.

Die Mutter erzählt ihnen, dass sie aus dem Armenviertel Hammerbrook kämen und auf der Suche nach der St. Petri-Kirche seien. Sie soll ein Zufluchtsort für gesunde und sündenfreie Menschen sein. Niemand werde dort krank! Für sie sei es zu spät, hustet die Frau, aber ihr kleines Kind, das sei noch zu retten. Sie wissen das alles von einem Mann, der im Armenviertel nach den Leuten sehe. Er werde nicht krank, egal was er tue... Er habe sie losgeschickt. Sie wüssten gern den Weg.

Die Gruppe kann eine Beschreibung von Didrich von Sinnfeld (\blue{\ref{Didrich}}) erhalten. Außerdem geben ihnen die beiden auf Nachfrage den Tipp, einmal beim Lumpensammler im Armenviertel vorbeizuschauen.

Gehe weiter zu \red{\blue{\ref{weiter}}}.

\subsection*{Die Entscheidung}
\label{weiter}

\textbf{Ab hier können die Spieler frei entscheiden, wohin sie gehen wollen! Nach Ablauf der Frist von vier Tagen müssen sie erneut beim Hamburger Rat vorsprechen. Bis dahin sollten sie sich auf eine Handlungsempfehlung festgelegt haben!}

Weiter mit:

Hafen (gehe zu \blue{\ref{Hafen}}) \\
Eeksdurf (gehe zu \blue{\ref{xd}}) \\
Hammerbrook (gehe zu \blue{\ref{arm}}) \\
Die Kirche St. Petri (gehe zu \blue{\ref{Petri}}) \\
Der Nikolaifleet (gehe zu \blue{\ref{Fleet}}) \\
