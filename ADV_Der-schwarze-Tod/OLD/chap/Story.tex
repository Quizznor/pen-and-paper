%!TEX root = ../main.tex

\chapter{Story}

\section{\textbf{Prolog}}
%!TEX root = ../main.tex

Wir schreiben das Jahr 1350. Hamburg wächst dank des erstarkenden Seehandels stetig und die Hanse trägt ihren Teil dazu bei. Täglich gehen am Rheinhafen Schiffe aus aller Herren Länder vor Anker, während vom Rathaus an der Troßtbrücke der Rat die Geschicke der Stadt lenkt. Es ist eine Zeit des Aufbruchs, aber auch eine Zeit der Angst. Denn neben Piraten und anderen finsteren Gestalten, die zunehmend in den Gassen der Stadt umherstreifen, greift etwas noch viel gefährlicheres um sich. Die Leute hatten bereits von einem „Schwarzen Tod" gehört, der binnen weniger Tage einen gesunden Mann seiner Lebenskraft zu berauben vermag. Nun scheint es so, als sei die Plage auch in die Wohnungen von Hamburg eingedrungen. Der düstere Geruch des Todes zieht durch die Docks und Armenviertel, während der Rat darüber entscheidet, was zu tun ist.

Und was auch immer das sein mag, es muss schnell geschehen.


\section{\textbf{* Das Gespräch mit dem Rat}}
%!TEX root = ../main.tex

\red{\textbf{Szene}}:

Ihr sitzt geduldig vor einer gewaltigen Holztür. Gleich wird man euch vor den Hamburger Rat rufen.

Diese Versammlung lenkt seit einigen Jahren Hamburgs Geschicke. Achtzehn Männer, davon neun Rechtskundige, sieben Kaufleute und zwei Vertreter der Kirche, entscheiden, was mit Hamburgs neuem Reichtum geschehen soll und wie man die Stadt in eine noch goldenere Zukunft lenken könnte.

Allerdings müssen sie sich auch um weniger Erfreuliches kümmern: Konflikte, Kriminalität und neuerdings auch Krankheit.

Raumbeschreibung: Die Gruppe sitzt in einer Art Warteraum, der für damalige Verhältnisse sehr üppig eingerichtet ist. Es stehen kleine exotische Leckereien bereit, und auch Getränke werden angeboten.

Interaktionen:

Die Gruppe hat noch etwas Zeit sich zu unterhalten und etwas umzusehen, bevor sie vor den Hamburger Rat gerufen werden.

Umsehen: An der Wand hängt eine Karte der Stadt, welche die Charaktere sich ansehen können.


Das Gespräch mit dem Rat
\red{\textbf{Szene}}:

Gespräch mit dem Rat: Der Hamburger Rat ruft sie zu sich und berichtet ihnen von Fällen der Pest in der Stadt. Weihnachten steht vor der Tür, und eine Panik, die die Leute in ihren Häusern einsperrt, ist das Letzte, was die Stadt in dieser Zeit gebrauchen kann. Aus diesem Grund beruft der Rat einen Beirat aus Experten ein, der der Krankheit auf den Grund gehen soll: die Charaktere unserer Spieler! Der Beirat soll sofort mit Nachforschungen beginnen. Eine Kapelle unweit der Kirche St. Petri wird zum Sitz des Beirates bestimmt. Dort habe man alle eventuell wichtigen Unterlagen hinterlegt. Sie sollen sich, betonen die Räte, gefälligst schleunigst auf den Weg machen. In vier Tagen erwarte der Rat ihre Ergebnisse und will unmittelbar Maßnahmen beschließen!

Alles, was man bisher weiß, ist:

Die Krankheit schlägt in den Armenvierteln der Stadt besonders schlimm zu.
Die wohlhabenderen Gegenden bleiben bisher weitestgehend verschont. Das solle auch so bleiben.
Im Sitz des Beirats
\red{\textbf{Szene}}:

Als ihr am Sitz des Beirats ankommt, erkennt ihr, dass es sich dabei um ein durchaus anschauliches Gebäude handelt. Im Inneren stehen allerhand Tische und Stühle herum. Ausserdem gibt es Schlafmöglichkeiten und so ziemlich alles, was man zum Leben so gebrauchen kann. Selbst ausgewählte Speisen stehen bereits zum Verzehr bereit.

Raumbeschreibung: Der Sitz des Beirats ist eine kleine Kapelle, die zum Anlass des Aufenthalts der Gruppe frisch mit Lebensmitteln, Decken und Lampen aufgestockt wurde.

Ereignis: Die Gruppe kann sich erst mal unterhalten. Ihr Gespräch wird allerdings von einem Klopfen unterbrochen ...

Wurf: Wer oder was unterbricht das Gespräch unserer Gruppe?
Das Militär (1 bis 33)
Der Tod (34 bis 66)
Ein Kind (67 bis 99)
Bei 100: würfle erneut.
Option 1 – Der militärische Besuch
\red{\textbf{Szene}}:

Ereignis: Während die Gruppe noch debattiert, wo man ansetzen solle, klopft es plötzlich an der großen Holztür.

General zur Brügge steht davor und bittet um Einlass.

Gespräch mit Brügge: Dieser berichtet ihnen in vehementem Ton, dass die ganze Krise ein Werk der Dithmarscher sei. Diese hätten sich jahrelang an Hamburgs Handelsschiffen gütlich getan. Nun, da es einen Vertrag gibt, der das verhindert, versuchen einige von ihnen die Stadt zu schwächen, um davon zu profitieren oder sie gar ganz an sich zu reißen. Die Dithmarscher operieren von ihrem Versteck aus, das sich in einer Hafenkaschemme namens „Beim Gelockten Hund“ befinden soll. Ein gewisser Gorich leite das Ganze. Dort sollten sie mit ihren Recherchen beginnen.

Interaktionen:

Probe auf Menschenkenntnis:

General zur Brügge sagt schon die Wahrheit, aber seine Perspektive könnte durchaus verzerrt beziehungsweise einseitig sein.
Die Charaktere können sich aber sicher sein, dass er nicht lügt, zumindest seiner Auffassung nach nicht.

Ab hier können die Spieler frei entscheiden, wohin sie gehen wollen! Nach Ablauf der Frist von vier Tagen müssen sie beim Hamburger Rat vorsprechen. Bis dahin müssen sie sich auf eine Handlungsempfehlung festgelegt haben!


Option 2 – Der Tod
\red{\textbf{Szene}}:

Ereignis: Während sich die Gruppe noch unterhält, hören die Charaktere plötzlich ein lautes Klopfen an der Tür.

Davor steht der Totensammler Hanno. Er fragt, ob es Tote gäbe, die abzuholen seien, und ob im Haus bereits die Pest wüte.

Gespräch mit Hanno: Im Armenviertel sei es am Schlimmsten. Die Leichen könne er kaum mehr entsorgen. Man müsse kreativ werden.

Gegen Bestechung verrät er, dass er jemandem Leichen verkaufe. Dazu müsse er sie allerdings recht weit fortbringen, nämlich in einen kleinen Ort namens Eeksdurf vor den Toren der Stadt. Dort hinterlege er die leblosen Körper in einem Lagerhaus, wo bereits seine Bezahlung auf ihn warte. Die Absprache habe er dereinst mit einer jungen rothaarigen Frau getroffen.

Sie habe ihn angesprochen, nachdem sie ihn beim Abholen von Leichen im Dorf sah...


Ab hier können die Spieler frei entscheiden, wohin sie gehen wollen! Nach Ablauf der Frist von vier Tagen müssen sie beim Hamburger Rat vorsprechen. Bis dahin müssen sie sich auf eine Handlungsempfehlung festgelegt haben!


Option 3 – Ein Kind
\red{\textbf{Szene}}:

Ereignis: Es klopft plötzlich an der Tür und davor steht ein Kind zusammen mit seiner stark vermummten Mutter.

Gespräch mit der Mutter und dem Kind: Sie kämen aus dem Armenviertel Hammerbrook und seien auf der Suche nach der St. Petri-Kirche. Sie soll ein Zufluchtsort für gesunde und sündenfreie Menschen sein. Niemand werde dort krank! Für sie sei es zu spät, hustet die Frau, aber ihr kleines Kind, das sei noch zu retten. Sie wissen das alles von einem Mann, der im Armenviertel nach den Leuten sehe. Er werde nicht krank, egal was er tue ... Er habe sie losgeschickt. Sie wüssten gern den Weg.

Die Gruppe kann eine Beschreibung von Didrich von Sinnfeld erhalten. Außerdem geben ihnen die beiden auf Nachfrage den Tipp, einmal beim Lumpensammler im Armenviertel vorbeizuschauen.


Ab hier können die Spieler frei entscheiden, wohin sie gehen wollen! Nach Ablauf der Frist von vier Tagen müssen sie beim Hamburger Rat vorsprechen. Bis dahin müssen sie sich auf eine Handlungsempfehlung festgelegt haben!


Hafen
Einleitung des Hafens
Der Hafen steht seit Hamburgs Beitritt zur Hanse vor etwa 30 Jahren im Mittelpunkt der städtischen Wirtschaft. Hier wächst der neue Wohlstand heran. Unzählige Waren und Handelsgüter werden Tag ein Tag aus umgeschlagen, und mit ihnen kommen immer mehr Fremde in die Stadt. Eine ganz eigene Welt entsteht. Mit eigenen Regeln, die man erst lernen muss ...

Die Ankunft
\red{\textbf{Szene}}:

Die Gruppe steht nun inmitten des pulsierenden Hafens.

Ortsbeschreibung: Allerhand Gesindel und unzählige Hafenarbeiter treiben sich herum. Prostituierte bieten ihre Dienste an, und Kinder betteln um ein wenig Brot. Hin und wieder sieht man Menschen, die sich vermummen oder gar ihr ganzes Gesicht hinter Tüchern verbergen.

Interaktionen:

Die Gruppe kann sich nun zum „Gelockten Hund“ aufmachen oder sich zunächst ein wenig umsehen.

Hier treffen sie auf allerhand Charaktere, die ihnen etwas zum Hafen erzählen können.

Gespräch mit Personen am Hafen:

Es fallen immer mehr Arbeiter aus, die mit der Lagerung von Lebensmitteln und anderen verderblichen Waren zu tun haben.
Gasthaus zum "Gelockten Hund"
Vor dem "Gelockten Hund"
\red{\textbf{Szene}}:

Ortsbeschreibung: Eine kleine Absteige, die nicht mehr ganz so gut in Schuss ist. Eine Tatsache, die sich auch in der Kundschaft wiederspiegelt.

Ereignis: Schon vor der Tür wird die Gruppe unangenehm begrüßt. Fünf finstere Gestalten behaupten, es würde Eintritt kosten, in den Laden zu kommen.

Die Spieler müssen schauen, wie ihre Charaktere am besten in die Kneipe kommen. Es gibt unzählige Möglichkeiten.


Interaktionen:

Probe auf Menschenkenntnis:

Dass der Eintritt etwas kostet, ist eine Lüge.
Kampf:

Betrunkener Pirat:
Leben: 80
Waffe: Fäuste(70)
Schaden: 15
Parieren 5
Loot:
Sie finden bei den Männern ein verziertes Kreuz aus Silber. Es ist sogar ein besonders schönes Exemplar.

Zusatz (erfordert Wissen zu christlicher Kultur):
Es ist ein Kreuz, wie es sonst nur Priester tragen würden.
Auf der Rückseite ist Apostel Petrus eingraviert. Kunstfertig!
Im „Gelockten Hund”
\red{\textbf{Szene}}:

Der „Gelockte Hund” ist eine finstere Absteige und sieht dementsprechend heruntergekommen aus. Der Großteil des Publikums ist Gesindel, das Fremden gegenüber nicht sonderlich wohlgesonnen sein dürfte.

Raumbeschreibung: Es herrscht ausgelassene Stimmung. In der Kaschemme selbst ist erst mal aber nichts besonders ungewöhnlich.

Interaktionen:

Die Charaktere können sich umhören, ob jemand Gorich kennt.

Nach Gorich fragen:

Bis auf den Wirt Gert will ihnen niemand Auskunft geben.
Dieser hat allerdings gerade alle Hände voll zu tun.
Es muss gekocht und serviert werden, Musik ist auch keine da. Wenn sie ihm allerdings helfen würden …
Wirt Gert helfen:

Entscheidet sich die Gruppe dem Wirt Gert zu helfen, dann muss sie ihn nun bei Aufgaben in der Kneipe unterstützen.

Die Aufgaben:
Eintopf kochen.
Gäste bedienen.
Einen Streit schlichten.
Musizieren.

Für einen Erfolg müssen mindestens zwei der Aufgaben erfolgreich bestanden werden.

Erfolg:
Gorich gibt sich zu erkennen.

Misserfolg:
Gorich gibt sich nicht zu erkennen und kann nur noch mit Gewalt oder Tricks dazu gebracht werden, sich zu offenbaren. Dies ist aber STARK erschwert.
Gorichs Offenbarung \& Gespräch
\red{\textbf{Szene}}:

Gespräch mit Gorich: Er und seine Leute haben nichts mit der ganzen Sache zu tun. Er zeigt der Gruppe sogar, dass seine eigenen Kinder im Hinterzimmer liegen … krank. Sie hatten schon früh von der Seuche gehört und auch erfahren, dass es irgendwas mit Lebensmitteln zu tun haben könnte. Denn es waren zuallererst die Bauern und Karrenlenker krank geworden, die im Umland lebten. Also kaufte er all seine Nahrung nur noch aus Einfuhr. Das schien aber auch nichts zu helfen. Denn seine Frau ist bereits gestorben, und auch seinen Kindern gehe es immer schlechter. Ein gewisser Hagen habe es ihm verkauft. Dieser lebe in Hammerbrook, direkt vor den Toren der Stadt. Gekauft habe er die Güter direkt bei einem Lagerarbeiter. Er wisse, dass auch andere, die bei ihm gekauft haben, krank wurden.

Gorich bittet die Gruppe seine Kinder nicht zu verraten. Das würde den Untergang seines Geschäfts bedeuten, und dann könne er sich erst recht nicht mehr um sie kümmern.

Interaktionen:

Probe auf Menschenkenntnis:

Gorich scheint die Wahrheit zu sagen.
Jeder, der sich den Kindern von Gorich nähert, erhält Pestilenz +2.

Moral:

Was tut die Gruppe also mit Gorich und seinen Kinder?


Ende des relevanten Hafen-Plots.

Eeksdurf
Der Weg nach Eeksdurf
\red{\textbf{Szene}}:

Die Gruppe macht sich auf den Weg in Richtung des Westens der Stadt. Dort liegt das kleine Dorf Eeksdurf. Bisher wissen sie nicht viel über das Örtchen, aber das soll sich bald ändern.

Nach einer kurzen Strecke werden die Häuser am Straßenrand immer weniger.

Sie nehmen ein bis zwei Abzweigungen immer den Wegweisern nach.

Ereignis: Als sie etwa eine Stunde unterwegs sind, hält der Kutscher abrupt. Sie sehen im Straßengraben einen Karren liegen.

Daneben: Leichen.


Der Überfall
\red{\textbf{Szene}}:

Als sie näher kommen, können sie das ganze Ausmaß des Unheils sehen. Am Wegesrand liegt ein Wagen mit gebrochenem Rad. Daneben entdecken sie drei Leichen, eine davon offensichtlich die eines Priesters, sogar verhältnismäßig gut gekleidet. Leider scheint auch ihn ein unschönes Schicksal ereilt zu haben.

Interaktionen:

Umsehen:

Allen Opfern wurde der Schädel eingeschlagen und Teile ihrer Kleidung gestohlen.
Alle Wertsachen fehlen, und auf dem Karren finden sich Reste von Lebensmitteln. Auch von diesen fehlt aber jede Spur.

Zusatz (erfordert Wissen zu christlicher Kultur): Selbst das Kreuz, welches der Priester mit Sicherheit um den Hals trug, ist verschwunden.

Information für den Spielleiter: Das Kreuz kann die Gruppe an einem betrunkenen Piraten vor dem "Gelockten Hund" im im Hafen finden.

Sie finden keinen Hinweis darauf, wer der verstorbene Priester sein könnte. Allerdings entdecken sie einige Pergamente, die in lateinischer Schrift verfasst sind.

Zusatz (erfordert Lateinkenntnisse): Die Pergamente handeln von schwarzer Magie und anderem Volksglauben.

Ein gutes Dutzend Fußspuren führt weiter in Richtung Eeksdurf, aber auch genügend in Richtung Hamburg. Schwer zu sagen, ob sie von den Tätern oder anderen Reisenden stammen. Aber in jedem Falle sind die Spuren am Ort der Tat nicht älter als eine Stunde. Mehr ist hier nicht zu finden.

Ankunft in Eeksdurf
\red{\textbf{Szene}}:

Eeksdurf liegt vor den westlichen Toren der Stadt, am Rande des Eekshult.

Die Bewohner des Ortes gehen den üblichen Handwerksberufen und der Landwirtschaft nach. Ein beschaulicher Ort. Binnen zweier Wochen starben oder erkrankten bereits über die Hälfte der Bewohner des Dörfchens.

Das kann kein Zufall sein, vermuten einige im Dorf. Jemand sei mit dem Teufel im Bunde, wird gemutmaßt. Eine Hexenjagd entbrennt, vor der niemand sicher zu sein scheint.

Ortsbeschreibung: Als die Gruppe das Dorf betritt, ist alles verlassen. Alle Türen sind verrammelt, und es laufen keine Menschen auf den Straßen herum.

Klopfen sie an eine der Türen, dann antwortet man ihnen, wenn sie passend würfeln.

Interaktionen:

An Türen klopfen:

0 bis 49: Ihnen wird geantwortet.
50 bis 99: Ihr Klopfen bleibt unbeantwortet.

Gespräch mit Dorfbewohnern:

Niemand will sie hereinlassen.
Angeblich weiß niemand, wo all die Männer des Ortes sind.

Fragen sie nach einer rothaarigen Frau, erzählt man ihnen von Ruth, der Tochter der Kräutersammlerin Ottilde.
Diese wohne gleich der Hauptstraße nach am Rand des Eekshult.
Probe auf Menschenkenntnis:

Das die Bewohner nicht wissen, wo genau die Männer des Ortes gerade sind, ist eine Lüge.
Bei Ottildes Haus
\red{\textbf{Szene}}:

Ortsbeschreibung: Am Waldrand steht eine kleine Hütte. Sie ist nahezu verfallen, aber offensichtlich noch bewohnt.

Ereignis: Als die Gruppe sich der Hütte nähert, erkennt sie...

Wurf: Was sieht die Gruppe vor Ottildes Haus?
Einen wütenden Mob (0 bis 49)
Ein großes Feuer (50 bis 99)

Option 1 – Wütender Mob
\red{\textbf{Szene}}:

Ereignis: Vor der Hütte stehen wütend grölende Menschen und schwingen Mistforken, während eine alte, merkwürdig gekleidete Dame vor der Hütte versucht sie zu besänftigen.

Aus dem gegröle lässt sich schließen, dass es sich bei der Dame um die Kräuterfrau Ottilde handelt.

Der Rädelsführer des Mobs ist ein Bauer namens Wolfgang. Dieser fordert vehement die Herausgabe von Ruth. Sie sei an allem Schuld! Schwarze Magie habe sie angewandt und nicht mal die Kirche wolle den Bewohnern mehr helfen!

Interaktionen:

Die Gruppe sollte versuchen den wütenden Mob zu beruhigen.

Gespräch mit Leuten aus dem Mob:

Die Leute im Mob berufen sich immer wieder auf vier Argumente:

*Ruth soll wiederholt dabei gesehen worden sein, wie sie nachts tote Körper in die große Scheune gebracht habe.
*Ruth habe den Brief an die Kirche, welchen sie dem Totensammler mitgeben sollte, nie übergeben.
*Ruth habe die Pest gebracht. Sie war in Hamburg gewesen, um Korn zu kaufen.
*Kaum war sie zurück, brach die Krankheit aus! Zu diesem Zeitpunkt sei noch nirgends sonst in der Gegend jemand erkrankt! (Das ist drei Wochen her)
Probe auf Beruhigen:

Erfolg: Schaffen Sie es den Mob zu beruhigen, dann zieht sich der Mob zurück und sie können mit Ottilde in Ruhe das Haus betreten.
Dort treffen sie auch auf Ruth, die verängstigt in
einer Ecke sitzt.

Misserfolg: Schaffen sie es nicht den Mob zu beruhigen, dann bricht dieser in das Haus ein.
Von Ruth fehlt allerdings jede Spur. Ottilde ist dankbar für ihre Hilfe und verrät der Gruppe,
dass Ruth sich in der großen Scheune verstecke.
Gespräch mit Ruth:

Ruth sagt, sie habe nichts gemacht und wisse nicht, warum die Bauern so aufgebracht seien.
Probe auf Menschenkenntnis:

Ruth lügt. Hakt die Gruppe nach, dann erzählt sie, was vorgefallen ist:
Sie habe direkt am Nikolaifleet Korn holen wollen. Da sei es am günstigsten, habe man ihr gesagt.
Als sie dort ankam, sei es bereits spät gewesen. Sie habe am Lagerhaus, das man ihr beschrieben hatte, angeklopft, allerdings ohne Erfolg.
Also habe sie durchs Fenster geschaut und einen Mann erblickt, der sich über etwas gebeugt habe. Er soll eine Art Vogel-Maske und einen weiten Mantel getragen haben.
Ruth sagt, sie habe Angst gehabt und habe davonlaufen wollen. Sie sei auf dem Schnee ausgerutscht und habe das Gleichgewicht verloren.
Dabei sei mit dem Kopf aufgeschlagen und im Lagerhaus wieder zu sich gekommen. Der Mann habe sich ihr dann als Didrich vorgestellt. Er forsche an der Pestilenz, habe er erklärt.
Nur darum sei er so gekleidet und habe sich an der Leiche zu schaffen gemacht. Ruth habe ihm dann erzählt, dass in Eeksdurf zwar niemand krank sei, aber dass der Totensammler regelmäßig mit den Leichen durch ihr Dorf fahre, um diese tief im Eekhult zu verscharren.
Daraufhin habe Didrich ihr einen Handel vorgeschlagen, und … naja … Ruth habe angenommen. „Wieso auch nicht. Die sind schließlich tot und ich arm. Eins davon kann man noch ändern.“
Die Leichen habe sie also von da an immer in ein kleines Lagerhaus im Armenviertel Hammerbrook gebracht. Dort habe auch ihr Geld gelegen …
Ruth sagt ihnen genau, welches Lagerhaus das gewesen sei. Die Gruppe kann anschließend zum Armenviertel aufbrechen.

Option 2 – Ein großes Feuer
\red{\textbf{Szene}}:

Ereignis: Im Näherkommen sehen sie, dass vor der Hütte ein gewaltiges Feuer brennt.

Niemand ist zu sehen. Im Feuer erkennen sie allerdings die Überreste zweier Körper. Beides Frauen.

Die Tür der Hütte steht offen.

Interaktionen:

In der Hütte umsehen:

Sehen sie sich in der Hütte um, dann finden sie ein Blatt Pergament. Auf das Blatt wurde eine Art Wegbeschreibung gekritzelt – nur sehr grob. Der Weg führt zu einem Gebäude im Armenviertel Hammerbrook.
Genaueres Durchsuchen:

Suchen sie ausgiebiger, dann finden sie Geld. Etwas mehr Geld, als jemand, der hier lebt, besitzen sollte.
Verlassen der Hütte:

Beim Verlassen der Hütte trifft die Gruppe auf Wolfgang und Hermann. Diese werfen ihnen umgehend vor, mit der bösen Zauberin im Bunde zu sein.
Probe auf Beruhigen:

Erfolg: Beruhigen sie die beiden Bauern, erzählen diese der Gruppe, warum sie Ottildes Haus in Brand gesetzt haben.

Sie bringen folgende Argumente:
*Ruth soll wiederholt dabei gesehen worden sein, wie sie nachts tote Körper in die große Scheune gebracht habe.
*Ruth habe den Brief an die Kirche, welchen sie dem Totensammler mitgeben sollte, nie übergeben.
*Ruth habe die Pest gebracht. Sie war in Hamburg gewesen, um Korn zu kaufen.
*Kaum war sie zurück, brach die Krankheit aus! Zu diesem Zeitpunkt sei noch nirgends sonst in der Gegend jemand erkrankt! (Das ist drei Wochen her)

Misserfolg: Schaffen sie es nicht Wolfgang und Hermann zu beruhigen, so kommt es zum Kampf.
Wolfgang ist mit einer Mistforke bewaffnet. Außerdem sehen beide Bauern bereits von der Krankheit gezeichnet aus.

Kampf:

Wolfgang:

Leben: 70
Waffe: Mistgabel (70)
Schaden: 40
Parieren 30

Hermann:

Leben: 70
Waffe: Fäuste(70)
Schaden: 15
Parieren 5

Moral:

Die beiden Bauern brüsten sich mit der Tat. Wie reagiert unsere Gruppe darauf?


Ende des relevanten Eeksdurf-Plots.

Hammerbrook
Jeder, der das Armenviertel betritt, erhält Pestilenz +1.

Einleitung
\red{\textbf{Szene}}:

In Hamburgs Armenvierteln spürt man nichts vom neuen Reichtum. Dicht gedrängt leben hier Hafenarbeiter, einfache Leute und anderes Gesindel in ärmlichen Hütten und Häusern. Die Straßen sind gesäumt von Toten und Kranken, und in kaum einem Hause brennt Licht. Der Tod geht um und zeigt hier seine hässliche Fratze.

Auf den Straßen
\red{\textbf{Szene}}:

Ihr kommt also in Hammerbrook an, einem der Stadtviertel, in denen sich die Ärmsten der Stadt zusammenpferchen, in der Hoffnung, vom Reichtum zu profitieren, den der Handel in Hamburgs Kassen spült.

Ortsbeschreibung: Die Straßen sind verschneit und verlassen. Nur vereinzelt ziehen vermummte Gestalten umher und werfen unserer Gruppe argwöhnische Blicke zu.

Die Gruppe ist hier nicht willkommen …

Ereignis: Eine Gruppe Kinder kommt auf die Gruppe zu.

Probe auf Gassenwissen:
Alle würfeln einmal eine um 10 erschwerte Probe auf Gassenwissen.
Scheitert die Probe, wird dem Charakter etwas gestohlen, und die Kinder laufen damit davon. Es muss ein wichtiger Gegenstand sein, etwas, das die Person nicht missen möchte.
Anschließend können die Charaktere die Kinder verfolgen.
Probe auf Rennen:
Alle verfolgenden Charaktere würfeln eine Probe auf Rennen o.ä.

Erfolg: Holen sie die Kinder ein, geht es beim Lumpensammler weiter.

Misserfolg: Holen sie die Kinder nicht ein, ist der Gegenstand verloren.
Die Gruppe hat daraufhin eine Begegnung mit Sigrun.

Option 1 – Sie holen die Kinder ein und kommen zum Lumpensammler
\red{\textbf{Szene}}:

Gespräch mit dem Lumpensammler: Hier wird die Gruppe künstlich, freudig begrüßt.

Der alte Lumpensammler ist im Viertel gut bekannt und handelt mit allem, was Menschen so zu entbehren haben.

Woher seine Waren kommen, ist ihm dabei recht egal. Ist ja nicht sein Problem. Nun hat er den Gegenstand oder die Gegenstände der Gruppe „erworben“ und will ihn/sie auch nicht ohne Weiteres wieder herausgeben.

Allerdings lässt er sich auf eine kleine Wette ein. Denn so piekfeine Schnösel können doch bestimmt gut ...

Was können so piekfeine Schnösel bestimmt gut?
Ein Rätsel lösen (0 bis 49)
Ein „Problem“ lösen (50 bis 99)
Option 1 – Ein Rätsel
Ereignis: Voller Hochmut stellt der Lumpensammler unserer Gruppe nun ein Rätsel, das es zu lösen gilt.


Das Rätsel:
Einst wurde ein Pirat gefasst und sollte hängen. Es war aber üblich, den zum Tode verurteilten Dieben eine letzte Chance zu geben.

Sie mussten aus einem Säckchen einen Stein ziehen. Im Säckchen befanden sich ein weißer und ein schwarzer Stein. Zog der Dieb den weißen Stein, wurde ihm die Freiheit geschenkt. Zog er hingegen den schwarzen Stein, so baumelte er. Eines Tages kam nun jener Pirat an die Reihe, der einst das Gold des Königs geraubt hatte.

Der König wollte also sichergehen, dass dieser Halunke hängt. Er befahl dem Henker heimlich, zwei schwarze Steine ins Säckchen zu legen.

Am nächsten Tag ging der König zuversichtlich und voller Rachelust zum Galgen. Dort lagen überall schwarze und weiße Steine.

Der Henker nahm zwei von ihnen auf, aber der Verurteilte konnte sehen, dass der Henker zwei schwarze Steine in das Säckchen legte.

Der Dieb hatte den Strick schon um den Hals, als ihm die rettende Idee kam.

Er zog und musste freigelassen werden.

Was war es, dass dem Piraten das Leben rettete?


Lösung 1:

Er nimmt beide Steine aus dem Beutel und zeigt so, dass beide schwarz sind und der Henker ihn betrügen wollte.
Lösung 2:

Er zieht einen der Steine und wirft ihn weg. Da der verbleibende Stein im Sack schwarz ist, muss der gezogene Stein scheinbar weiß gewesen sein.
Erfolg: Beantwortet die Gruppe das Rätsel richtig, bekommen sie ihren Gegenstand zurück.

Misserfolg: Liegen sie falsch, will der Lumpensammler nicht verkaufen. Er genießt den Triumph viel zu sehr! Die entwendeten Gegenstände sind verloren.
Erfolgreiches Abschliessen der Aufgabe: Lösen sie die Aufgabe, berichtet der Lumpensammler ihnen, dass sie nicht die ersten Schnösel seien, die hier waren. Einer mit ganz komischem Aufzug sei dagewesen. Der habe nach abstrusen Pflanzen gefragt. Chillies und Zitronen seien darunter gewesen. Als ob man sowas hier bekomme. Außerdem habe er noch einen Blasebalg gewollt und etwas grobes Leinen, Lumpen und Teer. Komischer Typ.


Option 2 – Ein Problem
Ereignis: Der Lumpensammler führt die Gruppe in ein Hinterzimmer.

Als er die Tür öffnet, drückt sich ihnen der Gestank des Todes entgegen. In dem Raum liegen zwischen allerlei Unrat zwei Menschen, dem Tod bereits nahe.

Aufgabe: Der Lumpensammler will sie loswerden, aber denkt nicht daran, sie anzufassen. Er will sich schließlich nicht anstecken. Die Gruppe soll die beiden „entsorgen“, egal wie.

Jeder der die Sterbenden anfasst: Pestilenz +3.

Ereignis: Allerdings kommt in diesem Moment der Totensammler vorbei und nimmt ihnen die Leichen nur zu gerne ab. Er klagt, dass die Entsorgung nicht so einfach sei.

Interaktionen:

Nachfragen beim Totensammler: Fragen sie genauer nach, erzählt er ihnen von Eeksdurf, wo jemand die Leichen kaufe, aber nur Pestkranke.

Hilfe verweigern: Helfen sie nicht, gibt der Lumpensammler ihnen den Gegenstand bzw. die Gegenstände nicht wieder, egal was sie tun.


Erfolgreiches Abschliessen der Aufgabe: Lösen sie die Aufgabe, berichtet der Lumpensammler ihnen, dass sie nicht die ersten Schnösel seien, die hier waren. Einer mit ganz komischem Aufzug sei dagewesen. Der habe nach abstrusen Pflanzen gefragt. Chillies und Zitronen seien darunter gewesen. Als ob man sowas hier bekomme. Außerdem habe er noch einen Blasebalg gewollt und etwas grobes Leinen, Lumpen und Teer. Komischer Typ.

Option 2 – Das Treffen mit Sigrun
\red{\textbf{Szene}}:

Nachdem die Gruppe es nicht schafft die Kinder einzuholen begegnen sie auf der Straße einer Frau namens Sigrun.

Gespräch mit Sigrun: Sie ist gerade auf dem Weg zur St. Petri-Kirche. Dort werde niemand krank, erzählt sie. Es wären unzählige Menschen dort, die dem Ruf von Pater Salus gefolgt seien. Sie sei dort sicher. Allerdings müsse sie erst in Erfahrung bringen, ob auch ihre Kinder dort willkommen seien. Die lägen zu Hause … krank. Sigrun bittet die Gruppe, bei ihr zu Hause vorbeizuschauen und nach den Kindern zu sehen, während sie weg ist.

Moral:

Wie wird die Gruppe mit der Bitte von Sigrun umgehen?

Entscheidung: Die Gruppe muss sich entscheiden.

Sie helfen den Kindern nicht
Sie helfen den Kindern
Option 1 – Sie helfen den Kindern nicht
Ereignis: Sigrun zieht traurig davon, und die Gruppe bleibt auf sich gestellt.


Option 2 – Sie helfen den Kindern
\red{\textbf{Szene}}:

Die Gruppe kommt am Haus an. Es stinkt nach Tod.

Sie alle erhalten sofort +2 Pestilenz.

Die Kinder weinen und klagen und sind kaum noch bei Verstand.

Kümmern sie sich weiter um die Kinder, erhalten sie nochmals Pestilenz +2.

Ereignis: Plötzlich klopft es an der Tür.

Gespräch mit Hagen: Hagen, der Nachbar, steht vor der Tür und fragt die Gruppe, was sie hier suchten und wo Sigrun sei. Nach einer Erklärung lädt er sie zu sich auf Tee und einen Plausch ein. Er habe ihnen etwas Spannendes zu erzählen!

Bei Hagen vom Fleet
\red{\textbf{Szene}}:

Hagen begrüßt die Gruppe zunächst sehr freundlich.

Raumbeschreibung: Sein Haus ist aufgeräumt. Nirgendwo liegen Waren oder ähnliches herum.

Gespräch mit Hagen:

Anmerkung: Hat Hagen die Gruppe nicht zu sich eingeladen (falls die Gruppe ihn von selbst aufsucht), dann verlangt er im Gegenzug für Informationen einen Aufseherposten innerhalb der Hanse. Die Gruppe kann zustimmen oder ablehnen. Hauptsache sie kommen an die Infos.

Kommen sie von der Nachbarin und werden von ihm eingeladen, dann erzählt er von sich aus.
Hagen berichtet, dass er seit geraumer Zeit heimlich Waren abzweige. Er sei nicht stolz darauf, aber man müsse eben sehen, wo man bleibt. Jedenfalls komme er nachts am Nikolaifleet an diese Waren. Dort arbeite er. Als immer mehr seiner Kollegen krank wurden – das startete bereits vor vier Wochen, also vor allen anderen Ausbrüchen – wurde das sogar noch leichter. Eines Nachts jedenfalls schlich er sich wieder ins Lager, als ihm ein eigenartiger Mann begegnete. Er trug eine lange Maske, die beinahe wie der Schnabel eines Vogels aussah, ein weites Gewand und einen Stock bei sich. Außerdem roch es nach … Parfüm. Der Mann floh, als er Hagen sah, und ließ nichts außer einer Rattenfalle zurück. Diese war aber leer.

Anmerkung: Hagen will sie auf keinen Fall begleiten. Das ist ihm zu gefährlich. Auch mit Gewalt oder Drohungen kann man ihn nicht überzeugen!

Das Lagerhaus von Didrich
\red{\textbf{Szene}}:

Das Lagerhaus, welches Ruth beschrieben hatte, findet die Gruppe recht nahe der Stadtmauer, gerade vor den Toren der wohlhabenden Stadt.

Dennoch sieht es hier erbärmlich aus. Als sie am Lagerhaus ankommen ist dieses verschlossen. Drinnen brennt jedoch Licht.

Interaktionen:

Einbruch:

Es gibt verschiedene Möglichkeiten einzudringen.
Ereignis: Wenn sie einbrechen, tauchen aus der Dunkelheit zwei kräftige Männer auf, die sehr skeptisch sind. Ablenkung ist gefragt!


Im Inneren
\red{\textbf{Szene}}:

Raumbeschreibung: Im Lagerhaus ist es eiskalt. Allerdings brennt eine riesige Öllampe, die auf einem Tisch steht, auf dem allerlei Dinge liegen.

Darunter liegt eine tote Ratte, eine Zitronenschale und etwas von einem roten, scharfschmeckenden Pulver (Chili).

Interaktionen:

Türen öffnen:

Es gibt drei Türen zu kleineren Räumen. Bei allen sind jegliche Ritze und Schlitze säuberlich mit Lumpen und Lappen verstopft und verteert. Eine Ritze ist offen.
Öffnen sie diese Tür, entdecken sie schlimm zugerichtete Leichen von Pestkranken.

Alle im Haus bekommen Pestilenz +2.

Eine Tür ist offen. Dahinter finden sie üppige Kornvorräte. Auf den Säcken steht die Adresse eines Lagerhauses am Nikolaifleet geschrieben.
Öffnen sie die dritte Tür, strömen Ratten heraus. Sie sind ausgehungert und aggressiv. Sie greifen die Gruppe an!

Kampf:

Rattenschwarm:
Leben: 10 Ratten mit jeweils 10 Lebenspunkten
Waffe: Biss (40)
Schaden: Jeder Angriff wird mit einem W10 geworfen. Das Ergebnis sind die Ratten, die erfolgreich angreifen. Jeder Biss macht dabei 4 Schaden.
Ausweichen 10

Sind die Ratten besiegt, finden sie auch hier Säcke mit der Adresse eines Lagerhauses am Nikolaifleet.

Ende des relevanten Hammerbrook-Plots.

Die Kirche St. Petri
Einleitung
\red{\textbf{Szene}}:

Die Kirche St. Petri steht ganz im Zeichen des neuen Hamburger Wohlstands. Noch immer wird gebaut, aber man sieht ihr schon jetzt an, dass sie eines der zukünftigen Wahrzeichen der Handelsmetropole sein wird.

Viele Menschen pilgern angesichts ihrer Machtlosigkeit gegenüber der Pest hierher, um Schutz und Trost zu suchen. Kranke und Verzweifelte säumen die Gänge, und der Geruch von Tod liegt in der Luft.


Ankunft an der Kirche
\red{\textbf{Szene}}:

Ortsbeschreibung:Sie stehen vor der gewaltigen Pforte einer der schönsten Kirchen Hamburgs. In neuem Glanz erstrahlt St. Petri und ist aufgrund der Lage auf einem Geestrücken bereits von Weitem sichtbar.

Ereignis: Sobald jemand an die Tür klopft werden sie von Helfern begrüßt und hereingelassen.

Interaktionen:

Probe auf Wahrnehmung, Aufmerksamkeit, Kirchenwissen oder ähnliches:

Sie tragen zwar Mönchsgewänder, scheinen aber keinem Orden zugehörig zu sein.
Zumindest nicht erkennbar.
Nach Pater Salus fragen:

Wird einer der Helfer nach Pater Salus gefragt, so wird ihnen berichtet, dass dieser aufgebrochen sei, um Gerüchten auf den Grund zu gehen.
Nun ist von schwarzer Magie die Rede, böse Mächte seien in der Stadt am Werk. Die Gruppe ist aber herzlich eingeladen zu bleiben, sofern sie gesund sind und dies beschwören.
\red{\textbf{Szene}}:

Ereignis: Dann lässt man sie in den Hauptraum der Kirche. Hier sitzen überall Menschen mit besorgten Gesichtern herum. Andere wiederum blicken voller Zuversicht und lauschen gebannt Gesprächen, Gesängen und Predigten, welche Mönche überall im Raum zum Besten geben. Außerdem gehen verdächtig viele Klingelbeutel herum. Es führen einige Türen aus dem Hauptraum. Alle sind verschlossen.

Raumbeschreibung: Das innere einer großen Kirche. Man sieht, dass die Bauarbeiten noch im Gange sind, doch im Hauptraum kann man sich bereits ohne größere Probleme aufhalten. Trotz Bau lässt sich bereits eine prunkvolle Kirche erahnen.

Interaktionen:

Probe auf Wahrnehmung,Aufmerksamkeit, Lauschen oder ähnliches:

Hört die Gruppe bei den Mönchen genauer hin, ist die Rede vom Jüngsten Gericht,
Sünde und der Strafe des Herren, die alle Sünder trifft.
Man könne sich aber von seinen Sünden befreien.
Nur etwas weltliches Gut müsse man opfern, um hier,
bei den Frommen, bleiben zu dürfen.
Gespräch mit Anwesenden:

Fragen sie die Leute, so berichten manche davon, dass sie mit Angehörigen hier waren.
Die wurden dann aber irgendwann nach hinten gebeten, wohl weil sie besonders fromm waren. Während sie so lauschen, bemerken sie, dass immer wieder einzelne Menschen aus dem Saal durch die Seitentüren gebracht werden.
Man will sie hier aber nicht durchlassen. Das seien „private Gemächer“ der Priester und Mönche. Die Gruppe kann versuchen, sich Zugang zu verschaffen.
Im Hinterzimmer
\red{\textbf{Szene}}:

Schaffen sie es irgendwie hier hinein, geht es erstmal durch ein paar Gänge.

Es stinkt nach Tod. Alle erhalten Pestilenz +1.

Nach einer Weile kommen sie in eine Art Gewölbe. Hier liegen unzählige Kranke, von gerade so infiziert bis zu bereits verstorben, hinter einem Gitter eingesperrt wie in einer Zelle.

Gehen sie näher an die Menschen heran, erhalten sie Pestilenz +3.

Sie alle sind überzeugt davon, dass sie das hier als Sünder verdient haben.

Am hinteren Ende des Raumes gibt es eine Tür. Sie scheint ins Freie zu führen. Davor am Gitter steht eine junge Frau, Gundel. Sie ist bereits von der Pest gezeichnet, winkt die Gruppe aber zu sich.

Gespräch mit Gundel: Sie bittet die Gruppe, ihr dabei zu helfen zu entkommen. Sie müsse zu diesem Arzt, diesem Didrich. Der wisse, was zu tun ist. Da ist sie sich sicher! Gundel arbeitet am Nikolaifleet. Dort kocht sie für die Männer. Nach und nach wurden sie alle krank.

Plötzlich tauchte der Arzt auf, Didrich von Sinnfeld. Er versprach ihr, wenn sie ihm ein paar Ratten aus der Küche fange, würde er sie fürstlich bezahlen. Sie lehnte ab. Das schien ihr doch sehr merkwürdig, und die Bezahlung sei nicht üppig gewesen. Als er aber sagte, dass er an etwas arbeite, um den Schwarzen Tod zu besiegen, lenkte sie ein. Sie gab ihm die Ratten. Doch dann wurden immer mehr Leute krank. Sie bekam es mit der Angst zu tun und kam hierher. Doch es war zu spät. Man erleichterte sie um ihr Geld und sperrte sie dann hier zum Sterben hinein.


Moral:

Lassen sie Gundel und die anderen raus und gefährden damit die Stadt? Oder gehen sie und überlassen die Kranken ihrem unumgänglichen Schicksal?

Vorm Ausgang
\red{\textbf{Szene}}:

Ereignis: Hier treffen sie noch einmal den Totensammler. Er schaut sie vorwurfsvoll an und zieht dann weiter klingelnd seine Runden.


Ende des relevanten St.Petri-Plots.

Der Nikolaifleet
Einleitung
Erst seit wenigen Jahren wächst am Nikolaifleet ein gewaltiges Lagerzentrum für die Waren aus aller Welt.

Der neue Reichtum trägt hier Früchte, und in direkter Nähe zu Tee, Gewürzen und Tulpen siedeln sich die betuchteren Bürger Hamburgs in prächtigen Villen an. Alles scheint makellos und vom Chaos der übrigen Stadt unberührt. Fast schon zu makellos.

Ankunft am Fleet
\red{\textbf{Szene}}:

Ortsbeschreibung: Der Nikolaifleet ist ordentlich und sauber. Weißer Schnee fällt auf frisch gepflasterte Straßen. Niemand ist zu sehen, obwohl hier reges Treiben herrschen sollte. Die Gruppe findet recht unkompliziert das Warenlager.


Lagerhaus Grote Buur
\red{\textbf{Szene}}:

„Grote Buur“ steht in eisernen Lettern überm Eingang. Zur Überraschung aller ist die Tür nicht verschlossen.

Raumbeschreibung: Ein typisches Lagerhaus. Auf der Rückseite sind die Tore gut sichtbar, die zum Löschen der Boote genutzt werden, welche die Waren an den Fleet bringen. Das Lagerhaus hat gleich mehrere Etagen, und es gibt diverse Kräne und Seilzüge, um Waren auf den Ebenen zu transportieren. Bei näherer Betrachtung erkennt man, dass sich hier einige Ratten niedergelassen haben.

Ereignis: Niemand scheint hier zu sein. Doch dann hören sie eine Art Jammern aus den Untergeschossen. Sie finden eine Luke. Darunter sitzt ein gefesselter Mann. Er hat überall am Körper Wunden, die nicht nur von der Pest verursacht wurden.

Interaktionen:

Begegnung mit dem Mann:

Dem Mann fällt das Sprechen sehr schwer, aber er kann noch „Sinnfeld“ und „gegenüber“ stammeln.

Die Gruppe kann ihn retten, erhält dann aber Pestilenz +3.

Andernfalls stirbt er vor ihren Augen.
Körper untersuchen: Die Wunden an seinem Körper sind unter anderem Rattenbisse. Außerdem ist er übersäht mit Flöhen.

Das Handelsregister
\red{\textbf{Szene}}:

Das Handelsregister ist der Ort, an dem alles Wissen der Stadt um jeden Geschäftsmann und seine Machenschaften zusammenkommt.

Ortsbeschreibung: Das Gebäude ist modern, fein verziert und kann sich sehen lassen.

Die Gruppe kann hierhin gehen, um Auskunft über den Eigentümer des „Grote Buur“ und Didrich von Sinnfeld einzuholen.

Beim Eintreten sehen sie eine Beamtin, die offensichtlich das Sagen hat, und ein paar um sie herum wuselnde Mitarbeiter.

Interaktionen:

Auf Anfrage kann die Gruppe hier ein Dokument erhalten, das belegt, dass Didrich von Sinnfeld der Eigentümer des „Grote Buur“ ist und gegenüber wohnt.

Das Haus gegenüber
\red{\textbf{Szene}}:

Gleich gegenüber des „Grote Buur“ steht ein ansehnliches Bürgerhaus.

Ortsbeschreibung: Es ist gepflegt, sauber und von außen in hervorragendem Zustand.

Ereignis: Die Gruppe kann an die Tür und anklopfen. Als sie klopfen öffnet ihnen...

Wurf: Wer öffnet?
Ein Mann (0 bis 49)
Eine Frau (50 bis 99)
Option 1 – Ein Mann
\red{\textbf{Szene}}:

Ereignis: Die Tür öffnet sich einen Spalt, und eine Stimme ertönt.

Man sieht aber niemanden. „Wer da?“

Die Person reagiert erschrocken auf die Stimmen unserer Charaktere und stürmt nach hinten ins Haus davon. Verfolgen sie den Mann, sehen sie diesen um eine Ecke rennen und hören dann ein gewaltiges Klirren und Krachen.

Sie kommen in einen düsteren Raum voller medizinischer Instrumente und anderer abstruser Gerätschaften. Überall im Raum sind Notizen und Tagebücher verteilt. Auf dem Tisch in der Mitte des Raumes liegt ein kleiner Junge. Er ist tot und gezeichnet von der Pest. Sein Brustkorb und sein Kopf sind geöffnet. Seine Organe liegen auf Tellern neben seinem Körper. Sein Gesicht ist in Panik und Schmerz erstarrt. Vor dem Tisch auf dem Boden liegt der Körper eines bewusstlosen Mannes, der eben noch vor ihnen geflohen war. Er hat das Blut des Jungen an den Händen.

Interaktionen:

Dursuchen oder Lesen der Notizen und Tagebücher:

Die Notizen im Raum verraten, dass der Mann an etwas rund um die Pest geforscht hat und das Ganze etwas mit Ratten zu tun gehabt haben dürfte. Jedenfalls tauchen diese überall auf. Auch Beschreibungen zum Bau von Geräten (Masken und Kleidung), die eine Ansteckung verhindern, gibt es.

Probe auf Medizin:

Außerdem finden sie unzählige Zeichnungen sezierter Körper sowie Berichte von Obduktionen lebender Patienten und Krankheitsverläufen bei Gefangenen. Harter Toback. Aber auch wichtige Erkenntnisse. Diesen Unterlagen zufolge hat die Krankheit ihren Ursprung in der Rattenpopulation, und hält man sich von Kranken und Ratten fern, so kann man eine Ansteckung verhindern.
Moral:

Was macht die Gruppe mit Didrich und den Informationen?

Option 2 – Eine Frau
\red{\textbf{Szene}}:

Ereignis: Klopfen sie an, öffnet die Magd Traudel.

Gespräch mit Traudel: Im Haus wohne ein Herr von Sinnfeld, Didrich von Sinnfeld. Ja, er sei zu Hause. Die Gruppe möge eintreten!

Die Magd führt sie in den Salon. Sie werden gebeten, zu warten.

Ereignis: Nach einer Weile hören sie einen Schrei. Dann rennt das Hausmädchen an ihnen vorbei zur Tür hinaus.

Geht die Gruppe dem nach, finden sie Didrich in seinem Arbeitszimmer. Dieser beugt sich gerade über den leblosen Körper eines kleinen Jungen und ist im Begriff, ihn zu sezieren.

Gespräch mit Didrich:

„Ein Kleingeist, die gute Traudel. Viel zu leicht zu erschrecken. Und nun?! Habe ich nichts als Ärger am Hals.

Denn die werten Herrschaften, so nehme ich an, stehen meinem Treiben hier ebenso wenig wohlgesonnen gegenüber wie andere Vertreter ihrer Stände.

Sehe ich das richtig?”

Didrich schildert ihnen, dass er an der Krankheit forsche. Er sei kurz vor einem Durchbruch. Es wisse nun mit Sicherheit, dass es etwas mit Ratten zu tun habe. Nur der genaue Ablauf der Ansteckung sei ihm noch schleierhaft. Aber ein Fehlen von Ratten in einer Stadt ginge unmittelbar mit einem Abhandensein der Seuche einher, obwohl durch Kontakt zu Kranken ebenfalls eine Seuche zustande kommen könne. Seine Forschungen seien nicht immer ganz lupenrein gewesen. Er brauchte tote Körper, später lebendige, das gebe er zu. Zum Glück gäbe es genug Kranke direkt hier am Fleet. Gleich gegenüber fielen die Arbeiter reihenweise um. Aber dann blieben sie zu Haus. Ein Besuch auf Hammerbrook sei also unumgänglich geworden. Dann noch einer. Und noch einer. Jedenfalls könne man solche Forschungen in einer kleingeistigen Stadt wie Hamburg nicht ohne Weiteres betreiben. Daher die Heimlichtuerei. Was nun? Soll er seine Forschungen fortsetzen?

Ihm ist durchaus bewusst, weshalb die Gruppe hier ist.


Moral:

Was tun sie mit Didrich und seinen Forschungen?

Ende des relevanten Nikolaifleet-Plots.

Der Abschlussbericht
\red{\textbf{Szene}}:

Gespräch mit dem Rat: Am letzten Tage, oder wenn sie sich eher im Stande sieht, muss die Gruppe vor den Hamburger Rat treten. Sie werden aufgefordert, ihre Ergebnisse zu präsentieren. Dabei stellt der Rat durchaus Fragen und ist kritisch. Außerdem wurden dem Rat Informationen über das Handeln der Gruppe zugetragen, gerade was die moralischen Fragen betrifft:

Wie ist die Gruppe mit Gorichs Familie umgegangen?
Haben sie die Kranken in St. Petri befreit?
Wie sind sie mit den Bauern Wolfgang und Hermann umgegangen?
Was machten sie mit Didrich?
Haben sie Sigruns Kindern geholfen?
Hat die Gruppe alle Fragen beantwortet, will sich der Rat beraten.

Was hält der Spielleiter vom Ergebnis der Recherche, und was soll nun geschehen?

1. Die Gruppe war sorgfältig. Setzen wir ihren Vorschlag um!
2. Die Gruppe war nicht überzeugend genug. Warten wir ab!
3. Die Gruppe war nachlässig. Sie sollen bestraft werden!
Die Entscheidung des Spielleiters beendet das Abenteuer.


\section{\textbf{* Im Sitz des Beirats}}
%!TEX root = ../main.tex

\red{\textbf{Szene}}:

Nachdem ihr aus dem Rat entlassen wurdet macht ihr euch also auf den Weg zum Sitz des Beirats. In der unmittelbaren Umgebung des Ratshauses spürt man den Aufstieg Hamburgs als Handelsmetropole am deutlichsten. Ihr schlendert durch breite Straßen die von hohen Häusern gesäumt werden. Dienstboten eilen über den Pflasterstein und auch sonst herrscht geschäftiges treiben, als ihr unweit der St. Michaelis Kirche vor eine Kapelle tretet, die euch der Rat als eure Operationszentrale genannt hat

\textbf{Raumbeschreibung}: Als ihr an der kleinen Kapelle ankommt, erkennt ihr, dass es sich dabei um ein durchaus anschauliches Gebäude handelt, dass erst kürzlich einen neuen Anstrich mit weißer Farbe erhalten hat. Im Inneren stehen allerhand Tische und Stühle herum. Außerdem gibt es Schlafmöglichkeiten und so ziemlich alles, was man zum Leben so gebrauchen kann. Selbst ausgewählte Speisen stehen bereits zum Verzehr bereit.

\textbf{Ereignis}: Die Gruppe kann sich erst mal unterhalten. Ihr Gespräch wird allerdings von einem Klopfen unterbrochen ...

\begin{tcolorbox}
  Wurf: Wer oder was unterbricht das Gespräch unserer Gruppe?
  Das Militär (1 bis 33) (gehe zu \ref{militär}) \\
  Der Tod (34 bis 66) (gehe zu \ref{tot}) \\
  Ein Kind (67 bis 99) \ref{kind} \\
  Bei 100: würfle erneut.
\end{tcolorbox}

\section{Der militärische Besuch}
\label{militär}

\red{\textbf{Szene}}:

General zur Brügge steht davor und bittet um Einlass.

Gespräch mit Brügge: Dieser berichtet ihnen in vehementem Ton, dass die ganze Krise ein Werk der Dithmarscher sei. Diese hätten sich jahrelang an Hamburgs Handelsschiffen gütlich getan. Nun, da es einen Vertrag gibt, der das verhindert, versuchen einige von ihnen die Stadt zu schwächen, um davon zu profitieren oder sie gar ganz an sich zu reißen. Die Dithmarscher operieren von ihrem Versteck aus, das sich in einer Hafenkaschemme namens „Beim Gelockten Hund“ befinden soll. Ein gewisser Gorich leite das Ganze. Dort sollten sie mit ihren Recherchen beginnen.

Interaktionen:

Probe auf Menschenkenntnis:

General zur Brügge sagt schon die Wahrheit, aber seine Perspektive könnte durchaus verzerrt beziehungsweise einseitig sein.
Die Charaktere können sich aber sicher sein, dass er nicht lügt, zumindest seiner Auffassung nach nicht.

Ab hier können die Spieler frei entscheiden, wohin sie gehen wollen! Nach Ablauf der Frist von vier Tagen müssen sie beim Hamburger Rat vorsprechen. Bis dahin müssen sie sich auf eine Handlungsempfehlung festgelegt haben!


Option 2 – Der Tod
\red{\textbf{Szene}}:

Ereignis: Während sich die Gruppe noch unterhält, hören die Charaktere plötzlich ein lautes Klopfen an der Tür.

Davor steht der Totensammler Hanno. Er fragt, ob es Tote gäbe, die abzuholen seien, und ob im Haus bereits die Pest wüte.

Gespräch mit Hanno: Im Armenviertel sei es am Schlimmsten. Die Leichen könne er kaum mehr entsorgen. Man müsse kreativ werden.

Gegen Bestechung verrät er, dass er jemandem Leichen verkaufe. Dazu müsse er sie allerdings recht weit fortbringen, nämlich in einen kleinen Ort namens Eeksdurf vor den Toren der Stadt. Dort hinterlege er die leblosen Körper in einem Lagerhaus, wo bereits seine Bezahlung auf ihn warte. Die Absprache habe er dereinst mit einer jungen rothaarigen Frau getroffen.

Sie habe ihn angesprochen, nachdem sie ihn beim Abholen von Leichen im Dorf sah...


Ab hier können die Spieler frei entscheiden, wohin sie gehen wollen! Nach Ablauf der Frist von vier Tagen müssen sie beim Hamburger Rat vorsprechen. Bis dahin müssen sie sich auf eine Handlungsempfehlung festgelegt haben!


Option 3 – Ein Kind
\red{\textbf{Szene}}:

Ereignis: Es klopft plötzlich an der Tür und davor steht ein Kind zusammen mit seiner stark vermummten Mutter.

Gespräch mit der Mutter und dem Kind: Sie kämen aus dem Armenviertel Hammerbrook und seien auf der Suche nach der St. Petri-Kirche. Sie soll ein Zufluchtsort für gesunde und sündenfreie Menschen sein. Niemand werde dort krank! Für sie sei es zu spät, hustet die Frau, aber ihr kleines Kind, das sei noch zu retten. Sie wissen das alles von einem Mann, der im Armenviertel nach den Leuten sehe. Er werde nicht krank, egal was er tue ... Er habe sie losgeschickt. Sie wüssten gern den Weg.

Die Gruppe kann eine Beschreibung von Didrich von Sinnfeld erhalten. Außerdem geben ihnen die beiden auf Nachfrage den Tipp, einmal beim Lumpensammler im Armenviertel vorbeizuschauen.


Ab hier können die Spieler frei entscheiden, wohin sie gehen wollen! Nach Ablauf der Frist von vier Tagen müssen sie beim Hamburger Rat vorsprechen. Bis dahin müssen sie sich auf eine Handlungsempfehlung festgelegt haben!


\section{\textbf{Am Hafen}}
\label{Hafen}
%!TEX root = ../main.tex

\red{\textbf{Szene}}:

Nach einem kurzen Fußmarsch steht ihr nun also inmitten des Hafens. Es herrscht geschäftiges Treiben. Allerlei Matrosen (\blue{\ref{Matrosen}}) und Hafenarbeiter (\blue{\ref{Hafenarbeiter}}) gehen ihren Aufgaben nach. An den Kais stehen einige Aufseher (\blue{\ref{Aufseher}}) und löschen die Ladungen der vor Anker liegenden Handelsschiffen.

\textbf{Ortsbeschreibung}: Allerhand Gesindel und unzählige Hafenarbeiter treiben sich herum. Prostituierte (\blue{\ref{Prostituierte}}) bieten ihre Dienste an, und Kinder betteln um ein wenig Brot. Hin und wieder sieht man Menschen, die sich vermummen oder gar ihr ganzes Gesicht hinter Tüchern verbergen.

\red{\textbf{Interaktionen}}:

Die Gruppe kann sich nun zum „Gelockten Hund“ aufmachen oder sich zunächst ein wenig umsehen. An den Docks treffen sie auf allerhand Charaktere, die ihnen etwas zum Hafen erzählen können.
Diese werden ihnen erzählen, dass immer mehr Arbeiter ausfallen, die mit der Lagerung von Lebensmitteln und anderen verderblichen Waren zu tun haben.

\subsection{Vor dem \gqm{Gelockte Hund}}
\label{vorhund}

\red{\textbf{Szene}}:

Ihr haltet vor einem kleinen Wirtshaus dessen besten Tage bereits weit in der Vergangenheit liegen. Es sieht - wie seine Kundschaft - ein wenig heruntergekommen aus. Vor der Türe des Gebäudes halten sich drei finstere Gestalten, bei denen es sich wohl um Seeräuber handelt, auf.

Ereignis: Schon vor der Tür wird die Gruppe unangenehm begrüßt. Die Seeräuber behaupten es würde Eintritt kosten, in den Laden zu kommen.

Es gibt nun verschiedene Möglichkeiten an den Seeräubern vorbei in das Gasthaus zu gelangen:

\begin{itemize}
  \item \red{\textbf{Probe auf Menschenkenntnis (erleichtert):} Dass der Eintritt etwas kostet, ist eine Lüge.} \\
  \item \textbf{\red{Kampf:}} Betrunkener Pirat \\
\begin{center}
  \begin{tabular}{cc}
  \toprule
  Fähigkeit & Punkte \\
  \midrule
  Leben & 80 \\
  Fäuste & 70 \\
  Schaden & 15 \\
  Parieren & 5 \\
  \bottomrule
\end{tabular}

\end{center}

Werden die Piraten besiegt können die Spieler ein verziertes Kreuz aus Silber in der Tasche eines Piraten finden.

\red{\textbf{Probe auf christliche Kultur o.ä.}: Es ist ein Kreuz, wie es sonst nur Priester tragen würden.
Auf der Rückseite ist Apostel Petrus eingraviert.}
\end{itemize}

\subsection{Im \gqm{Gelockten Hund}}
\label{imhund}

\red{\textbf{Szene}}:

Der Eindruck, dass es sich hier um eine finstere Absteige handelt bestätigt sich, als ihr eintretet. Auch das Innere dementsprechend heruntergekommen aus. Der Großteil des Publikums ist Gesindel, das Fremden gegenüber nicht sonderlich wohlgesonnen sein dürfte.

\textbf{Raumbeschreibung}: Es herrscht ausgelassene Stimmung. In der Kaschemme selbst ist erst mal aber nichts besonders ungewöhnlich.

\red{\textbf{Interaktionen}}:

Die Charaktere können sich umhören, ob jemand Gorich kennt. Sollten sie nach ihm fragen wird niemand bis auf den Wirt Gert (\blue{\ref{Gert}}) ihnen antworten.

Gert: \gqm{\textit{Edle Herren, ich würde euch gerne Auskunft geben, doch seht was hier los ist! Heute morgen ist meine Wirtin nicht zur Arbeit erschienen. Auch meine Frau, die das Essen zubereitete ist letzte Woche der Pest zum Opfer gefallen. Wenn ihr mir... ein wenig unter die Arme greifen könntet? Danach werde ich euch freilich gerne helfen!}}

Entscheidet sich die Gruppe dem Wirt Gert zu helfen, dann muss sie ihn nun bei Aufgaben in der Kneipe unterstützen. Diese Aufgaben sind:

\begin{itemize}
  \item Eintopf kochen (Probe auf Kochen o.Ä)
  \item Gäste bedienen (Probe auf Menschenkenntnis (erleichtert))
  \item Einen Streit schlichten (Probe auf Beruhigen o.Ä)
  \item Ein wenig Musizieren (Probe auf passendes Talent)
\end{itemize}

Für einen Erfolg müssen mindestens zwei der Aufgaben erfolgreich bestanden werden. Sind mehr Aufgaben erledigt kann der Wirt je nach Ermessen des Spielleiters den Abenteurern einen Schilling für ihre Dienste geben.

Schafft die Gruppe es nicht zwei der vier Aufgaben zu bewältigen gibt Gorich sich nicht zu erkennen. Er kann nur noch mit Gewalt oder Tricks dazu gebracht werden, sich zu offenbaren. Dies ist aber stark erschwert.

\red{\textbf{Szene}}:

Ein großer, bärbeißiger Mann mit schmalem Gesicht und noch schmaleren Augen tritt an euch heran. Er stellt sich als Gorich (\blue{\ref{Gorich}}) vor. Er erzählt den Abenteurern er und seine Leute haben nichts mit der ganzen Sache zu tun. Er zeigt der Gruppe sogar, dass seine eigenen Kinder im Hinterzimmer liegen... krank. Sie hatten schon früh von der Seuche gehört und auch erfahren, dass es irgendwas mit Lebensmitteln zu tun haben könnte. Denn es waren zuallererst die Bauern und Karrenlenker krank geworden, die im Umland lebten. Also kaufte er all seine Nahrung nur noch aus Einfuhr. Das schien aber auch nichts zu helfen. Denn seine Frau ist bereits gestorben, und auch seinen Kindern gehe es immer schlechter. Ein gewisser Hagen habe es ihm verkauft. Dieser lebe in Hammerbrook, direkt vor den Toren der Stadt. Gekauft habe er die Güter direkt bei einem Lagerarbeiter. Er wisse, dass auch andere, die bei ihm gekauft haben, krank wurden.

Bevor die Abenteurer aufbrechen wird Gorich sie bitten seine Kinder und ihn nicht zu verraten. Das würde den Untergang seines Geschäfts bedeuten, und dann könne er sich erst recht nicht mehr um sie kümmern.

\red{\textbf{Interaktionen}}:

\red{\textbf{Probe auf Menschenkenntnis}: Gorich scheint die Wahrheit zu sagen.}

\purple{\textbf{Pestilenz}: Jeder, der sich den Kindern von Gorich nähert, erhält Pestilenz +2.}

\green{\textbf{Moral}: Was tut die Gruppe also mit Gorich und seinen Kinder?}

Ende des relevanten Hafen-Plots. Weiter mit:

Eeksdurf (gehe zu \blue{\ref{Eeksdurf}}) \\
Hammerbrook (gehe zu \blue{\ref{Hammerbrook}}) \\
Die Kirche St. Petri (gehe zu \blue{\ref{Petri}}) \\
Der Nikolaifleet (gehe zu \blue{\ref{Fleet}}) \\

\newpage

\section{\textbf{In Eeksdurf}}
\label{xd}
%!TEX root = ../main.tex

\subsection{Der Weg nach Eeksdurf}
\label{nachxd}

Ihr macht euch also auf den Weg nach Eeksdurf. Schon bald drängen sich die Häuser weniger dicht und ihr verlasst Hamburg in Richtung Westen. Größtenteils verläuft der Weg aus gestampfter Erde gerade durch die Felder, an ein paar Stellen teilt sich der Weg und ihr folgt den Wegweisern nach Eeksdurf. So seid ihr also etwa eine halbe Stunde unterwegs als ihr plötzlich am Wegesrand etwas bemerkt.

\red{\textbf{Szene}}:

Im Straßengraben liegt umgekippt ein Karren. Wollen die Spieler die Szene genauer untersuchen entdecken sie, dass die Speichen der Räder gebrochen sind. Außerdem liegen im Straßengraben drei Leichen. Einer der Toten trägt ein Priestergewand und liegt mit durchgeschnittener Kehle dahingestreckt.

\red{\textbf{Probe auf Medizin o.ä.(erleichtert)}: Die anderen beiden Toten wurden durch Stichwunden in den Oberkörper getötet}

Es scheint sich um einen Raubüberfall zu handeln. Wertsachen finden sich hier keine, auch Teile ihrer Kleidung wurden den Leichen vom Leib gerissen.

\red{\textbf{Probe auf christliche Kultur o.ä. (erleichtert)}: Selbst das Kreuz, welches der Priester mit Sicherheit um den Hals trug, ist verschwunden.}

\red{\textbf{Information für den Spielleiter}: Das Kreuz kann die Gruppe an einem betrunkenen Piraten vor dem "Gelockten Hund" im Hafen finden.}

Ihr findet sonst keinen Hinweis darauf, wer die Verstorbenen sein könnte. Allerdings entdeckt ihr einige Pergamente, die in lateinischer Schrift verfasst sind.

\red{\textbf{Probe auf Latein o.ö}: Die Pergamente handeln von schwarzer Magie und anderem Volksglauben.}

Außerdem finden die Spieler bei genauerem Hinsehen in der Brusttasche des Priesters ein Brief von einem Wolfgang (\blue{\ref{Wolfgang}}) aus Eeksdurf, in dem dieser die Kirche um sofortigen Beistand in einer dringenden Sache bittet.

Ein gutes Dutzend Fußspuren führt weiter in Richtung Eeksdurf, aber auch genügend in Richtung Hamburg. Schwer zu sagen, ob sie von den Tätern oder anderen Reisenden stammen. Aber in jedem Falle sind die Spuren am Ort der Tat nicht älter als eine Stunde. Mehr ist hier nicht zu finden.

\subsection{Ankunft in Eeksdurf}
\label{inxd}

\red{\textbf{Szene}}:

So macht ihr euch also weiter auf den Weg nach Eeksdurf. Nach einer weiteren halben Stunde seht ihr wie zwischen Bäumen die ersten Häuser des kleinen Dorfes hervortreten. Ihr folgt dem Weg und steht bald in dem kleinen Örtchen auf einer menschenleeren Straße. Durch den alltäglichen Klatsch und Tratsch wisst ihr, dass die meisten Einwohner hier einfache Handwerker und Bauern sind. Vom sonst so geschäftigen Treiben ist nun nichts mehr zu spüren, es ist alles verlassen. Türen und Fenster sind verrammelt.

Klopfen die Spieler an eine der Türen, dann antwortet man ihnen, wenn sie passend würfeln.

\begin{tcolorbox}
  Wurf: Wird ihnen auf das Klopfen geantwortet? \\
  0 bis 49: Ihnen wird geantwortet. \\
  50 bis 99: Ihr Klopfen bleibt unbeantwortet.\\
\end{tcolorbox}

Sollte einer der Dorfbewohner mit ihnen sprechen macht er einen verängstigten Eindruck. Wo die anderen alle seien wisse er nicht.

\red{\textbf{Probe auf Menschenkenntnis}: Das er nicht wisse wo alle seien ist eine Lüge.}

Fragen die Spieler nach einer rothaarigen Frau erzählt man ihnen von Ruth (\blue{\ref{Ruth}}), der Tochter der Kräutersammlerin Ottilde (\blue{\ref{Ottilde}}). Diese wohne gleich die Straße runter am Dorfesrand in einer kleinen verfallenen Hütte. Wenn man der Straße folge könne man es nicht verfehlen.

\red{\textbf{Probe auf Menschenkenntnis}: Auch die Wegweisung ist eine Lüge.}

Folgen die Spieler diesem Weg gelangen sie bald an den Dorfesrand, dort ist nirgends ein Haus zu sehen, das auf die Beschreibung des Dorfbewohners passt. Jedoch hören die Abenteurer ein Geräusch...

\begin{tcolorbox}
  Wurf: Was hört die Gruppe? \\
  1 bis 50: Sie hören aufgebrachte Stimmen. (gehe zu \blue{\ref{mob}})\\
  51 bis 100: Sie hören ein Knacken und Rauschen. (gehe zu \blue{\ref{feuer}})\\
\end{tcolorbox}

\subsubsection{Option 1 - Der wütende Mob}
\label{mob}

\red{\textbf{Szene}}:

Ihr folgt den Geräuschen und gelangt an eine kleine Hütte vor dem Waldrand. Die Hütte macht einen verfallenen Eindruck, scheint aber noch bewohnt zu sein. Vor der Haustüre stehen wütend grölende Menschen und schwingen Mistforken, während eine alte, merkwürdig gekleidete Dame vor der Hütte versucht sie zu besänftigen.
Aus den aufgebrachten Wortgefechten lässt sich schließen, dass es sich bei der Dame um die Kräuterfrau Ottilde handelt.

Der Rädelsführer des Mobs ist ein Bauer namens Wolfgang (\blue{\ref{Wolfgang}}). Dieser fordert vehement, dass Ruth aus dem Haus herauskommt. Sie sei an dem Unglück Schuld! Schwarze Magie habe sie angewandt und nicht mal die Kirche wolle den Bewohnern mehr helfen, obwohl man mit einem Brief um Hilfe gebeten hatte!

\red{\textbf{Interaktionen}}:

Die Gruppe sollte versuchen den wütenden Mob zu beruhigen. Im Gespräch werden die Leute im Mob sich immer wieder auf vier Argumente berufen:

\begin{itemize}
  \item Ruth soll wiederholt dabei gesehen worden sein, wie sie nachts tote Körper in die große Scheune gebracht habe.
  \item Ruth habe den Brief an die Kirche, welchen sie dem Totensammler mitgeben sollte, nie übergeben.
  \item Ruth habe die Pest gebracht. Sie war in Hamburg gewesen, um Korn zu kaufen.
  \item Kaum war sie zurück, brach die Krankheit aus! Zu diesem Zeitpunkt sei noch nirgends sonst in der Gegend jemand erkrankt!
\end{itemize}

\red{\textbf{Probe auf Beruhigen}}:

\begin{itemize}
  \item Erfolg \\
  Schaffen Sie es den Mob zu beruhigen, dann zieht sich der Mob zurück und sie können mit Ottilde in Ruhe das Haus betreten. Dort treffen sie auch auf Ruth, die verängstigt in einer Ecke sitzt.
  \item Misserfolg \\
  Schaffen sie es nicht den Mob zu beruhigen, dann bricht dieser in das Haus ein. Wollen die Abenteurer den Mob hindern kommt es zum Kampf (\blue{\ref{kampf}}) mit Bauern. Im Haus fehlt jedenfalls so oder so jede Spur von Ruth. Ottilde ist dankbar für ihre Hilfe und verrät der Gruppe, dass Ruth sich in der großen Scheune verstecke.
\end{itemize}

Ruth wird der Gruppe kleinlaut erzählen, dass sie nichts gemacht habe und nicht wisse warum die Bauern so aufgebracht seien

\red{\textbf{Probe auf Menschenkenntnis}}:

Ruth lügt. Hakt die Gruppe nach, dann erzählt sie, ihnen die gesamte Geschichte. \\
Sie habe direkt am Nikolaifleet Korn holen wollen. Da sei es am günstigsten, habe man ihr gesagt. Als sie dort ankam, sei es bereits spät gewesen. Sie habe am Lagerhaus, das man ihr beschrieben hatte, angeklopft, allerdings ohne Erfolg. Also habe sie durchs Fenster geschaut und einen Mann erblickt, der sich über etwas gebeugt habe. Er soll eine Art Vogel-Maske und einen weiten Mantel getragen haben. Ruth sagt, sie habe Angst gehabt und habe davonlaufen wollen. Sie sei auf dem Schnee ausgerutscht und habe das Gleichgewicht verloren. Dabei sei mit dem Kopf aufgeschlagen und im Lagerhaus wieder zu sich gekommen. Der Mann habe sich ihr dann als Didrich vorgestellt. Er forsche an der Pestilenz, habe er erklärt. Nur darum sei er so gekleidet und habe sich an der Leiche zu schaffen gemacht. Ruth habe ihm dann erzählt, dass in Eeksdurf zwar niemand krank sei, aber dass der Totensammler regelmäßig mit den Leichen durch ihr Dorf fahre, um diese tief im Eekshult zu verscharren. Daraufhin habe Didrich ihr einen Handel vorgeschlagen, und... naja... Ruth habe angenommen. \gqm{\textit{Wieso auch nicht. Die sind schließlich tot und ich arm. Eins davon kann man wenigstens noch ändern.}}
Die Leichen habe sie also von da an immer in ein kleines Lagerhaus im Armenviertel Hammerbrook gebracht. Dort habe auch ihr Geld gelegen...

Sollte die Gruppe nachfragen, kann Ruth ihnen genau sagen, in welchem Lagerhaus sie die Leichen ablegt.

\subsubsection{Option 2 – Ein großes Feuer}
\label{feuer}

\red{\textbf{Szene}}:

Ihr folgt den Geräuschen und gelangt an eine kleine Hütte vor dem Waldrand. Die Hütte macht einen verfallenen Eindruck, scheint aber noch bewohnt zu sein. Es ist weit und breit niemand zu sehen, vor der offenen Tür der Hütte seht ihr jedoch ein gewaltiges Feuer. Im Feuer zeichnen sich die verkohlten Überreste zweier Menschen ab.

Die Gruppe kann wenn gewünscht die Hütte betreten.

\textbf{Raumbeschreibung}: Das Innere der Hütte macht genauso wenig her wie die Fassade vermuten lässt. In einer Ecke des einzigen Raumes ist Stroh auf dem Boden ausgebettet. Ein morscher Holztisch und zwei Stühle stehen in der anderen Ecke. Das einzige herausstechende im Raum ist ein Regal, dass eine ganze Wand einnimmt, und auf dem getrocknete Kräuter säuberlich aufgereiht sind.

Bei genauerem Hinsehen findet die Gruppe ein Blatt Pergament, auf dem eine Art Wegbeschreibung gekritzelt zu sein scheint - nur sehr grob. Der Weg führt zu einem Ort im Armenviertel Hammerbrook, mehr lässt die Karte nicht erkennen. Außerdem finden die Abenteurer Geld. Etwas mehr Geld als jemand der hier lebt besitzen sollte.

Beim Verlassen der Hütte trifft die Gruppe auf Wolfgang (\blue{\ref{Wolfgang}}) und Hermann (\blue{\ref{Hermann}}). Diese werfen ihnen umgehend vor, mit der bösen Zauberin im Bunde zu sein und werden sie angreifen (gehe zu \blue{\ref{kampf}}) falls die Gruppe nicht schafft die beiden zu beruhigen.

\red{\textbf{Probe auf Beruhigen o.ä}: Die Bauern erzählen, weshalb sie Otthilde und Ruth verbrannt haben}

\begin{itemize}
  \item Ruth soll wiederholt dabei gesehen worden sein, wie sie nachts tote Körper in die große Scheune gebracht habe.
  \item Ruth habe den Brief an die Kirche, welchen sie dem Totensammler mitgeben sollte, nie übergeben.
  \item Ruth habe die Pest gebracht. Sie war in Hamburg gewesen, um Korn zu kaufen.
  \item Kaum war sie zurück, brach die Krankheit aus! Zu diesem Zeitpunkt sei noch nirgends sonst in der Gegend jemand erkrankt!
\end{itemize}

\subsubsection{Kampf gegen die Bauern}
\label{kampf}

\begin{center}
  \begin{tabular}{lcc}

    \toprule
    Fähigkeit & \textbf{Wolfgang} & \textbf{Hermann} \\
    \midrule
    Leben & 70 & 70 \\
    Waffe & Fäuste (70) & Mistgabel (70) \\
    Schaden & 15 & 40 \\
    Parieren & 5 & 30 \\
    \bottomrule
  \end{tabular}
\end{center}



\green{\textbf{Moral}: Wie reagiert unsere Gruppe auf die Situation in Eeksdurf?}

Ende des relevanten Eeksdurf-Plots.

Hafen (gehe zu \blue{\ref{Hafen}}) \\
Hammerbrook (gehe zu \blue{\ref{arm}}) \\
Die Kirche St. Petri (gehe zu \blue{\ref{Petri}}) \\
Der Nikolaifleet (gehe zu \blue{\ref{Fleet}}) \\

\newpage

\section{\textbf{In Hammerbrook}}
\label{arm}
%!TEX root = ../main.tex

\brown{\textbf{Pestilenz}: Jeder, der das Armenviertel betritt, erhält +1 Pestilenz.}

\red{\textbf{Szene}}:

Ihr kommt also in Hammerbrook an, einem der Stadtviertel in denen sich die Ärmsten der Stadt zusammenpferchen in der Hoffnung vom Reichtum zu profitieren den der Handel in Hamburgs Kassen spült. Dicht gedrängt leben hier Hafenarbeiter, einfache Leute und anderes Gesindel in ärmlichen Hütten und Häusern. Man merkt, dass hier andere Regeln zu gelten scheinen als im Rest der Stadt, die Straßen sind gesäumt von Toten und Kranken, und in kaum einem Haus brennt Licht. Der Tod geht um, und tagtäglich rechnen hier viele mit dem schlimmsten.

\subsection{Auf den Straßen}
\label{strasse}

\red{\textbf{Szene}}:

Die Straßen sind verschneit und verlassen. Nur vereinzelt ziehen vermummte Gestalten umher und werfen unserer Gruppe argwöhnische Blicke zu. Ihr scheint hier nicht willkommen zu sein.

\red{\textbf{Information für den Spielleiter}: Sollte einer der Spieler in auffallend feinen Kleidern durch das Armenviertel wandern wird die Gruppe später überfallen. (gehe zu} \blue{\ref{kampf2}}\red{).}

Plötlich kommen mehrere verhärmte Kinder unter lautem Geschrei auf euch zu. Sie haben einen unförmigen Lederklumpen dabei und scheinen ganz offensichtlich Fußball zu spielen. Als sie auf eurer Höhe sind rempeln einige der größeren Kinder euch an und murmeln eine Entschuldigung, bevor sie wieder in einer Seitengasse verschwinden.

\red{\textbf{Probe auf Gassenwissen o.ä.}: Alle würfeln eine Probe. Wer diese erfolgreich besteht bemerkt, dass ihm ein Gegenstand geklaut wurde.}

Während die Spieler das bemerken (oder auch nicht!) biegen die Kinder gerade aus ihrem Sichtfeld in eine kleine Seitengasse ein. Die Spieler können die Kinder jedoch noch versuchen mit einer Probe auf Rennen (um 10 erschwert) die Kinder zu verfolgen.

\begin{itemize}
  \item Erfolg: Holen sie die Kinder ein (gehe zu \blue{\ref{eingeholt}}).
  \item Misserfolg: Die Gegenstände sind verloren (gehe zu \blue{\ref{neingeholt}})
\end{itemize}

\subsection{Option 1 - Beim Lumpensammler}
\label{eingeholt}

\red{\textbf{Szene}}:

Ihr verfolgt also die Kinder bis diese die Tür zu einem Laden aufstoßen und darin verschwinden. Wenige Momente später betretet auch ihr außer Atem den Laden. Von den Kindern ist keine Spur zu sehen. Jedoch begrüßt der Lumpensammler Luis (\blue{\ref{Luis}}) überschwänglich die neue \gqm{Kundschaft}.

Auf Nachfrage stellt sich heraus, dass der Lumpensammler die Gegenstände der Spieler hat, aber nicht ohne weiteres herausgeben will. Nun hat die Gruppe folgende Optionen:

\begin{itemize}
  \item Probe auf Einschüchtern, um 30 erschwert (gehe zu \blue{\ref{fertig}})
  \item Die Gruppe geht wieder (gehe zu \blue{\ref{neingeholt}})
  \item Der Lumpensammler möchte mit den Spielern wetten (gehe zu \blue{\ref{wette}})
\end{itemize}

\subsubsection{Das Rätsel}
\label{wette}

Lumpensammler: \gqm{\textit{Ich hab es! Wir schließen eine Wette ab, ich stelle euch ein Rätsel das es zu lösen gilt. Wenn ihr es löst gebe ich euch eure Sachen wieder, schließlich können so piekfeine Schnösel wie ihr es seid bestimmt gut ein Rätsel lösen?}}

Voller Hochmut und Vorfreude trägt der Lumpensammler das Rätsel vor:

Lumpensammler: \gqm{\textit{Einst wurde ein Pirat gefasst der dem König sein Gold gestohlen hatte. Der König tobte und verlangte, dass der Pirat unverzüglich am Galgen aufgeknüpft werde. Es war aber üblich, den zum Tode verurteilten Dieben eine letzte Chance zu geben und Gott über sie richten zu lassen. Daher mussten sie aus einem schwarzen Säckchen einen Stein ziehen. Im Säckchen befanden sich immer genau ein weißer und ein schwarzer Stein. Zog der Dieb den weißen Stein, wurde ihm die Freiheit geschenkt. Zog er hingegen den schwarzen Stein, so baumelte er.  Eines Tages kam nun jener Pirat der einst das Gold des Königs geraubt hatte vor den Scharfrichter und wartete auf sein Gottesurteil.}}

Lumpensammler: \gqm{\textit{Der König aber wollte sichergehen, dass der Halunke hängt und hat dem Henker am Abend zuvor im Heimlichen befohlen zwei schwarze Steine in das Säckchen zu legen.
So ging der König am nächsten Tage also voller Zuversicht also zum Richtplatz, wo überall weiße und schwarze Steine herumlagen. Als es Zeit wurde für den Verurteilten sein Urteil zu erhalten bückte sich der Henker und nahm zwei schwarze Steine vom Boden auf, die er im Säckchen ablegte. Der Pirat sah dies jedoch und wusste somit, dass er kein gerechtes Urteil erhalten würde. Er glaubte die Schlinge schon um seinen Hals als ihm die rettende Idee kam. \\
Er zog einmal und musste freigelassen werden. \\ Was war es, dass dem Piraten das Leben rettete?}}

\begin{itemize}
  \item Erfolg: Die Lösung des Rätsels besteht darin, dass der Dieb einen Stein zieht und ihn sogleich wegwirft. Da der verbleibende Stein im Sack schwarz ist, muss der gezogene Stein scheinbar weiß gewesen sein. Beantwortet die
  \item Misserfolg: Können sie das Rätsel nicht lösen oder liegen falsch wird der Lumpensammler ihnen ihre Gegenstände nicht zurückgeben, er genießt seinen Triumph viel zu sehr!
\end{itemize}

\subsubsection{Abschließen der Aufgabe}
\label{fertig}

Ungeachtet der vorigen Ereignisse wird der Lumpensammler ihnen erzählen, dass sie nicht die ersten wohlbetuchten Personen sind, die in den letzten Tagen in seinem Laden waren. Es ist noch nicht allzu lange her, da sei einer in einem ganz komischen Aufzug hier hereingeschneit. Er habe nach abstrusen Pflanzen gefragt. Chillies und Zitronen seien darunter gewesen. Als ob man sowas hier bekomme. Außerdem habe er noch einen Blasebalg gewollt und etwas grobes Leinen, Lumpen und Teer. Komischer Typ.

Er kann den Abenteurern jedoch die grobe Richtung beschreiben in die der Kunde verschwunden ist. Nachdem er ihnen das erzählt hat drängt Luis die Gruppe zu gehen, er hat noch andere Kundschaft um die er sich kümmern muss. Wieder auf der Straße stehend können die Spieler nach der Person fragen die der Lumpensammler ihnen beschrieben hat oder dem vorgeschlagenen Weg folgen. Nach einem kurzen Fußmarsch kommen sie dann an einem Lagerhaus an (gehe zu \blue{\ref{Lagerhaus}})

\subsection{Das Treffen mit Sigrun}
\label{neingeholt}

\red{\textbf{Szene}}:

Ihr probiert noch den Kindern nachzueilen, doch als ihr um die Ecke biegt hinter der ihr sie eben noch verschwinden sehen habt ist keine Spur mehr zu finden, sie scheinen sich in Luft aufgelöst zu haben.

Jedoch schaut euch eine dicklichere Frau verwundert an.

Frau: \gqm{\textit{Ihr wurdet wohl von den Kindern hier ausgeraubt? Findet euch lieber schnell mit dem Verlust ab, in den Gassen hier habt ihr keine Chance sie noch irgendwo aufzufinden.}}

Im weiteren Verlauf des Gesprächs stellt sie sich als die Bäckerin Sigrun (\blue{\ref{Sigrun}}) vor. Sie folgt dem Ruf von Pater Salus und ist gerade auf dem Weg nach St. Petri. Dort sind unzählige Menschen und keiner von ihnen ist krank. Ein sicherer Aufenthalt, so scheint es. Allerdings müsse sie erst in Erfahrung bringen, ob auch ihre Kinder dort Willkommen seien, da sie mit Fieber zu Hause im Bett liegen. Sigrun bittet die Gruppe, bei ihr zu Hause vorbeizuschauen und nach den Kindern zu sehen während sie weg ist.

\violet{\textbf{Moral}: Wie wird die Gruppe mit der Bitte von Sigrun umgehen?}

\begin{itemize}
  \item Die Gruppe hilft den Kindern nicht, Sigrun zieht niedergeschlagen davon, und die Gruppe bleibt auf sich gestellt.
  \item Sie helfen den Kindern, Sigrun weist ihnen den Weg zu ihrem Haus und verrät, wo der Zweitschlüssel hinterlegt ist.
\end{itemize}

\subsubsection{Bei Sigrun Zuhause}

\red{\textbf{Szene}}:

Nach einem kurzen Fußmarsch steht ihr vor Sigruns Haus. Vor der Türe des Hauses steht ein Gestell, auf dem Brote zum Verkauf angeboten werden können, momentan ist es jedoch leer. Ansonsten gibt es nicht viel zu sehen. Ein unangenehmer Geruch liegt in der Luft. Öffnen die Spieler die Türe wird der Geruch stärker.

\brown{\textbf{Pestilenz}: Alle die das Haus betreten erhalten +2 Pestilenz.}

In einem der Zimmer findet ihr zwei schweissnasse Kinder zugedeckt in einem Bett liegen. Eines der Kinder weint lautstark, das andere regt sich nicht.

\red{\textbf{Probe auf Medizin o.ä}: Den Kindern ist nicht mehr zu helfen, zumindest nicht mit euren Fähigkeiten.}

\brown{\textbf{Pestilenz}: Kümmern sich die Spieler weiter um die Kinder erhalten sie +1 Pestilenz.}

\subsubsection{Das Treffen mit Hagen}
\label{klopf}

\red{\textbf{Szene}:}

Plötzlich klopft es an der Türe. Ein blonder, kräftig gebauter staht auf der Straße und fragt die Gruppe wo Sigrun sei und was sie hier zu suchen hätten. Nach einer Erklärung stellt er sich als Hagen (\blue{\ref{Hagen}}) vor und läd sie zu sich nach Hause ein, er habe ihnen etwas spannendes zu erzählen! Folgt ihm die Gruppe gelangen sie nebenan zu Hagens Haus.

\textbf{Raumbeschreibung}: Hagens Haus ist aufgeräumt. Nirgendwo liegen Waren oder ähnliches herum und auch sonst ist es nur kärglich eingerichtet Einige Kerzen spenden ein wenig Helligkeit und werfen lange Schatten auf einen Holztisch in der Mitte des Hauses. Hagen fordert die Gruppe auf sich um den Tisch zu setzen.

\red{\textbf{Information für den Spielleiter}: Hat Hagen die Gruppe nicht zu sich eingeladen (falls die Gruppe ihn beispielsweise von selbst aufsucht), dann verlangt er im Gegenzug für Informationen einen Aufseherposten innerhalb der Hanse. Die Gruppe kann zustimmen oder ablehnen. Hauptsache sie kommen an die Infos.}

Hagen berichtet, dass er seit geraumer Zeit heimlich Waren abzweige. Er sei nicht stolz darauf, aber man müsse eben sehen, wo man bleibt. Jedenfalls komme er nachts am Nikolaifleet an diese Waren. Dort arbeite er. Als immer mehr seiner Kollegen krank wurden – das startete bereits vor vier Wochen, also vor allen anderen Ausbrüchen – wurde das sogar noch leichter. Eines Nachts jedenfalls schlich er sich wieder ins Lager, als ihm ein eigenartiger Mann begegnete. Er trug eine lange Maske, die beinahe wie der Schnabel eines Vogels aussah, ein weites Gewand und einen Stock bei sich. Außerdem roch es nach... Parfüm. Der Mann floh, als er Hagen sah, und ließ nichts außer einer Rattenfalle zurück. Diese war aber leer. Auf Nachfrage kann Hagen ihnen eine Wegbeschreibung zum Lagerhaus am Nikolaifleet geben. Unter keinen Umständen will er die Gruppe jedoch begleiten. Auch mit Gewalt oder Drohungen kann man ihn nicht überzeugen.

Ende des relevanten Hammerbrook-Plots.

Weiter mit:

Hafen (gehe zu \blue{\ref{Hafen}}) \\
Eeksdurf (gehe zu \blue{\ref{xd}}) \\
Die Kirche St. Petri (gehe zu \blue{\ref{Petri}}) \\
Der Nikolaifleet (gehe zu \blue{\ref{Fleet}}) \\

\subsection{Kampf gegen Räuber}
\label{kampf2}

Handelt es sich um einen Überfall und die Räuber können sich unbemerkt anschleichen und einen Angriff ohne Verteidigungsmöglichkeit auswürfeln, diese verursacht zudem den doppelten Schaden eines normalen Angriffes.

\begin{center}
  \begin{tabular}{lc}
    \toprule
    Fähigkeit & Punkte \\
    \midrule
    Leben & 80 \\
    Fäuste & 70 \\
    Schaden & 15 \\
    Parieren & 5 \\
    \bottomrule
  \end{tabular}
\end{center}

Sinkt das Leben der Räuber unter 30 Punkte werden sie versuchen zu fliehen (Würfle 1W100, bei einer Augenzahl von 60 oder weniger gelingt den Räubern die Flucht).
Sind die Räuber besiegt so können die Spieler ein Dolch bei einem der beiden finden. Ansonsten findet sich nichts von Wert.

\subsection{Didrichs Lagerhaus}
\label{Lagerhaus}

\subsubsection{Vor dem Lagerhaus}

\red{\textbf{Szene}}:

Das Lagerhaus, das Ruth euch beschrieben hatte, steht recht nahe der Stadtmauer, gerade vor den Toren der wohlhabenden Stadt. Es sieht hier noch erbärmlicher aus als sonst wo im Armenviertel. Durch die Ritzen zwischen der Holzverkleidung strahlt jedoch ein wenig Helligkeit, im Inneren scheint Licht zu brennen.

Bei genauerem Betrachten entdecken die Spieler ein Loch im Dach des Lagerhauses

\red{\textbf{Interaktionen}:}

Die Gruppe kann nun auf zwei Arten versuchen in das Lager einzudringen:

\begin{itemize}
  \item \textbf{Tür aufbrechen (Probe auf Körperkraft o.ä)}: Wenn sie einbrechen, tauchen zwei kräftige Männer auf, die sehr skeptisch sind. Ablenkung ist gefragt! Schafft die Gruppe nicht das Misstrauen der beiden zu vertreiben werden sie die Stadtwache holen oder gegen die Spieler kämpfen (gehe zu \blue{\ref{kampf2}})
  \item \textbf{Durch das Loch im Dach klettern (Probe auf Klettern, erschwert)}: Vom Dach müssen die Spieler ins Innere springen, bestehen sie dabei eine Probe auf Geschick o.ä nicht nehmen sie 10 Schaden.
\end{itemize}

\subsubsection{Im Inneren}

\textbf{Raumbeschreibung}: Im Lagerhaus ist es eiskalt. Allerdings brennt eine riesige Öllampe, die auf einem Tisch steht, auf dem allerlei Dinge liegen, unter anderem eine tote Ratte, eine Zitronenschale und etwas von einem roten Pulver, das einen scharfen Geruch verströmt (Chili).

Außerdem führen drei Türen zu weiteren Räumen. Bei allen sind jegliche Ritze und Schlitze säuberlich mit Lumpen und Lappen verstopft und verteert.

\begin{itemize}
  \item \textbf{Linke Türe}: Die linke Türe ist nur angelehnt und scheint offen zu stehen. Öffnen sie die Türe finden sie üppige Kornvorröte. Auf den Säcken steht \gqm{De groote Buur} geschrieben.
  \item \textbf{Mittlere Türe}: Die mittlere Türe ist geschlossen. Öffnen die Spieler diese erhalten sie \brown{+2 Pestilenz}. Hinter der Türe liegt ein Raum der bis auf einen großen Tisch in der Mitte leer ist. Auf dem Tisch liegt nackt ein Toter mit aufgeschnittenem Bauch. Auch in den Ecken liegen mehrere Tote, die meisten von ihnen von der Pest gezeichnet.
  \item \textbf{Rechte Türe}: Auch die rechte Türe ist verschlossen. Öffnen sie diese strömen Ratten heraus. Sie sind ausgehungert und aggresiv. Sie greifen die Gruppe an! (gehe zu \blue{\ref{kampf3}})
\end{itemize}

\subsubsection{Kampf gegen Ratten}
\label{kampf3}

Der Kampf gegen die Ratten läuft wie ein normaler Kampf gegen jeden anderen Gegener ab. Nun besteht der Gegner jedoch aus 10 Ratten mit jeweils 10 Lebenspunkten. Für Angriffe der Ratten würfle 1W10, das Ergebnis ist die Anzahl an Ratten, die die Spieler angreifen und pro Biss jeweils 4 Schaden austeilen. Werden die Ratten angegriffen so versuchen sie auszuweichen (Ausweichen: 10).

\brown{\textbf{Pestilenz}: Alle Spieler die gebissen werden erhalten einmalig +1 Pestilenz.}

Sind alle Ratten besiegt können die Spieler in den rechten Raum eintreten. Auch hier finden sie Getreidesäcke auf denen \gqm{De groote Buur} geschrieben steht.

Ende des relevanten Hammerbrook-Plots.

Weiter mit:

Hafen (gehe zu \blue{\ref{Hafen}}) \\
Eeksdurf (gehe zu \blue{\ref{xd}}) \\
Die Kirche St. Petri (gehe zu \blue{\ref{Petri}}) \\
Der Nikolaifleet (gehe zu \blue{\ref{Fleet}}) \\

\newpage

\section{\textbf{Bei St. Petri}}
\label{Petri}
Die Kirche St. Petri
Einleitung
\red{\textbf{Szene}}:

Die Kirche St. Petri steht ganz im Zeichen des neuen Hamburger Wohlstands. Noch immer wird gebaut, aber man sieht ihr schon jetzt an, dass sie eines der zukünftigen Wahrzeichen der Handelsmetropole sein wird.

Viele Menschen pilgern angesichts ihrer Machtlosigkeit gegenüber der Pest hierher, um Schutz und Trost zu suchen. Kranke und Verzweifelte säumen die Gänge, und der Geruch von Tod liegt in der Luft.


Ankunft an der Kirche
\red{\textbf{Szene}}:

Ortsbeschreibung:Sie stehen vor der gewaltigen Pforte einer der schönsten Kirchen Hamburgs. In neuem Glanz erstrahlt St. Petri und ist aufgrund der Lage auf einem Geestrücken bereits von Weitem sichtbar.

Ereignis: Sobald jemand an die Tür klopft werden sie von Helfern begrüßt und hereingelassen.

Interaktionen:

Probe auf Wahrnehmung, Aufmerksamkeit, Kirchenwissen oder ähnliches:

Sie tragen zwar Mönchsgewänder, scheinen aber keinem Orden zugehörig zu sein.
Zumindest nicht erkennbar.
Nach Pater Salus fragen:

Wird einer der Helfer nach Pater Salus gefragt, so wird ihnen berichtet, dass dieser aufgebrochen sei, um Gerüchten auf den Grund zu gehen.
Nun ist von schwarzer Magie die Rede, böse Mächte seien in der Stadt am Werk. Die Gruppe ist aber herzlich eingeladen zu bleiben, sofern sie gesund sind und dies beschwören.
\red{\textbf{Szene}}:

Ereignis: Dann lässt man sie in den Hauptraum der Kirche. Hier sitzen überall Menschen mit besorgten Gesichtern herum. Andere wiederum blicken voller Zuversicht und lauschen gebannt Gesprächen, Gesängen und Predigten, welche Mönche überall im Raum zum Besten geben. Außerdem gehen verdächtig viele Klingelbeutel herum. Es führen einige Türen aus dem Hauptraum. Alle sind verschlossen.

\textbf{Raumbeschreibung}: Das innere einer großen Kirche. Man sieht, dass die Bauarbeiten noch im Gange sind, doch im Hauptraum kann man sich bereits ohne größere Probleme aufhalten. Trotz Bau lässt sich bereits eine prunkvolle Kirche erahnen.

Interaktionen:

Probe auf Wahrnehmung,Aufmerksamkeit, Lauschen oder ähnliches:

Hört die Gruppe bei den Mönchen genauer hin, ist die Rede vom Jüngsten Gericht,
Sünde und der Strafe des Herren, die alle Sünder trifft.
Man könne sich aber von seinen Sünden befreien.
Nur etwas weltliches Gut müsse man opfern, um hier,
bei den Frommen, bleiben zu dürfen.
Gespräch mit Anwesenden:

Fragen sie die Leute, so berichten manche davon, dass sie mit Angehörigen hier waren.
Die wurden dann aber irgendwann nach hinten gebeten, wohl weil sie besonders fromm waren. Während sie so lauschen, bemerken sie, dass immer wieder einzelne Menschen aus dem Saal durch die Seitentüren gebracht werden.
Man will sie hier aber nicht durchlassen. Das seien „private Gemächer“ der Priester und Mönche. Die Gruppe kann versuchen, sich Zugang zu verschaffen.
Im Hinterzimmer
\red{\textbf{Szene}}:

Schaffen sie es irgendwie hier hinein, geht es erstmal durch ein paar Gänge.

Es stinkt nach Tod. Alle erhalten Pestilenz +1.

Nach einer Weile kommen sie in eine Art Gewölbe. Hier liegen unzählige Kranke, von gerade so infiziert bis zu bereits verstorben, hinter einem Gitter eingesperrt wie in einer Zelle.

Gehen sie näher an die Menschen heran, erhalten sie Pestilenz +3.

Sie alle sind überzeugt davon, dass sie das hier als Sünder verdient haben.

Am hinteren Ende des Raumes gibt es eine Tür. Sie scheint ins Freie zu führen. Davor am Gitter steht eine junge Frau, Gundel. Sie ist bereits von der Pest gezeichnet, winkt die Gruppe aber zu sich.

Gespräch mit Gundel: Sie bittet die Gruppe, ihr dabei zu helfen zu entkommen. Sie müsse zu diesem Arzt, diesem Didrich. Der wisse, was zu tun ist. Da ist sie sich sicher! Gundel arbeitet am Nikolaifleet. Dort kocht sie für die Männer. Nach und nach wurden sie alle krank.

Plötzlich tauchte der Arzt auf, Didrich von Sinnfeld. Er versprach ihr, wenn sie ihm ein paar Ratten aus der Küche fange, würde er sie fürstlich bezahlen. Sie lehnte ab. Das schien ihr doch sehr merkwürdig, und die Bezahlung sei nicht üppig gewesen. Als er aber sagte, dass er an etwas arbeite, um den Schwarzen Tod zu besiegen, lenkte sie ein. Sie gab ihm die Ratten. Doch dann wurden immer mehr Leute krank. Sie bekam es mit der Angst zu tun und kam hierher. Doch es war zu spät. Man erleichterte sie um ihr Geld und sperrte sie dann hier zum Sterben hinein.


Moral:

Lassen sie Gundel und die anderen raus und gefährden damit die Stadt? Oder gehen sie und überlassen die Kranken ihrem unumgänglichen Schicksal?

Vorm Ausgang
\red{\textbf{Szene}}:

Ereignis: Hier treffen sie noch einmal den Totensammler. Er schaut sie vorwurfsvoll an und zieht dann weiter klingelnd seine Runden.


Ende des relevanten St.Petri-Plots.

\newpage

\section{\textbf{Am Nikolaifleet}}
\label{Fleet}
\red{\textbf{Szene}}:

Erst seit wenigen Jahren wächst am Nikolaifleet ein gewaltiges Lagerzentrum für die Waren aus aller Welt. Der neue Reichtum trägt hier Früchte, und in direkter Nähe zu Tee, Gewürzen und Tulpen siedeln sich die betuchteren Bürger Hamburgs in prächtigen Villen an. Alles scheint makellos und vom Chaos der übrigen Stadt unberührt. Fast schon zu makellos. Weißer Schnee fällt auf frisch gepflasterte Straßen. Niemand ist zu sehen, obwohl hier reges Treiben herrschen sollte.

\subsection*{Der Handelsregister}
\label{handelsregister}

\red{\textbf{Szene}}:
Das Handelsregister ist ein fein herausgeputztes Gebäude. Die Fassade ist gesäumt von in Metall gefassten Glasfenstern. Eine geöffnete Türe führt ins Innere des Gebäudes, in die hin und wieder Kaufläute ein und aus gehen.

Nach dem Eintreten führt ein kurzer Gang in einen Raum, in dessen Mitte an einem imposanten Schreibtisch eine Beamtin sitzt.
Sie reagiert forsch auf Fragen der Gruppe, kann jedoch mit einem passenden Talent überredet werden Informationen über die Häuslichkeiten am Nikolaifleet preiszugeben.

\red{\textbf{Probe auf passendes Talent:} Die Beamtin erzählt, dass die Geschäfte und das Leben am Nikolaifleet seit dem Ausbruch des schwarzen Todes stark zurückgegangen sind. Nur ein Herr Didrich von Sinnfeld lebe noch in einem Haus gleich am Fleet.}

Das Handelsregister ist ein fein herausgeputztes Gebäude. Die Fassade ist gesäumt von in Metall gefassten Glasfenstern. Eine geöffnete Türe führt ins Innere des Gebäudes. Im Inneren sitzt an einem
massiven Eichentisch eine Beamtin. Die Gruppe kann sie um Auskunft fragen über den Eigentümer des „Grote Buur“ und Didrich von Sinnfeld einzuholen.

\subsection*{Grote Buur}
\label{"grote buur"}

\red{\textbf{Szene}}:

„Grote Buur“ steht in eisernen Lettern über dem Eingang eines mehrstöckigen Lagerhauses. Es besticht gegenüber den anderen Lagerhäusern durch sein Aussehen und scheint erst kürzlich weiß getüncht worden zu sein. Eine große Eichentür versperrt den Blick ins Innere.

\red{\textbf{Interaktion}}:

Die Gruppe kann probieren die Türe des Lagerhauses zu öffnen, überraschenderweise ist diese nur angelehnt. Auch kompliziertere Methoden (Klettern, Tür eintreten, etc.) führen zum Ziel, erregen aber unter Umständen die Aufmerksamkeit der Anwohner...

\textbf{Raumbeschreibung}: Es handelt sich wohl um ein typisches Lagerhaus. Auf der Rückseite befinden sich die Tore, die zum Löschen der Handelsboote genutzt werden. Momentan sind diese jedoch verschlossen. Das Lagerhaus hat gleich mehrere Etagen, und es gibt diverse Kräne und Seilzüge, um Waren auf den Ebenen zu transportieren. Bei näherer Betrachtung erkennt man, dass sich hier einige Ratten niedergelassen haben. Bei genauerer Betrachtung entdecken die Spieler Korn, Tücher und andere Handelswaren, allerdings nicht so viele, wie ein Lagerhaus von dieser Größe vermuten lassen würde.

\red{\textbf{Ereignis}}: Das Lagerhaus scheint alles in allem verlassen zu sein. Doch dann hören die Spieler ein Rumpeln, das aus den Untergeschossen stammt. Dort angekommen finden Sie eine Luke die mit einem Metallschloss befestigt ist.

\red{\textbf{Probe auf Kraft oder Schlösser knacken o.ä.}: Das Schloss lässt sich von der Luke entfernen, die sich jetzt problemlos öffnen lässt.}

Unter der Luke sitzt ein gefesselter Mann. Er hat überall am Körper Wunden, die nicht nur von der Pest zu stammen scheinen. Ihm fällt das Sprechen sehr schwer, aber es sind schwach die Worte „Sinnfeld“ und „gegenüber“ zu hören, bevor der Mann in Ohnmacht fällt.

\brown{\textbf{Pestilenz:} Die Gruppe kann probieren ihn zu retten bzw. untersuchen, erhält dann aber +1 Pestilenz}.

Eine nähere Inspektion legt offenbar, dass die Wunden an seinem Körper unter anderem von Rattenbissen stammen, außerdem entdeckt die Gruppe einige Schnittwunden entlang der schwarzen Beulen. Einige Flöhe krabbeln auf dem Boden herum. Abgesehen von dem Mann gibt es sonst nichts zu entdecken in dem Lagerhaus.

\subsection*{Sinnfelds Haus}
\label{sinnfelds haus}
\red{\textbf{Szene}}:

Dem "Groote Buur" gegenüber steht ein ansehliches Herrenhaus, dem es in Architektur und Größe ähnelt. Im zweiten Stock befinden sich große Fenster in eingefassten Metallrahmen. Das Haus wirkt alles in allem gepflegt.

\red{\textbf{Ereignis}: Die Gruppe kann an die Tür und anklopfen. Als sie klopfen wird ihnen nach kurzer Zeit geöffnet.}

\subsubsection{Eine Frau öffnet}
\label{frau öffnet}
\red{\textbf{Ereignis}: Klopfen sie an, öffnet die Magd Traudel.}

Ein Gespräch mit Traudel legt offen, dass im Haus ein Herr von Sinnfeld wohne, Didrich von Sinnfeld. Ja, er sei zu Hause. Die Gruppe möge eintreten! Die Magd führt sie in den Salon, wo Sie gebeten werden, zu warten.

\red{\textbf{Ereignis}: Nach einer Weile hören sie einen Schrei. Dann rennt das Hausmädchen an ihnen vorbei zur Tür hinaus.}

Geht die Gruppe dem nach, finden sie Didrich in seinem Arbeitszimmer. Dieser beugt sich gerade über den leblosen Körper eines kleinen Jungen und ist im Begriff, ihn zu sezieren.
Als er aufblickt und die Gruppe ihn entdeckt entwickelt sich ein Gespräch:

Didrich: \gqm{\textit{Ein Kleingeist, die gute Traudel. Viel zu leicht zu erschrecken. Und nun?! Habe ich nichts als Ärger am Hals. Denn die werten Herrschaften, so nehme ich an,
stehen meinem Treiben hier ebenso wenig wohlgesonnen gegenüber wie andere Vertreter ihrer Stände. Sehe ich das richtig?}}

Didrich schildert ihnen, dass er an der Krankheit forsche. Er sei kurz vor einem Durchbruch. Er wisse nun mit Sicherheit, dass es etwas mit Ratten zu tun habe. Nur der genaue Ablauf der Ansteckung sei ihm noch schleierhaft. Aber ein Fehlen von Ratten in einer Stadt ginge unmittelbar mit einem Abhandensein der Seuche einher, obwohl durch Kontakt zu Kranken ebenfalls eine Seuche zustande kommen könne. Seine Forschungen seien nicht immer ganz lupenrein gewesen. Er brauchte tote Körper, später lebendige, das gebe er zu. Zum Glück gäbe es genug Kranke direkt hier am Fleet. Gleich gegenüber fielen die Arbeiter reihenweise um. Aber dann blieben sie zu Haus. Ein Besuch auf Hammerbrook sei also unumgänglich geworden. Dann noch einer. Und noch einer. Jedenfalls könne man solche Forschungen in einer kleingeistigen Stadt wie Hamburg nicht ohne Weiteres betreiben. Daher die Heimlichtuerei. Was nun? Soll er seine Forschungen fortsetzen?

\violet{\textbf{Moral}: Was tun sie mit Didrich und seinen Forschungen?}

Ende des relevanten Nikolaifleet-Plots.

Weiter mit:

Hafen (gehe zu \blue{\ref{Hafen}}) \\
Eeksdurf (gehe zu \blue{\ref{xd}}) \\
Hammerbrook (gehe zu \blue{\ref{arm}}) \\
Die Kirche St. Petri (gehe zu \blue{\ref{Petri}}) \\


\section{\textbf{Abschlussbericht}}
\label{Fertig}

Der Abschlussbericht
\red{\textbf{Szene}}:

Am letzten Tage, oder wenn sich die Gruppe eher im Stande sieht, folgt der Bericht vor dem Hamburger Rat. Die Abenteurer werden aufgefordert, ihre Ergebnisse zu präsentieren. Dabei stellt der Rat durchaus Fragen und ist kritisch. Außerdem wurden dem Rat Informationen über das Handeln der Gruppe zugetragen, gerade was die moralischen Fragen betrifft:

\begin{itemize}
  \item Wie ist die Gruppe mit Gorichs Familie umgegangen?
  \item Haben sie die Kranken in St. Petri befreit?
  \item Wie sind sie mit den Bauern Wolfgang und Hermann umgegangen?
  \item Was machten sie mit Didrich?
  \item Haben sie Sigruns Kindern geholfen?
\end{itemize}

Hat die Gruppe alle Fragen beantwortet, will sich der Rat beraten. Hier gibt es grundsätzlich drei Möglichkeiten zwischen denen der Spielleiter abhängig von der Leistung der Gruppe wählen kann.

1. Die Gruppe war sorgfältig. Setzen wir ihren Vorschlag um!
2. Die Gruppe war nicht überzeugend genug. Warten wir ab!
3. Die Gruppe war nachlässig. Sie sollen bestraft werden!
Die Entscheidung des Spielleiters beendet das Abenteuer.
