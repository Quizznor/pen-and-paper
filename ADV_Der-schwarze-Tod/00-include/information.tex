%!TEX root = ../adventure.tex

\thispagestyle{fancy-info}
\section*{Generelle Informationen}

\begin{itemize}
  \item \textbf{\textit{Wörtliche Rede}} \\
  Alles, was \textbf{\textit{kursiv}} geschrieben ist, soll wörtlich vorgetragen werden. Dabei handelt es sich in der Regel
  um wörtliche Rede in Gesprächen, in der sich unter Umständen wichtige Informationen verstecken können, oder eine
  stimmungsvolle Einleitung zu verschiedenen Abschnitten, beispielsweise ein Pro- oder Epilog.

  \begin{say}
  Franz rast in einem komplett verwahrlosten Taxi quer durch Bayern. Vogel Quax zwickt Johnnys Pferd Bim. Sylvia wagt quick den Jux bei Pforzheim.
  \end{say}

  \item \textcolor{RoyalRed}{\textbf{Interaktionen}} \\
  Pen and Paper lebt von der \textcolor{RoyalRed}{Interaktion} der Spieler mit ihrer Umwelt. Wann immer eine Probe auf ein Talent
  möglich ist ist diese Information \textcolor{RoyalRed}{dunkelrot} markiert. Am Beispiel unten sieht man dabei auf welches Talent
  gewürfelt werden kann (Lesen), und ob dieser Wurf beeinflusst ist. Die Zahl in Klammern ($-10$) wird dabei immer zum Talentwert
  des Spielers dazuaddiert, ein negativer Wert bedeutet also, dass der Wurf erleichtert ist.

  \begin{probe}{Lesen}{-10}
  Bei erfolgreicher Probe geschieht, was hier steht.
  \end{probe}
  \vspace{-5pt}

  \item \textcolor{violet}{\textbf{Charaktere}} \\
  Während der Handlung treffen die Spieler auf verschiedene \textcolor{violet}{\textbf{Charaktere}}. Referenzen zu Informationen über
  die jeweiligen Charaktere sind \textcolor{violet}{\textbf{violett}} hinterlegt. In den jeweiligen Abschnitten befinden sich
  Hintergrundinformationen wie Motivationen und Verhaltensweisen, etc. die der Person Leben einhauchen sollen. Hier ein
  Beispiel:

  \begin{centering}
    \noindent\makebox[\textwidth][c]{%
    \begin{minipage}{0.85\textwidth}
      \charakter{Beispielcharakter} sieht aus dem Fenster. In der Spiegelung des Fensters sieht er sich selbst,
      \charakter{Beispielcharakter}. Er dreht sich um und fragt sich, wer er eigentlich ist. Dann erinnert er sich. Er ist ganz eindeutig \charakter{Beispielcharakter}.
    \end{minipage}}
  \end{centering}

  \item \textcolor{RoyalBlue}{\textbf{Referenzen}} \\
  Zur einfachen Navigation sind an relevanten Stellen \textcolor{RoyalBlue}{Referenzen} eingebaut, mithilfe derer sich einfach und
  schnell zwischen verschiedenen Textpassagen hin und her springen lässt. Solche Referenzen lassen sich (in der pdf
  Version) anklichen und sind \textcolor{RoyalBlue}{blau} markiert.

  \vspace{-5pt}
  \begin{refbox}{die Einführung}

      Was steckt denn eigentlich in diesem Abenteuer? \then Gehe zu Seite \page{tableofcontents}. \\
      Finde heraus was in der Story geschieht \then Gehe zur Story auf Seite \page{adventure}.
  \end{refbox}
  \vspace{-5pt}

\end{itemize}
