%!TEX root = ./adventure.tex

\newcommand{\adventurename}{Der schwarze Tod}	% Name of adventure here
\newcommand{\mainauthor}{Hauke Gerdes}								% Name of author here
\newcommand{\coauthor}{Paul Filip}								% Name of coauthor here

% Ruleset
\newcommand{\ruleset}{How to be a hero}					% Name of ruleset
\newcommand{\ruleseturl}{www.howtobeahero.de}		% Webadress of ruleset

% info about adventure
\newcommand{\place}{Hamburg}											% Location of the adventure
\newcommand{\storytime}{1350 n. Chr.}							% Time in which adventure plays
\newcommand{\playercount}{drei bis vier}					% Number of players that can play
\newcommand{\difficulty}{Durschnittlich}						% Difficulty of the adventure
\newcommand{\duration}{etwa 5 Std.}								% Length it should take to complete

\newcommand{\quotelength}{13}										% Length of quote on story page
\newcommand{\quoteauthor}{Unbekannt}								% Author of adventure quote
\newcommand{\quotetext}{												% Quote text on story page
Wie viele tapfere Männer und Frauen saßen um den Morgen noch mit Ihren Kindern,
und speisten am Abend mit Ihren Vorfahren in der nächsten Welt. Welch Tragödie!
}

% Sectioning glossary
\newglossary[glg]{Beispiel}{gls}{glo}{Beispielcharaktere}
\newglossary[glg]{Personen}{gls}{glo}{Personen}

% aux pdf info
\hypersetup
{
	pdfauthor={\mainauthor},
	pdftitle={Pen and Paper: \adventurename},
	pdfkeywords={pen and paper,adventure,\mainauthor}
}
