%!TEX root = ../adventure.tex

\thispagestyle{fancy-info}
\section*{Generelle Informationen}

\begin{itemize}
  \item \textcolor{OrangeBoxFrame}{\textbf{Informationen}}

  Informationen die eventuell für die Handlung relevant sein können, von denen die Spieler a priori jedoch nichts wissen sollen sind sind \textcolor{OrangeBoxFrame}{orange}
  markiert. Selbstverständlich heißt das, dass die Spieler durch geschicktes Handeln oder Proben selber darauf kommen können.

  \begin{info-box}

  Psst! Was hier steht ist vorab nur für die Spielleiter gedacht.

  \end{info-box}

  \item \textcolor{OrangeBoxFrame}{\textbf{Referenzen}} \\
  Zur einfachen Navigation sind an relevanten Stellen \textcolor{OrangeBoxFrame}{Referenzen} eingebaut, mithilfe derer sich einfach und
  schnell zwischen verschiedenen Textpassagen hin und her springen lässt. Solche Referenzen sind wie Informationen für die Spielleiter ebenfalls
  \textcolor{OrangeBoxFrame}{orange} markiert.

  \vspace{-5pt}
  \begin{ref-box}{die Einführung}
      Was steckt denn eigentlich in diesem Abenteuer? \then Gehe zu Seite \page{tableofcontents}. \\
      Finde heraus was in der Story geschieht \then Gehe zur Story auf Seite \page{adventure}. \\
      Welche Personen kommen in diesem Abenteuer vor? \then Gehe zu Seite \page{characters}
  \end{ref-box}
  \vspace{-5pt}

  \item \textcolor{violet}{\textbf{Charaktere}} \\
  Während der Handlung treffen die Spieler auf verschiedene \textcolor{violet}{Charaktere}. Referenzen zu Informationen über
  die jeweiligen Charaktere sind \textcolor{violet}{violett} hinterlegt. In den jeweiligen Abschnitten befinden sich
  Hintergrundinformationen wie Motivationen und Verhaltensweisen, etc. die der Person Leben einhauchen sollen. Hier ein
  Beispiel:

  \vspace{16pt}
  \hspace{-0.7cm}
  \begin{centering}
    \noindent\makebox[\textwidth][c]{%
    \begin{minipage}{0.8\textwidth}
      \charakter{Beispielcharakter} sieht aus dem Fenster. In der Spiegelung des Fensters sieht er sich selbst,
      \charakter{Beispielcharakter}. Er dreht sich um und fragt sich, wer eigentlich \charakter{Beispielcharakter} ist. Dann erinnert er sich. Das ist er selbst.
    \end{minipage}}
  \end{centering}

  Natürlich haben diese Charaktere auch verschiedene Eigenschaften und Fähigkeiten die in einem zusammengefassten Charakterbogen (sprich einer Tabelle)
  aufgelistet sind. Diese findet sich ebenfalls bei den Informationen zu den Charakteren am Ende des Dokuments.

  \begingroup
\renewcommand{\arraystretch}{1.3}
\begin{char-box}{Max Mustermann}


\centering
\begin{tabular*}{\textwidth}{@{\extracolsep{\fill}} p{0.19\textwidth}c|p{0.19\textwidth}c|p{0.19\textwidth}c}

\textbf{Handeln} & \raisebox{-1pt}{\includegraphics[width=0.08\textwidth, height=6mm]{01-img/strength.png}} &
\textbf{Wissen} & \raisebox{-1pt}{\includegraphics[width=0.08\textwidth, height=6mm]{01-img/knowledge.png}} &
\textbf{Soziales} & \raisebox{-1pt}{\includegraphics[width=0.08\textwidth, height=6mm]{01-img/social.png}} \\

% Handeln        & P   & Wissen   & P  & Soziales   & P
Spitze anstiften & 100 & Deutsch  & 80 & Langweilen & 1000 \\
Pflaster machen  & 20  & Englisch & 40 &            & \\
& & Excel & 30 & & \\

\hline
Talentwert & \textbf{12} & Talentwert & \textbf{15} & Talentwert & \textbf{100}

\end{tabular*}

\end{char-box}
\endgroup


  \item \textcolor{BrownBoxFrame}{\textbf{Ortsbeschreibungen}} \\
  Als Spielleiter seid ihr Augen und Ohren der Spieler. Damit das auch zur Zufriedenheit aller klappt sind in \textcolor{BrownBoxFrame}{braunen}
  Boxen Informationen über den Ort an dem sich die Spieler gerade befinden hinterlegt. Diese können einfach an die Spieler weitergegeben werden.

  \begin{place-box}{Dokument}

  Das Dokument, dass du/ihr gerade liest hat einen weißen Hintergrund auf dem in schwarzer Schrift Text steht. Es sieht sehr schön aus.

  \end{place-box}

  \item \textcolor{RedBoxFrame}{\textbf{Interaktionen}} \\
  Pen and Paper lebt von der \textcolor{RedBoxFrame}{Interaktion} der Spieler mit ihrer Umwelt. Wann immer eine Probe auf ein Talent
  möglich ist ist diese Information \textcolor{RedBoxFrame}{dunkelrot} markiert. Am Beispiel unten sieht man dabei auf welches Talent
  gewürfelt werden kann (Lesen), und ob dieser Wurf beeinflusst ist. Die Zahl in Klammern ($-10$) wird dabei immer zum Talentwert
  des Spielers dazuaddiert, ein negativer Wert bedeutet also, dass der Wurf erleichtert ist.

  \begin{probe-box}{Lesen}{-10}
  Bei erfolgreicher Probe geschieht, was hier steht. Oder wie genau der Versuch der Spieler in die Hose gehen kann. Kreativität ist natürlich auch gerne gesehen.
  \end{probe-box}
  \vspace{-5pt}


  \item \textcolor{BlueBoxFrame}{\textbf{Wörtliche Rede}} \\
  Alles, was \textcolor{RoyalBlue}{blau} geschrieben ist, soll wörtlich vorgetragen werden. Dabei handelt es sich in der Regel
  um wörtliche Rede in Gesprächen, in der sich unter Umständen wichtige Informationen verstecken können, oder eine
  stimmungsvolle Einleitung zu verschiedenen Abschnitten, beispielsweise ein Pro- oder Epilog.

  \begin{say-box}{Jemand}

  Franz rast in einem komplett verwahrlosten Taxi quer durch Bayern. Vogel Quax zwickt Johnnys Pferd Bim. Sylvia wagt quick den Jux bei Pforzheim.
  Lirum, larum Löffelstiel, alte Weiber essen viel, junge müssen fasten. S'Brot liegt im Kasten. Dunkel wars der Mond schien helle als ein Auto
  blitzeschnelle langsam um die Ecke fuhr. Drinnen saßen stehend Leute, schweigend ins Gespräch vertieft.

  \end{say-box}
  \vspace{-5pt}

\end{itemize}
