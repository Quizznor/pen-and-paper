\thispagestyle{fancy-info}
\section*{Generelle Informationen}

Auf den Seiten \pageref{info-start}-\pageref{info-end} befinden sich generelle Informationen über den Aufbau des Abenteuers. Beginnend mit Erklärungen zur
Textformatierung folgen Erläuterungen über die Rahmenhandlung und abenteuerspezifischen Regelwerken. Anschließend befindt sich auf den Seiten
\pageref{adv-start}-\pageref{adv-end} die Haupthandlungen des Abenteuers. Abschluss bilden Informationen über im Abenteuer auftretende Charaktere auf den
Seiten \pageref{char-start}-\pageref{char-end}.

\subsubsection*{Formatierung:}

\begin{itemize}
  \item \textit{Kursive Texte} \\
  Alles, was \textit{kursiv} geschrieben ist, soll wörtlich vorgetragen werden. Dabei handelt es sich in der Regel
  um wörtliche Rede in Gesprächen, in der sich unter Umständen wichtige Informationen verstecken können, oder eine
  stimmungsvolle Einleitung zu verschiedenen Abschnitten.

  \say{\lipsum[1-3]}

  \item \textbf{Beschreibungen} \\
  \textbf{Fett} gedruckte Passagen beinhalten Informationen und Beschreibungen über Orte an denen sich die
  Abenteurer aufhalten. Hierzu zählt auch die Einrichtung eines Raums oder die Beschaffenheit eines Objektes.
  In der Regel befinden sie sich am Beginn eines jeden Abschnittes oder Ortes, an dem sich die Gruppe aufhalten
  kann. Allerdings finden sich solche Informationen unter anderem auch zu Objekten die es gilt aufzusammeln.

  \item \textcolor{RoyalRed}{\textbf{Interaktionen}} \\
  Pen and Paper lebt von der \textcolor{RoyalRed}{Interaktion} der Spieler mit ihrer Umwelt. Wann immer eine Probe auf ein Talent
  möglich ist ist diese Information \textcolor{RoyalRed}{dunkelrot} markiert. Sie ermöglichen optionale Handlungsstränge oder
  Lösungsstrategien, die den Spielern entgehen können, wenn sie nicht die entsprechenden Aktionen durchführen oder
  sich für die entsprechende Option entscheiden.

  \item \textcolor{RoyalBlue}{\textbf{Referenzen}} \\
  Zur einfachen Navigation sind an relevanten Stellen \textcolor{RoyalBlue}{Referenzen} eingebaut, mithilfe derer sich einfach und
  schnell zwischen verschiedenen Textpassagen hin und her springen lässt. Solche Referenzen lassen sich (in der pdf
  Version) anklichen und sind \textcolor{RoyalBlue}{blau} markiert.

  \item \textcolor{violet}{\textbf{Charaktere}} \\
  Während der Handlung treffen die Spieler auf verschiedene \textcolor{violet}{Charaktere}. Referenzen zu Informationen über
  die jeweiligen Charaktere sind \textcolor{violet}{violett} hinterlegt. In den jeweiligen Abschnitten befinden sich
  Hintergrundinformationen wie Motivationen und Verhaltensweisen, etc. die der Person Leben einhauchen sollen.

\end{itemize}
