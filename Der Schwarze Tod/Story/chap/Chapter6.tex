Die Kirche St. Petri
Einleitung
\red{\textbf{Szene}}:

Die Kirche St. Petri steht ganz im Zeichen des neuen Hamburger Wohlstands. Noch immer wird gebaut, aber man sieht ihr schon jetzt an, dass sie eines der zukünftigen Wahrzeichen der Handelsmetropole sein wird.

Viele Menschen pilgern angesichts ihrer Machtlosigkeit gegenüber der Pest hierher, um Schutz und Trost zu suchen. Kranke und Verzweifelte säumen die Gänge, und der Geruch von Tod liegt in der Luft.


Ankunft an der Kirche
\red{\textbf{Szene}}:

Ortsbeschreibung:Sie stehen vor der gewaltigen Pforte einer der schönsten Kirchen Hamburgs. In neuem Glanz erstrahlt St. Petri und ist aufgrund der Lage auf einem Geestrücken bereits von Weitem sichtbar.

Ereignis: Sobald jemand an die Tür klopft werden sie von Helfern begrüßt und hereingelassen.

Interaktionen:

Probe auf Wahrnehmung, Aufmerksamkeit, Kirchenwissen oder ähnliches:

Sie tragen zwar Mönchsgewänder, scheinen aber keinem Orden zugehörig zu sein.
Zumindest nicht erkennbar.
Nach Pater Salus fragen:

Wird einer der Helfer nach Pater Salus gefragt, so wird ihnen berichtet, dass dieser aufgebrochen sei, um Gerüchten auf den Grund zu gehen.
Nun ist von schwarzer Magie die Rede, böse Mächte seien in der Stadt am Werk. Die Gruppe ist aber herzlich eingeladen zu bleiben, sofern sie gesund sind und dies beschwören.
\red{\textbf{Szene}}:

Ereignis: Dann lässt man sie in den Hauptraum der Kirche. Hier sitzen überall Menschen mit besorgten Gesichtern herum. Andere wiederum blicken voller Zuversicht und lauschen gebannt Gesprächen, Gesängen und Predigten, welche Mönche überall im Raum zum Besten geben. Außerdem gehen verdächtig viele Klingelbeutel herum. Es führen einige Türen aus dem Hauptraum. Alle sind verschlossen.

\textbf{Raumbeschreibung}: Das innere einer großen Kirche. Man sieht, dass die Bauarbeiten noch im Gange sind, doch im Hauptraum kann man sich bereits ohne größere Probleme aufhalten. Trotz Bau lässt sich bereits eine prunkvolle Kirche erahnen.

Interaktionen:

Probe auf Wahrnehmung,Aufmerksamkeit, Lauschen oder ähnliches:

Hört die Gruppe bei den Mönchen genauer hin, ist die Rede vom Jüngsten Gericht,
Sünde und der Strafe des Herren, die alle Sünder trifft.
Man könne sich aber von seinen Sünden befreien.
Nur etwas weltliches Gut müsse man opfern, um hier,
bei den Frommen, bleiben zu dürfen.
Gespräch mit Anwesenden:

Fragen sie die Leute, so berichten manche davon, dass sie mit Angehörigen hier waren.
Die wurden dann aber irgendwann nach hinten gebeten, wohl weil sie besonders fromm waren. Während sie so lauschen, bemerken sie, dass immer wieder einzelne Menschen aus dem Saal durch die Seitentüren gebracht werden.
Man will sie hier aber nicht durchlassen. Das seien „private Gemächer“ der Priester und Mönche. Die Gruppe kann versuchen, sich Zugang zu verschaffen.
Im Hinterzimmer
\red{\textbf{Szene}}:

Schaffen sie es irgendwie hier hinein, geht es erstmal durch ein paar Gänge.

Es stinkt nach Tod. Alle erhalten Pestilenz +1.

Nach einer Weile kommen sie in eine Art Gewölbe. Hier liegen unzählige Kranke, von gerade so infiziert bis zu bereits verstorben, hinter einem Gitter eingesperrt wie in einer Zelle.

Gehen sie näher an die Menschen heran, erhalten sie Pestilenz +3.

Sie alle sind überzeugt davon, dass sie das hier als Sünder verdient haben.

Am hinteren Ende des Raumes gibt es eine Tür. Sie scheint ins Freie zu führen. Davor am Gitter steht eine junge Frau, Gundel. Sie ist bereits von der Pest gezeichnet, winkt die Gruppe aber zu sich.

Gespräch mit Gundel: Sie bittet die Gruppe, ihr dabei zu helfen zu entkommen. Sie müsse zu diesem Arzt, diesem Didrich. Der wisse, was zu tun ist. Da ist sie sich sicher! Gundel arbeitet am Nikolaifleet. Dort kocht sie für die Männer. Nach und nach wurden sie alle krank.

Plötzlich tauchte der Arzt auf, Didrich von Sinnfeld. Er versprach ihr, wenn sie ihm ein paar Ratten aus der Küche fange, würde er sie fürstlich bezahlen. Sie lehnte ab. Das schien ihr doch sehr merkwürdig, und die Bezahlung sei nicht üppig gewesen. Als er aber sagte, dass er an etwas arbeite, um den Schwarzen Tod zu besiegen, lenkte sie ein. Sie gab ihm die Ratten. Doch dann wurden immer mehr Leute krank. Sie bekam es mit der Angst zu tun und kam hierher. Doch es war zu spät. Man erleichterte sie um ihr Geld und sperrte sie dann hier zum Sterben hinein.


Moral:

Lassen sie Gundel und die anderen raus und gefährden damit die Stadt? Oder gehen sie und überlassen die Kranken ihrem unumgänglichen Schicksal?

Vorm Ausgang
\red{\textbf{Szene}}:

Ereignis: Hier treffen sie noch einmal den Totensammler. Er schaut sie vorwurfsvoll an und zieht dann weiter klingelnd seine Runden.


Ende des relevanten St.Petri-Plots.
