%!TEX root = ../main.tex

\subsection{Der Weg nach Eeksdurf}
\label{nachxd}

Ihr macht euch also auf den Weg nach Eeksdurf. Schon bald drängen sich die Häuser weniger dicht und ihr verlasst Hamburg in Richtung Westen. Größtenteils verläuft der Weg aus gestampfter Erde gerade durch die Felder, an ein paar Stellen teilt sich der Weg und ihr folgt den Wegweisern nach Eeksdurf. So seid ihr also etwa eine halbe Stunde unterwegs als ihr plötzlich am Wegesrand etwas bemerkt.

\red{\textbf{Szene}}:

Im Straßengraben liegt umgekippt ein Karren. Wollen die Spieler die Szene genauer untersuchen entdecken sie, dass die Speichen der Räder gebrochen sind. Außerdem liegen im Straßengraben drei Leichen. Einer der Toten trägt ein Priestergewand und liegt mit durchgeschnittener Kehle dahingestreckt.

\red{\textbf{Probe auf Medizin o.ä.(erleichtert)}: Die anderen beiden Toten wurden durch Stichwunden in den Oberkörper getötet}

Es scheint sich um einen Raubüberfall zu handeln. Wertsachen finden sich hier keine, auch Teile ihrer Kleidung wurden den Leichen vom Leib gerissen.

\red{\textbf{Probe auf christliche Kultur o.ä. (erleichtert)}: Selbst das Kreuz, welches der Priester mit Sicherheit um den Hals trug, ist verschwunden.}

\red{\textbf{Information für den Spielleiter}: Das Kreuz kann die Gruppe an einem betrunkenen Piraten vor dem "Gelockten Hund" im Hafen finden.}

Ihr findet sonst keinen Hinweis darauf, wer die Verstorbenen sein könnte. Allerdings entdeckt ihr einige Pergamente, die in lateinischer Schrift verfasst sind.

\red{\textbf{Probe auf Latein o.ö}: Die Pergamente handeln von schwarzer Magie und anderem Volksglauben.}

Ein gutes Dutzend Fußspuren führt weiter in Richtung Eeksdurf, aber auch genügend in Richtung Hamburg. Schwer zu sagen, ob sie von den Tätern oder anderen Reisenden stammen. Aber in jedem Falle sind die Spuren am Ort der Tat nicht älter als eine Stunde. Mehr ist hier nicht zu finden.

\subsection{Ankunft in Eeksdurf}
\label{inxd}

\red{\textbf{Szene}}:

So macht ihr euch also weiter auf den Weg nach Eeksdurf. Nach einer weiteren halben Stunde seht ihr wie zwischen Bäumen die ersten Häuser des kleinen Dorfes hervortreten. Ihr folgt dem Weg und steht bald in dem kleinen Örtchen auf einer menschenleeren Straße. Durch den alltäglichen Klatsch und Tratsch wisst ihr, dass die meisten Einwohner hier einfache Handwerker und Bauern sind. Vom sonst so geschäftigen Treiben ist nun nichts mehr zu spüren, es ist alles verlassen. Türen und Fenster sind verrammelt.

Klopfen die Spieler an eine der Türen, dann antwortet man ihnen, wenn sie passend würfeln.

\begin{tcolorbox}
  Wurf: Wird ihnen auf das Klopfen geantwortet? \\
  0 bis 49: Ihnen wird geantwortet. \\
  50 bis 99: Ihr Klopfen bleibt unbeantwortet.\\
\end{tcolorbox}

Sollte einer der Dorfbewohner mit ihnen sprechen macht er einen verängstigten Eindruck. Wo die anderen alle seien wisse er nicht.

\red{\textbf{Probe auf Menschenkenntnis}: Das er nicht wisse wo alle seien ist eine Lüge.}

Fragen die Spieler nach einer rothaarigen Frau erzählt man ihnen von Ruth (\blue{\ref{Ruth}}), der Tochter der Kräutersammlerin Ottilde (\blue{\ref{Ottilde}}). Diese wohne gleich die Straße runter am Dorfesrand in einer kleinen verfallenen Hütte. Wenn man der Straße folge könne man es nicht verfehlen.

\red{\textbf{Probe auf Menschenkenntnis}: Auch die Wegweisung ist eine Lüge.}

Folgen die Spieler diesem Weg gelangen sie bald an den Dorfesrand, dort ist nirgends ein Haus zu sehen, das auf die Beschreibung des Dorfbewohners passt. Jedoch hören die Abenteurer ein Geräusch...

\begin{tcolorbox}
  Wurf: Was hört die Gruppe? \\
  1 bis 50: Sie hören aufgebrachte Stimmen. (gehe zu \blue{\ref{mob}})\\
  51 bis 100: Sie hören ein Knacken und Rauschen. (gehe zu \blue{\ref{feuer}})\\
\end{tcolorbox}

\subsubsection{Option 1 - Der wütende Mob}
\label{mob}

\red{\textbf{Szene}}:

Ihr folgt den GeräuschenUnd gelangt an eine kleine Hütte vor dem Waldrand. Die Hütte macht einen verfallenen Eindruck, scheint aber noch bewohnt zu sein. Vor der Haustüre stehen wütend grölende Menschen und schwingen Mistforken, während eine alte, merkwürdig gekleidete Dame vor der Hütte versucht sie zu besänftigen.

Aus den aufgebrachten Wortgefechten lässt sich schließen, dass es sich bei der Dame um die Kräuterfrau Ottilde handelt.

STOPPED HERE



Der Rädelsführer des Mobs ist ein Bauer namens Wolfgang. Dieser fordert vehement die Herausgabe von Ruth. Sie sei an allem Schuld! Schwarze Magie habe sie angewandt und nicht mal die Kirche wolle den Bewohnern mehr helfen!

Interaktionen:

Die Gruppe sollte versuchen den wütenden Mob zu beruhigen.

Gespräch mit Leuten aus dem Mob:

Die Leute im Mob berufen sich immer wieder auf vier Argumente:

*Ruth soll wiederholt dabei gesehen worden sein, wie sie nachts tote Körper in die große Scheune gebracht habe.
*Ruth habe den Brief an die Kirche, welchen sie dem Totensammler mitgeben sollte, nie übergeben.
*Ruth habe die Pest gebracht. Sie war in Hamburg gewesen, um Korn zu kaufen.
*Kaum war sie zurück, brach die Krankheit aus! Zu diesem Zeitpunkt sei noch nirgends sonst in der Gegend jemand erkrankt! (Das ist drei Wochen her)
Probe auf Beruhigen:

Erfolg: Schaffen Sie es den Mob zu beruhigen, dann zieht sich der Mob zurück und sie können mit Ottilde in Ruhe das Haus betreten.
Dort treffen sie auch auf Ruth, die verängstigt in
einer Ecke sitzt.

Misserfolg: Schaffen sie es nicht den Mob zu beruhigen, dann bricht dieser in das Haus ein.
Von Ruth fehlt allerdings jede Spur. Ottilde ist dankbar für ihre Hilfe und verrät der Gruppe,
dass Ruth sich in der großen Scheune verstecke.
Gespräch mit Ruth:

Ruth sagt, sie habe nichts gemacht und wisse nicht, warum die Bauern so aufgebracht seien.
Probe auf Menschenkenntnis:

Ruth lügt. Hakt die Gruppe nach, dann erzählt sie, was vorgefallen ist:
Sie habe direkt am Nikolaifleet Korn holen wollen. Da sei es am günstigsten, habe man ihr gesagt.
Als sie dort ankam, sei es bereits spät gewesen. Sie habe am Lagerhaus, das man ihr beschrieben hatte, angeklopft, allerdings ohne Erfolg.
Also habe sie durchs Fenster geschaut und einen Mann erblickt, der sich über etwas gebeugt habe. Er soll eine Art Vogel-Maske und einen weiten Mantel getragen haben.
Ruth sagt, sie habe Angst gehabt und habe davonlaufen wollen. Sie sei auf dem Schnee ausgerutscht und habe das Gleichgewicht verloren.
Dabei sei mit dem Kopf aufgeschlagen und im Lagerhaus wieder zu sich gekommen. Der Mann habe sich ihr dann als Didrich vorgestellt. Er forsche an der Pestilenz, habe er erklärt.
Nur darum sei er so gekleidet und habe sich an der Leiche zu schaffen gemacht. Ruth habe ihm dann erzählt, dass in Eeksdurf zwar niemand krank sei, aber dass der Totensammler regelmäßig mit den Leichen durch ihr Dorf fahre, um diese tief im Eekhult zu verscharren.
Daraufhin habe Didrich ihr einen Handel vorgeschlagen, und … naja … Ruth habe angenommen. „Wieso auch nicht. Die sind schließlich tot und ich arm. Eins davon kann man noch ändern.“
Die Leichen habe sie also von da an immer in ein kleines Lagerhaus im Armenviertel Hammerbrook gebracht. Dort habe auch ihr Geld gelegen …
Ruth sagt ihnen genau, welches Lagerhaus das gewesen sei. Die Gruppe kann anschließend zum Armenviertel aufbrechen.

Option 2 – Ein großes Feuer
\red{\textbf{Szene}}:

Ereignis: Im Näherkommen sehen sie, dass vor der Hütte ein gewaltiges Feuer brennt.

Niemand ist zu sehen. Im Feuer erkennen sie allerdings die Überreste zweier Körper. Beides Frauen.

Die Tür der Hütte steht offen.

Interaktionen:

In der Hütte umsehen:

Sehen sie sich in der Hütte um, dann finden sie ein Blatt Pergament. Auf das Blatt wurde eine Art Wegbeschreibung gekritzelt – nur sehr grob. Der Weg führt zu einem Gebäude im Armenviertel Hammerbrook.
Genaueres Durchsuchen:

Suchen sie ausgiebiger, dann finden sie Geld. Etwas mehr Geld, als jemand, der hier lebt, besitzen sollte.
Verlassen der Hütte:

Beim Verlassen der Hütte trifft die Gruppe auf Wolfgang und Hermann. Diese werfen ihnen umgehend vor, mit der bösen Zauberin im Bunde zu sein.
Probe auf Beruhigen:

Erfolg: Beruhigen sie die beiden Bauern, erzählen diese der Gruppe, warum sie Ottildes Haus in Brand gesetzt haben.

Sie bringen folgende Argumente:
*Ruth soll wiederholt dabei gesehen worden sein, wie sie nachts tote Körper in die große Scheune gebracht habe.
*Ruth habe den Brief an die Kirche, welchen sie dem Totensammler mitgeben sollte, nie übergeben.
*Ruth habe die Pest gebracht. Sie war in Hamburg gewesen, um Korn zu kaufen.
*Kaum war sie zurück, brach die Krankheit aus! Zu diesem Zeitpunkt sei noch nirgends sonst in der Gegend jemand erkrankt! (Das ist drei Wochen her)

Misserfolg: Schaffen sie es nicht Wolfgang und Hermann zu beruhigen, so kommt es zum Kampf.
Wolfgang ist mit einer Mistforke bewaffnet. Außerdem sehen beide Bauern bereits von der Krankheit gezeichnet aus.

Kampf:

Wolfgang:

Leben: 70
Waffe: Mistgabel (70)
Schaden: 40
Parieren 30

Hermann:

Leben: 70
Waffe: Fäuste(70)
Schaden: 15
Parieren 5

Moral:

Die beiden Bauern brüsten sich mit der Tat. Wie reagiert unsere Gruppe darauf?


Ende des relevanten Eeksdurf-Plots.
