%!TEX root = ../main.tex

\subsection{Der Weg nach Eeksdurf}


\red{\textbf{Szene}}:

Die Gruppe macht sich auf den Weg in Richtung des Westens der Stadt. Dort liegt das kleine Dorf Eeksdurf. Bisher wissen sie nicht viel über das Örtchen, aber das soll sich bald ändern.

Nach einer kurzen Strecke werden die Häuser am Straßenrand immer weniger.

Sie nehmen ein bis zwei Abzweigungen immer den Wegweisern nach.

Ereignis: Als sie etwa eine Stunde unterwegs sind, hält der Kutscher abrupt. Sie sehen im Straßengraben einen Karren liegen.

Daneben: Leichen.


Der Überfall
\red{\textbf{Szene}}:

Als sie näher kommen, können sie das ganze Ausmaß des Unheils sehen. Am Wegesrand liegt ein Wagen mit gebrochenem Rad. Daneben entdecken sie drei Leichen, eine davon offensichtlich die eines Priesters, sogar verhältnismäßig gut gekleidet. Leider scheint auch ihn ein unschönes Schicksal ereilt zu haben.

Interaktionen:

Umsehen:

Allen Opfern wurde der Schädel eingeschlagen und Teile ihrer Kleidung gestohlen.
Alle Wertsachen fehlen, und auf dem Karren finden sich Reste von Lebensmitteln. Auch von diesen fehlt aber jede Spur.

Zusatz (erfordert Wissen zu christlicher Kultur): Selbst das Kreuz, welches der Priester mit Sicherheit um den Hals trug, ist verschwunden.

Information für den Spielleiter: Das Kreuz kann die Gruppe an einem betrunkenen Piraten vor dem "Gelockten Hund" im im Hafen finden.

Sie finden keinen Hinweis darauf, wer der verstorbene Priester sein könnte. Allerdings entdecken sie einige Pergamente, die in lateinischer Schrift verfasst sind.

Zusatz (erfordert Lateinkenntnisse): Die Pergamente handeln von schwarzer Magie und anderem Volksglauben.

Ein gutes Dutzend Fußspuren führt weiter in Richtung Eeksdurf, aber auch genügend in Richtung Hamburg. Schwer zu sagen, ob sie von den Tätern oder anderen Reisenden stammen. Aber in jedem Falle sind die Spuren am Ort der Tat nicht älter als eine Stunde. Mehr ist hier nicht zu finden.

Ankunft in Eeksdurf
\red{\textbf{Szene}}:

Eeksdurf liegt vor den westlichen Toren der Stadt, am Rande des Eekshult.

Die Bewohner des Ortes gehen den üblichen Handwerksberufen und der Landwirtschaft nach. Ein beschaulicher Ort. Binnen zweier Wochen starben oder erkrankten bereits über die Hälfte der Bewohner des Dörfchens.

Das kann kein Zufall sein, vermuten einige im Dorf. Jemand sei mit dem Teufel im Bunde, wird gemutmaßt. Eine Hexenjagd entbrennt, vor der niemand sicher zu sein scheint.

Ortsbeschreibung: Als die Gruppe das Dorf betritt, ist alles verlassen. Alle Türen sind verrammelt, und es laufen keine Menschen auf den Straßen herum.

Klopfen sie an eine der Türen, dann antwortet man ihnen, wenn sie passend würfeln.

Interaktionen:

An Türen klopfen:

0 bis 49: Ihnen wird geantwortet.
50 bis 99: Ihr Klopfen bleibt unbeantwortet.

Gespräch mit Dorfbewohnern:

Niemand will sie hereinlassen.
Angeblich weiß niemand, wo all die Männer des Ortes sind.

Fragen sie nach einer rothaarigen Frau, erzählt man ihnen von Ruth, der Tochter der Kräutersammlerin Ottilde.
Diese wohne gleich der Hauptstraße nach am Rand des Eekshult.
Probe auf Menschenkenntnis:

Das die Bewohner nicht wissen, wo genau die Männer des Ortes gerade sind, ist eine Lüge.
Bei Ottildes Haus
\red{\textbf{Szene}}:

Ortsbeschreibung: Am Waldrand steht eine kleine Hütte. Sie ist nahezu verfallen, aber offensichtlich noch bewohnt.

Ereignis: Als die Gruppe sich der Hütte nähert, erkennt sie...

Wurf: Was sieht die Gruppe vor Ottildes Haus?
Einen wütenden Mob (0 bis 49)
Ein großes Feuer (50 bis 99)

Option 1 – Wütender Mob
\red{\textbf{Szene}}:

Ereignis: Vor der Hütte stehen wütend grölende Menschen und schwingen Mistforken, während eine alte, merkwürdig gekleidete Dame vor der Hütte versucht sie zu besänftigen.

Aus dem gegröle lässt sich schließen, dass es sich bei der Dame um die Kräuterfrau Ottilde handelt.

Der Rädelsführer des Mobs ist ein Bauer namens Wolfgang. Dieser fordert vehement die Herausgabe von Ruth. Sie sei an allem Schuld! Schwarze Magie habe sie angewandt und nicht mal die Kirche wolle den Bewohnern mehr helfen!

Interaktionen:

Die Gruppe sollte versuchen den wütenden Mob zu beruhigen.

Gespräch mit Leuten aus dem Mob:

Die Leute im Mob berufen sich immer wieder auf vier Argumente:

*Ruth soll wiederholt dabei gesehen worden sein, wie sie nachts tote Körper in die große Scheune gebracht habe.
*Ruth habe den Brief an die Kirche, welchen sie dem Totensammler mitgeben sollte, nie übergeben.
*Ruth habe die Pest gebracht. Sie war in Hamburg gewesen, um Korn zu kaufen.
*Kaum war sie zurück, brach die Krankheit aus! Zu diesem Zeitpunkt sei noch nirgends sonst in der Gegend jemand erkrankt! (Das ist drei Wochen her)
Probe auf Beruhigen:

Erfolg: Schaffen Sie es den Mob zu beruhigen, dann zieht sich der Mob zurück und sie können mit Ottilde in Ruhe das Haus betreten.
Dort treffen sie auch auf Ruth, die verängstigt in
einer Ecke sitzt.

Misserfolg: Schaffen sie es nicht den Mob zu beruhigen, dann bricht dieser in das Haus ein.
Von Ruth fehlt allerdings jede Spur. Ottilde ist dankbar für ihre Hilfe und verrät der Gruppe,
dass Ruth sich in der großen Scheune verstecke.
Gespräch mit Ruth:

Ruth sagt, sie habe nichts gemacht und wisse nicht, warum die Bauern so aufgebracht seien.
Probe auf Menschenkenntnis:

Ruth lügt. Hakt die Gruppe nach, dann erzählt sie, was vorgefallen ist:
Sie habe direkt am Nikolaifleet Korn holen wollen. Da sei es am günstigsten, habe man ihr gesagt.
Als sie dort ankam, sei es bereits spät gewesen. Sie habe am Lagerhaus, das man ihr beschrieben hatte, angeklopft, allerdings ohne Erfolg.
Also habe sie durchs Fenster geschaut und einen Mann erblickt, der sich über etwas gebeugt habe. Er soll eine Art Vogel-Maske und einen weiten Mantel getragen haben.
Ruth sagt, sie habe Angst gehabt und habe davonlaufen wollen. Sie sei auf dem Schnee ausgerutscht und habe das Gleichgewicht verloren.
Dabei sei mit dem Kopf aufgeschlagen und im Lagerhaus wieder zu sich gekommen. Der Mann habe sich ihr dann als Didrich vorgestellt. Er forsche an der Pestilenz, habe er erklärt.
Nur darum sei er so gekleidet und habe sich an der Leiche zu schaffen gemacht. Ruth habe ihm dann erzählt, dass in Eeksdurf zwar niemand krank sei, aber dass der Totensammler regelmäßig mit den Leichen durch ihr Dorf fahre, um diese tief im Eekhult zu verscharren.
Daraufhin habe Didrich ihr einen Handel vorgeschlagen, und … naja … Ruth habe angenommen. „Wieso auch nicht. Die sind schließlich tot und ich arm. Eins davon kann man noch ändern.“
Die Leichen habe sie also von da an immer in ein kleines Lagerhaus im Armenviertel Hammerbrook gebracht. Dort habe auch ihr Geld gelegen …
Ruth sagt ihnen genau, welches Lagerhaus das gewesen sei. Die Gruppe kann anschließend zum Armenviertel aufbrechen.

Option 2 – Ein großes Feuer
\red{\textbf{Szene}}:

Ereignis: Im Näherkommen sehen sie, dass vor der Hütte ein gewaltiges Feuer brennt.

Niemand ist zu sehen. Im Feuer erkennen sie allerdings die Überreste zweier Körper. Beides Frauen.

Die Tür der Hütte steht offen.

Interaktionen:

In der Hütte umsehen:

Sehen sie sich in der Hütte um, dann finden sie ein Blatt Pergament. Auf das Blatt wurde eine Art Wegbeschreibung gekritzelt – nur sehr grob. Der Weg führt zu einem Gebäude im Armenviertel Hammerbrook.
Genaueres Durchsuchen:

Suchen sie ausgiebiger, dann finden sie Geld. Etwas mehr Geld, als jemand, der hier lebt, besitzen sollte.
Verlassen der Hütte:

Beim Verlassen der Hütte trifft die Gruppe auf Wolfgang und Hermann. Diese werfen ihnen umgehend vor, mit der bösen Zauberin im Bunde zu sein.
Probe auf Beruhigen:

Erfolg: Beruhigen sie die beiden Bauern, erzählen diese der Gruppe, warum sie Ottildes Haus in Brand gesetzt haben.

Sie bringen folgende Argumente:
*Ruth soll wiederholt dabei gesehen worden sein, wie sie nachts tote Körper in die große Scheune gebracht habe.
*Ruth habe den Brief an die Kirche, welchen sie dem Totensammler mitgeben sollte, nie übergeben.
*Ruth habe die Pest gebracht. Sie war in Hamburg gewesen, um Korn zu kaufen.
*Kaum war sie zurück, brach die Krankheit aus! Zu diesem Zeitpunkt sei noch nirgends sonst in der Gegend jemand erkrankt! (Das ist drei Wochen her)

Misserfolg: Schaffen sie es nicht Wolfgang und Hermann zu beruhigen, so kommt es zum Kampf.
Wolfgang ist mit einer Mistforke bewaffnet. Außerdem sehen beide Bauern bereits von der Krankheit gezeichnet aus.

Kampf:

Wolfgang:

Leben: 70
Waffe: Mistgabel (70)
Schaden: 40
Parieren 30

Hermann:

Leben: 70
Waffe: Fäuste(70)
Schaden: 15
Parieren 5

Moral:

Die beiden Bauern brüsten sich mit der Tat. Wie reagiert unsere Gruppe darauf?


Ende des relevanten Eeksdurf-Plots.
