






Die Kirche St. Petri
Einleitung
\red{\textbf{Szene}}:

Die Kirche St. Petri steht ganz im Zeichen des neuen Hamburger Wohlstands. Noch immer wird gebaut, aber man sieht ihr schon jetzt an, dass sie eines der zukünftigen Wahrzeichen der Handelsmetropole sein wird.

Viele Menschen pilgern angesichts ihrer Machtlosigkeit gegenüber der Pest hierher, um Schutz und Trost zu suchen. Kranke und Verzweifelte säumen die Gänge, und der Geruch von Tod liegt in der Luft.


Ankunft an der Kirche
\red{\textbf{Szene}}:

Ortsbeschreibung:Sie stehen vor der gewaltigen Pforte einer der schönsten Kirchen Hamburgs. In neuem Glanz erstrahlt St. Petri und ist aufgrund der Lage auf einem Geestrücken bereits von Weitem sichtbar.

Ereignis: Sobald jemand an die Tür klopft werden sie von Helfern begrüßt und hereingelassen.

Interaktionen:

Probe auf Wahrnehmung, Aufmerksamkeit, Kirchenwissen oder ähnliches:

Sie tragen zwar Mönchsgewänder, scheinen aber keinem Orden zugehörig zu sein.
Zumindest nicht erkennbar.
Nach Pater Salus fragen:

Wird einer der Helfer nach Pater Salus gefragt, so wird ihnen berichtet, dass dieser aufgebrochen sei, um Gerüchten auf den Grund zu gehen.
Nun ist von schwarzer Magie die Rede, böse Mächte seien in der Stadt am Werk. Die Gruppe ist aber herzlich eingeladen zu bleiben, sofern sie gesund sind und dies beschwören.
\red{\textbf{Szene}}:

Ereignis: Dann lässt man sie in den Hauptraum der Kirche. Hier sitzen überall Menschen mit besorgten Gesichtern herum. Andere wiederum blicken voller Zuversicht und lauschen gebannt Gesprächen, Gesängen und Predigten, welche Mönche überall im Raum zum Besten geben. Außerdem gehen verdächtig viele Klingelbeutel herum. Es führen einige Türen aus dem Hauptraum. Alle sind verschlossen.

\textbf{Raumbeschreibung}: Das innere einer großen Kirche. Man sieht, dass die Bauarbeiten noch im Gange sind, doch im Hauptraum kann man sich bereits ohne größere Probleme aufhalten. Trotz Bau lässt sich bereits eine prunkvolle Kirche erahnen.

Interaktionen:

Probe auf Wahrnehmung,Aufmerksamkeit, Lauschen oder ähnliches:

Hört die Gruppe bei den Mönchen genauer hin, ist die Rede vom Jüngsten Gericht,
Sünde und der Strafe des Herren, die alle Sünder trifft.
Man könne sich aber von seinen Sünden befreien.
Nur etwas weltliches Gut müsse man opfern, um hier,
bei den Frommen, bleiben zu dürfen.
Gespräch mit Anwesenden:

Fragen sie die Leute, so berichten manche davon, dass sie mit Angehörigen hier waren.
Die wurden dann aber irgendwann nach hinten gebeten, wohl weil sie besonders fromm waren. Während sie so lauschen, bemerken sie, dass immer wieder einzelne Menschen aus dem Saal durch die Seitentüren gebracht werden.
Man will sie hier aber nicht durchlassen. Das seien „private Gemächer“ der Priester und Mönche. Die Gruppe kann versuchen, sich Zugang zu verschaffen.
Im Hinterzimmer
\red{\textbf{Szene}}:

Schaffen sie es irgendwie hier hinein, geht es erstmal durch ein paar Gänge.

Es stinkt nach Tod. Alle erhalten Pestilenz +1.

Nach einer Weile kommen sie in eine Art Gewölbe. Hier liegen unzählige Kranke, von gerade so infiziert bis zu bereits verstorben, hinter einem Gitter eingesperrt wie in einer Zelle.

Gehen sie näher an die Menschen heran, erhalten sie Pestilenz +3.

Sie alle sind überzeugt davon, dass sie das hier als Sünder verdient haben.

Am hinteren Ende des Raumes gibt es eine Tür. Sie scheint ins Freie zu führen. Davor am Gitter steht eine junge Frau, Gundel. Sie ist bereits von der Pest gezeichnet, winkt die Gruppe aber zu sich.

Gespräch mit Gundel: Sie bittet die Gruppe, ihr dabei zu helfen zu entkommen. Sie müsse zu diesem Arzt, diesem Didrich. Der wisse, was zu tun ist. Da ist sie sich sicher! Gundel arbeitet am Nikolaifleet. Dort kocht sie für die Männer. Nach und nach wurden sie alle krank.

Plötzlich tauchte der Arzt auf, Didrich von Sinnfeld. Er versprach ihr, wenn sie ihm ein paar Ratten aus der Küche fange, würde er sie fürstlich bezahlen. Sie lehnte ab. Das schien ihr doch sehr merkwürdig, und die Bezahlung sei nicht üppig gewesen. Als er aber sagte, dass er an etwas arbeite, um den Schwarzen Tod zu besiegen, lenkte sie ein. Sie gab ihm die Ratten. Doch dann wurden immer mehr Leute krank. Sie bekam es mit der Angst zu tun und kam hierher. Doch es war zu spät. Man erleichterte sie um ihr Geld und sperrte sie dann hier zum Sterben hinein.


Moral:

Lassen sie Gundel und die anderen raus und gefährden damit die Stadt? Oder gehen sie und überlassen die Kranken ihrem unumgänglichen Schicksal?

Vorm Ausgang
\red{\textbf{Szene}}:

Ereignis: Hier treffen sie noch einmal den Totensammler. Er schaut sie vorwurfsvoll an und zieht dann weiter klingelnd seine Runden.


Ende des relevanten St.Petri-Plots.

Der Nikolaifleet
Einleitung
Erst seit wenigen Jahren wächst am Nikolaifleet ein gewaltiges Lagerzentrum für die Waren aus aller Welt.

Der neue Reichtum trägt hier Früchte, und in direkter Nähe zu Tee, Gewürzen und Tulpen siedeln sich die betuchteren Bürger Hamburgs in prächtigen Villen an. Alles scheint makellos und vom Chaos der übrigen Stadt unberührt. Fast schon zu makellos.

Ankunft am Fleet
\red{\textbf{Szene}}:

Ortsbeschreibung: Der Nikolaifleet ist ordentlich und sauber. Weißer Schnee fällt auf frisch gepflasterte Straßen. Niemand ist zu sehen, obwohl hier reges Treiben herrschen sollte. Die Gruppe findet recht unkompliziert das Warenlager.


Lagerhaus Grote Buur
\red{\textbf{Szene}}:

„Grote Buur“ steht in eisernen Lettern überm Eingang. Zur Überraschung aller ist die Tür nicht verschlossen.

\textbf{Raumbeschreibung}: Ein typisches Lagerhaus. Auf der Rückseite sind die Tore gut sichtbar, die zum Löschen der Boote genutzt werden, welche die Waren an den Fleet bringen. Das Lagerhaus hat gleich mehrere Etagen, und es gibt diverse Kräne und Seilzüge, um Waren auf den Ebenen zu transportieren. Bei näherer Betrachtung erkennt man, dass sich hier einige Ratten niedergelassen haben.

Ereignis: Niemand scheint hier zu sein. Doch dann hören sie eine Art Jammern aus den Untergeschossen. Sie finden eine Luke. Darunter sitzt ein gefesselter Mann. Er hat überall am Körper Wunden, die nicht nur von der Pest verursacht wurden.

Interaktionen:

Begegnung mit dem Mann:

Dem Mann fällt das Sprechen sehr schwer, aber er kann noch „Sinnfeld“ und „gegenüber“ stammeln.

Die Gruppe kann ihn retten, erhält dann aber Pestilenz +3.

Andernfalls stirbt er vor ihren Augen.
Körper untersuchen: Die Wunden an seinem Körper sind unter anderem Rattenbisse. Außerdem ist er übersäht mit Flöhen.

Das Handelsregister
\red{\textbf{Szene}}:

Das Handelsregister ist der Ort, an dem alles Wissen der Stadt um jeden Geschäftsmann und seine Machenschaften zusammenkommt.

Ortsbeschreibung: Das Gebäude ist modern, fein verziert und kann sich sehen lassen.

Die Gruppe kann hierhin gehen, um Auskunft über den Eigentümer des „Grote Buur“ und Didrich von Sinnfeld einzuholen.

Beim Eintreten sehen sie eine Beamtin, die offensichtlich das Sagen hat, und ein paar um sie herum wuselnde Mitarbeiter.

Interaktionen:

Auf Anfrage kann die Gruppe hier ein Dokument erhalten, das belegt, dass Didrich von Sinnfeld der Eigentümer des „Grote Buur“ ist und gegenüber wohnt.

Das Haus gegenüber
\red{\textbf{Szene}}:

Gleich gegenüber des „Grote Buur“ steht ein ansehnliches Bürgerhaus.

Ortsbeschreibung: Es ist gepflegt, sauber und von außen in hervorragendem Zustand.

Ereignis: Die Gruppe kann an die Tür und anklopfen. Als sie klopfen öffnet ihnen...

Wurf: Wer öffnet?
Ein Mann (0 bis 49)
Eine Frau (50 bis 99)
Option 1 – Ein Mann
\red{\textbf{Szene}}:

Ereignis: Die Tür öffnet sich einen Spalt, und eine Stimme ertönt.

Man sieht aber niemanden. „Wer da?“

Die Person reagiert erschrocken auf die Stimmen unserer Charaktere und stürmt nach hinten ins Haus davon. Verfolgen sie den Mann, sehen sie diesen um eine Ecke rennen und hören dann ein gewaltiges Klirren und Krachen.

Sie kommen in einen düsteren Raum voller medizinischer Instrumente und anderer abstruser Gerätschaften. Überall im Raum sind Notizen und Tagebücher verteilt. Auf dem Tisch in der Mitte des Raumes liegt ein kleiner Junge. Er ist tot und gezeichnet von der Pest. Sein Brustkorb und sein Kopf sind geöffnet. Seine Organe liegen auf Tellern neben seinem Körper. Sein Gesicht ist in Panik und Schmerz erstarrt. Vor dem Tisch auf dem Boden liegt der Körper eines bewusstlosen Mannes, der eben noch vor ihnen geflohen war. Er hat das Blut des Jungen an den Händen.

Interaktionen:

Dursuchen oder Lesen der Notizen und Tagebücher:

Die Notizen im Raum verraten, dass der Mann an etwas rund um die Pest geforscht hat und das Ganze etwas mit Ratten zu tun gehabt haben dürfte. Jedenfalls tauchen diese überall auf. Auch Beschreibungen zum Bau von Geräten (Masken und Kleidung), die eine Ansteckung verhindern, gibt es.

Probe auf Medizin:

Außerdem finden sie unzählige Zeichnungen sezierter Körper sowie Berichte von Obduktionen lebender Patienten und Krankheitsverläufen bei Gefangenen. Harter Toback. Aber auch wichtige Erkenntnisse. Diesen Unterlagen zufolge hat die Krankheit ihren Ursprung in der Rattenpopulation, und hält man sich von Kranken und Ratten fern, so kann man eine Ansteckung verhindern.
Moral:

Was macht die Gruppe mit Didrich und den Informationen?

Option 2 – Eine Frau
\red{\textbf{Szene}}:

Ereignis: Klopfen sie an, öffnet die Magd Traudel.

Gespräch mit Traudel: Im Haus wohne ein Herr von Sinnfeld, Didrich von Sinnfeld. Ja, er sei zu Hause. Die Gruppe möge eintreten!

Die Magd führt sie in den Salon. Sie werden gebeten, zu warten.

Ereignis: Nach einer Weile hören sie einen Schrei. Dann rennt das Hausmädchen an ihnen vorbei zur Tür hinaus.

Geht die Gruppe dem nach, finden sie Didrich in seinem Arbeitszimmer. Dieser beugt sich gerade über den leblosen Körper eines kleinen Jungen und ist im Begriff, ihn zu sezieren.

Gespräch mit Didrich:

„Ein Kleingeist, die gute Traudel. Viel zu leicht zu erschrecken. Und nun?! Habe ich nichts als Ärger am Hals.

Denn die werten Herrschaften, so nehme ich an, stehen meinem Treiben hier ebenso wenig wohlgesonnen gegenüber wie andere Vertreter ihrer Stände.

Sehe ich das richtig?”

Didrich schildert ihnen, dass er an der Krankheit forsche. Er sei kurz vor einem Durchbruch. Es wisse nun mit Sicherheit, dass es etwas mit Ratten zu tun habe. Nur der genaue Ablauf der Ansteckung sei ihm noch schleierhaft. Aber ein Fehlen von Ratten in einer Stadt ginge unmittelbar mit einem Abhandensein der Seuche einher, obwohl durch Kontakt zu Kranken ebenfalls eine Seuche zustande kommen könne. Seine Forschungen seien nicht immer ganz lupenrein gewesen. Er brauchte tote Körper, später lebendige, das gebe er zu. Zum Glück gäbe es genug Kranke direkt hier am Fleet. Gleich gegenüber fielen die Arbeiter reihenweise um. Aber dann blieben sie zu Haus. Ein Besuch auf Hammerbrook sei also unumgänglich geworden. Dann noch einer. Und noch einer. Jedenfalls könne man solche Forschungen in einer kleingeistigen Stadt wie Hamburg nicht ohne Weiteres betreiben. Daher die Heimlichtuerei. Was nun? Soll er seine Forschungen fortsetzen?

Ihm ist durchaus bewusst, weshalb die Gruppe hier ist.


Moral:

Was tun sie mit Didrich und seinen Forschungen?

Ende des relevanten Nikolaifleet-Plots.

Der Abschlussbericht
\red{\textbf{Szene}}:

Gespräch mit dem Rat: Am letzten Tage, oder wenn sie sich eher im Stande sieht, muss die Gruppe vor den Hamburger Rat treten. Sie werden aufgefordert, ihre Ergebnisse zu präsentieren. Dabei stellt der Rat durchaus Fragen und ist kritisch. Außerdem wurden dem Rat Informationen über das Handeln der Gruppe zugetragen, gerade was die moralischen Fragen betrifft:

Wie ist die Gruppe mit Gorichs Familie umgegangen?
Haben sie die Kranken in St. Petri befreit?
Wie sind sie mit den Bauern Wolfgang und Hermann umgegangen?
Was machten sie mit Didrich?
Haben sie Sigruns Kindern geholfen?
Hat die Gruppe alle Fragen beantwortet, will sich der Rat beraten.

Was hält der Spielleiter vom Ergebnis der Recherche, und was soll nun geschehen?

1. Die Gruppe war sorgfältig. Setzen wir ihren Vorschlag um!
2. Die Gruppe war nicht überzeugend genug. Warten wir ab!
3. Die Gruppe war nachlässig. Sie sollen bestraft werden!
Die Entscheidung des Spielleiters beendet das Abenteuer.
