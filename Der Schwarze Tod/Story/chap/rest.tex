
Im Sitz des Beirats
\red{\textbf{Szene}}:

Als ihr am Sitz des Beirats ankommt, erkennt ihr, dass es sich dabei um ein durchaus anschauliches Gebäude handelt. Im Inneren stehen allerhand Tische und Stühle herum. Ausserdem gibt es Schlafmöglichkeiten und so ziemlich alles, was man zum Leben so gebrauchen kann. Selbst ausgewählte Speisen stehen bereits zum Verzehr bereit.

\textbf{Raumbeschreibung}: Der Sitz des Beirats ist eine kleine Kapelle, die zum Anlass des Aufenthalts der Gruppe frisch mit Lebensmitteln, Decken und Lampen aufgestockt wurde.

Ereignis: Die Gruppe kann sich erst mal unterhalten. Ihr Gespräch wird allerdings von einem Klopfen unterbrochen ...

Wurf: Wer oder was unterbricht das Gespräch unserer Gruppe?
Das Militär (1 bis 33)
Der Tod (34 bis 66)
Ein Kind (67 bis 99)
Bei 100: würfle erneut.
Option 1 – Der militärische Besuch
\red{\textbf{Szene}}:

Ereignis: Während die Gruppe noch debattiert, wo man ansetzen solle, klopft es plötzlich an der großen Holztür.

General zur Brügge steht davor und bittet um Einlass.

Gespräch mit Brügge: Dieser berichtet ihnen in vehementem Ton, dass die ganze Krise ein Werk der Dithmarscher sei. Diese hätten sich jahrelang an Hamburgs Handelsschiffen gütlich getan. Nun, da es einen Vertrag gibt, der das verhindert, versuchen einige von ihnen die Stadt zu schwächen, um davon zu profitieren oder sie gar ganz an sich zu reißen. Die Dithmarscher operieren von ihrem Versteck aus, das sich in einer Hafenkaschemme namens „Beim Gelockten Hund“ befinden soll. Ein gewisser Gorich leite das Ganze. Dort sollten sie mit ihren Recherchen beginnen.

Interaktionen:

Probe auf Menschenkenntnis:

General zur Brügge sagt schon die Wahrheit, aber seine Perspektive könnte durchaus verzerrt beziehungsweise einseitig sein.
Die Charaktere können sich aber sicher sein, dass er nicht lügt, zumindest seiner Auffassung nach nicht.

Ab hier können die Spieler frei entscheiden, wohin sie gehen wollen! Nach Ablauf der Frist von vier Tagen müssen sie beim Hamburger Rat vorsprechen. Bis dahin müssen sie sich auf eine Handlungsempfehlung festgelegt haben!


Option 2 – Der Tod
\red{\textbf{Szene}}:

Ereignis: Während sich die Gruppe noch unterhält, hören die Charaktere plötzlich ein lautes Klopfen an der Tür.

Davor steht der Totensammler Hanno. Er fragt, ob es Tote gäbe, die abzuholen seien, und ob im Haus bereits die Pest wüte.

Gespräch mit Hanno: Im Armenviertel sei es am Schlimmsten. Die Leichen könne er kaum mehr entsorgen. Man müsse kreativ werden.

Gegen Bestechung verrät er, dass er jemandem Leichen verkaufe. Dazu müsse er sie allerdings recht weit fortbringen, nämlich in einen kleinen Ort namens Eeksdurf vor den Toren der Stadt. Dort hinterlege er die leblosen Körper in einem Lagerhaus, wo bereits seine Bezahlung auf ihn warte. Die Absprache habe er dereinst mit einer jungen rothaarigen Frau getroffen.

Sie habe ihn angesprochen, nachdem sie ihn beim Abholen von Leichen im Dorf sah...


Ab hier können die Spieler frei entscheiden, wohin sie gehen wollen! Nach Ablauf der Frist von vier Tagen müssen sie beim Hamburger Rat vorsprechen. Bis dahin müssen sie sich auf eine Handlungsempfehlung festgelegt haben!


Option 3 – Ein Kind
\red{\textbf{Szene}}:

Ereignis: Es klopft plötzlich an der Tür und davor steht ein Kind zusammen mit seiner stark vermummten Mutter.

Gespräch mit der Mutter und dem Kind: Sie kämen aus dem Armenviertel Hammerbrook und seien auf der Suche nach der St. Petri-Kirche. Sie soll ein Zufluchtsort für gesunde und sündenfreie Menschen sein. Niemand werde dort krank! Für sie sei es zu spät, hustet die Frau, aber ihr kleines Kind, das sei noch zu retten. Sie wissen das alles von einem Mann, der im Armenviertel nach den Leuten sehe. Er werde nicht krank, egal was er tue ... Er habe sie losgeschickt. Sie wüssten gern den Weg.

Die Gruppe kann eine Beschreibung von Didrich von Sinnfeld erhalten. Außerdem geben ihnen die beiden auf Nachfrage den Tipp, einmal beim Lumpensammler im Armenviertel vorbeizuschauen.


Ab hier können die Spieler frei entscheiden, wohin sie gehen wollen! Nach Ablauf der Frist von vier Tagen müssen sie beim Hamburger Rat vorsprechen. Bis dahin müssen sie sich auf eine Handlungsempfehlung festgelegt haben!


Hafen
Einleitung des Hafens
Der Hafen steht seit Hamburgs Beitritt zur Hanse vor etwa 30 Jahren im Mittelpunkt der städtischen Wirtschaft. Hier wächst der neue Wohlstand heran. Unzählige Waren und Handelsgüter werden Tag ein Tag aus umgeschlagen, und mit ihnen kommen immer mehr Fremde in die Stadt. Eine ganz eigene Welt entsteht. Mit eigenen Regeln, die man erst lernen muss ...

Die Ankunft
\red{\textbf{Szene}}:

Die Gruppe steht nun inmitten des pulsierenden Hafens.

Ortsbeschreibung: Allerhand Gesindel und unzählige Hafenarbeiter treiben sich herum. Prostituierte bieten ihre Dienste an, und Kinder betteln um ein wenig Brot. Hin und wieder sieht man Menschen, die sich vermummen oder gar ihr ganzes Gesicht hinter Tüchern verbergen.

Interaktionen:

Die Gruppe kann sich nun zum „Gelockten Hund“ aufmachen oder sich zunächst ein wenig umsehen.

Hier treffen sie auf allerhand Charaktere, die ihnen etwas zum Hafen erzählen können.

Gespräch mit Personen am Hafen:

Es fallen immer mehr Arbeiter aus, die mit der Lagerung von Lebensmitteln und anderen verderblichen Waren zu tun haben.
Gasthaus zum "Gelockten Hund"
Vor dem "Gelockten Hund"
\red{\textbf{Szene}}:

Ortsbeschreibung: Eine kleine Absteige, die nicht mehr ganz so gut in Schuss ist. Eine Tatsache, die sich auch in der Kundschaft wiederspiegelt.

Ereignis: Schon vor der Tür wird die Gruppe unangenehm begrüßt. Fünf finstere Gestalten behaupten, es würde Eintritt kosten, in den Laden zu kommen.

Die Spieler müssen schauen, wie ihre Charaktere am besten in die Kneipe kommen. Es gibt unzählige Möglichkeiten.


Interaktionen:

Probe auf Menschenkenntnis:

Dass der Eintritt etwas kostet, ist eine Lüge.
Kampf:

Betrunkener Pirat:
Leben: 80
Waffe: Fäuste(70)
Schaden: 15
Parieren 5
Loot:
Sie finden bei den Männern ein verziertes Kreuz aus Silber. Es ist sogar ein besonders schönes Exemplar.

Zusatz (erfordert Wissen zu christlicher Kultur):
Es ist ein Kreuz, wie es sonst nur Priester tragen würden.
Auf der Rückseite ist Apostel Petrus eingraviert. Kunstfertig!
Im „Gelockten Hund”
\red{\textbf{Szene}}:

Der „Gelockte Hund” ist eine finstere Absteige und sieht dementsprechend heruntergekommen aus. Der Großteil des Publikums ist Gesindel, das Fremden gegenüber nicht sonderlich wohlgesonnen sein dürfte.

\textbf{Raumbeschreibung}: Es herrscht ausgelassene Stimmung. In der Kaschemme selbst ist erst mal aber nichts besonders ungewöhnlich.

Interaktionen:

Die Charaktere können sich umhören, ob jemand Gorich kennt.

Nach Gorich fragen:

Bis auf den Wirt Gert will ihnen niemand Auskunft geben.
Dieser hat allerdings gerade alle Hände voll zu tun.
Es muss gekocht und serviert werden, Musik ist auch keine da. Wenn sie ihm allerdings helfen würden …
Wirt Gert helfen:

Entscheidet sich die Gruppe dem Wirt Gert zu helfen, dann muss sie ihn nun bei Aufgaben in der Kneipe unterstützen.

Die Aufgaben:
Eintopf kochen.
Gäste bedienen.
Einen Streit schlichten.
Musizieren.

Für einen Erfolg müssen mindestens zwei der Aufgaben erfolgreich bestanden werden.

Erfolg:
Gorich gibt sich zu erkennen.

Misserfolg:
Gorich gibt sich nicht zu erkennen und kann nur noch mit Gewalt oder Tricks dazu gebracht werden, sich zu offenbaren. Dies ist aber STARK erschwert.
Gorichs Offenbarung \& Gespräch
\red{\textbf{Szene}}:

Gespräch mit Gorich: Er und seine Leute haben nichts mit der ganzen Sache zu tun. Er zeigt der Gruppe sogar, dass seine eigenen Kinder im Hinterzimmer liegen … krank. Sie hatten schon früh von der Seuche gehört und auch erfahren, dass es irgendwas mit Lebensmitteln zu tun haben könnte. Denn es waren zuallererst die Bauern und Karrenlenker krank geworden, die im Umland lebten. Also kaufte er all seine Nahrung nur noch aus Einfuhr. Das schien aber auch nichts zu helfen. Denn seine Frau ist bereits gestorben, und auch seinen Kindern gehe es immer schlechter. Ein gewisser Hagen habe es ihm verkauft. Dieser lebe in Hammerbrook, direkt vor den Toren der Stadt. Gekauft habe er die Güter direkt bei einem Lagerarbeiter. Er wisse, dass auch andere, die bei ihm gekauft haben, krank wurden.

Gorich bittet die Gruppe seine Kinder nicht zu verraten. Das würde den Untergang seines Geschäfts bedeuten, und dann könne er sich erst recht nicht mehr um sie kümmern.

Interaktionen:

Probe auf Menschenkenntnis:

Gorich scheint die Wahrheit zu sagen.
Jeder, der sich den Kindern von Gorich nähert, erhält Pestilenz +2.

Moral:

Was tut die Gruppe also mit Gorich und seinen Kinder?


Ende des relevanten Hafen-Plots.

Eeksdurf
Der Weg nach Eeksdurf
\red{\textbf{Szene}}:

Die Gruppe macht sich auf den Weg in Richtung des Westens der Stadt. Dort liegt das kleine Dorf Eeksdurf. Bisher wissen sie nicht viel über das Örtchen, aber das soll sich bald ändern.

Nach einer kurzen Strecke werden die Häuser am Straßenrand immer weniger.

Sie nehmen ein bis zwei Abzweigungen immer den Wegweisern nach.

Ereignis: Als sie etwa eine Stunde unterwegs sind, hält der Kutscher abrupt. Sie sehen im Straßengraben einen Karren liegen.

Daneben: Leichen.


Der Überfall
\red{\textbf{Szene}}:

Als sie näher kommen, können sie das ganze Ausmaß des Unheils sehen. Am Wegesrand liegt ein Wagen mit gebrochenem Rad. Daneben entdecken sie drei Leichen, eine davon offensichtlich die eines Priesters, sogar verhältnismäßig gut gekleidet. Leider scheint auch ihn ein unschönes Schicksal ereilt zu haben.

Interaktionen:

Umsehen:

Allen Opfern wurde der Schädel eingeschlagen und Teile ihrer Kleidung gestohlen.
Alle Wertsachen fehlen, und auf dem Karren finden sich Reste von Lebensmitteln. Auch von diesen fehlt aber jede Spur.

Zusatz (erfordert Wissen zu christlicher Kultur): Selbst das Kreuz, welches der Priester mit Sicherheit um den Hals trug, ist verschwunden.

Information für den Spielleiter: Das Kreuz kann die Gruppe an einem betrunkenen Piraten vor dem "Gelockten Hund" im im Hafen finden.

Sie finden keinen Hinweis darauf, wer der verstorbene Priester sein könnte. Allerdings entdecken sie einige Pergamente, die in lateinischer Schrift verfasst sind.

Zusatz (erfordert Lateinkenntnisse): Die Pergamente handeln von schwarzer Magie und anderem Volksglauben.

Ein gutes Dutzend Fußspuren führt weiter in Richtung Eeksdurf, aber auch genügend in Richtung Hamburg. Schwer zu sagen, ob sie von den Tätern oder anderen Reisenden stammen. Aber in jedem Falle sind die Spuren am Ort der Tat nicht älter als eine Stunde. Mehr ist hier nicht zu finden.

Ankunft in Eeksdurf
\red{\textbf{Szene}}:

Eeksdurf liegt vor den westlichen Toren der Stadt, am Rande des Eekshult.

Die Bewohner des Ortes gehen den üblichen Handwerksberufen und der Landwirtschaft nach. Ein beschaulicher Ort. Binnen zweier Wochen starben oder erkrankten bereits über die Hälfte der Bewohner des Dörfchens.

Das kann kein Zufall sein, vermuten einige im Dorf. Jemand sei mit dem Teufel im Bunde, wird gemutmaßt. Eine Hexenjagd entbrennt, vor der niemand sicher zu sein scheint.

Ortsbeschreibung: Als die Gruppe das Dorf betritt, ist alles verlassen. Alle Türen sind verrammelt, und es laufen keine Menschen auf den Straßen herum.

Klopfen sie an eine der Türen, dann antwortet man ihnen, wenn sie passend würfeln.

Interaktionen:

An Türen klopfen:

0 bis 49: Ihnen wird geantwortet.
50 bis 99: Ihr Klopfen bleibt unbeantwortet.

Gespräch mit Dorfbewohnern:

Niemand will sie hereinlassen.
Angeblich weiß niemand, wo all die Männer des Ortes sind.

Fragen sie nach einer rothaarigen Frau, erzählt man ihnen von Ruth, der Tochter der Kräutersammlerin Ottilde.
Diese wohne gleich der Hauptstraße nach am Rand des Eekshult.
Probe auf Menschenkenntnis:

Das die Bewohner nicht wissen, wo genau die Männer des Ortes gerade sind, ist eine Lüge.
Bei Ottildes Haus
\red{\textbf{Szene}}:

Ortsbeschreibung: Am Waldrand steht eine kleine Hütte. Sie ist nahezu verfallen, aber offensichtlich noch bewohnt.

Ereignis: Als die Gruppe sich der Hütte nähert, erkennt sie...

Wurf: Was sieht die Gruppe vor Ottildes Haus?
Einen wütenden Mob (0 bis 49)
Ein großes Feuer (50 bis 99)

Option 1 – Wütender Mob
\red{\textbf{Szene}}:

Ereignis: Vor der Hütte stehen wütend grölende Menschen und schwingen Mistforken, während eine alte, merkwürdig gekleidete Dame vor der Hütte versucht sie zu besänftigen.

Aus dem gegröle lässt sich schließen, dass es sich bei der Dame um die Kräuterfrau Ottilde handelt.

Der Rädelsführer des Mobs ist ein Bauer namens Wolfgang. Dieser fordert vehement die Herausgabe von Ruth. Sie sei an allem Schuld! Schwarze Magie habe sie angewandt und nicht mal die Kirche wolle den Bewohnern mehr helfen!

Interaktionen:

Die Gruppe sollte versuchen den wütenden Mob zu beruhigen.

Gespräch mit Leuten aus dem Mob:

Die Leute im Mob berufen sich immer wieder auf vier Argumente:

*Ruth soll wiederholt dabei gesehen worden sein, wie sie nachts tote Körper in die große Scheune gebracht habe.
*Ruth habe den Brief an die Kirche, welchen sie dem Totensammler mitgeben sollte, nie übergeben.
*Ruth habe die Pest gebracht. Sie war in Hamburg gewesen, um Korn zu kaufen.
*Kaum war sie zurück, brach die Krankheit aus! Zu diesem Zeitpunkt sei noch nirgends sonst in der Gegend jemand erkrankt! (Das ist drei Wochen her)
Probe auf Beruhigen:

Erfolg: Schaffen Sie es den Mob zu beruhigen, dann zieht sich der Mob zurück und sie können mit Ottilde in Ruhe das Haus betreten.
Dort treffen sie auch auf Ruth, die verängstigt in
einer Ecke sitzt.

Misserfolg: Schaffen sie es nicht den Mob zu beruhigen, dann bricht dieser in das Haus ein.
Von Ruth fehlt allerdings jede Spur. Ottilde ist dankbar für ihre Hilfe und verrät der Gruppe,
dass Ruth sich in der großen Scheune verstecke.
Gespräch mit Ruth:

Ruth sagt, sie habe nichts gemacht und wisse nicht, warum die Bauern so aufgebracht seien.
Probe auf Menschenkenntnis:

Ruth lügt. Hakt die Gruppe nach, dann erzählt sie, was vorgefallen ist:
Sie habe direkt am Nikolaifleet Korn holen wollen. Da sei es am günstigsten, habe man ihr gesagt.
Als sie dort ankam, sei es bereits spät gewesen. Sie habe am Lagerhaus, das man ihr beschrieben hatte, angeklopft, allerdings ohne Erfolg.
Also habe sie durchs Fenster geschaut und einen Mann erblickt, der sich über etwas gebeugt habe. Er soll eine Art Vogel-Maske und einen weiten Mantel getragen haben.
Ruth sagt, sie habe Angst gehabt und habe davonlaufen wollen. Sie sei auf dem Schnee ausgerutscht und habe das Gleichgewicht verloren.
Dabei sei mit dem Kopf aufgeschlagen und im Lagerhaus wieder zu sich gekommen. Der Mann habe sich ihr dann als Didrich vorgestellt. Er forsche an der Pestilenz, habe er erklärt.
Nur darum sei er so gekleidet und habe sich an der Leiche zu schaffen gemacht. Ruth habe ihm dann erzählt, dass in Eeksdurf zwar niemand krank sei, aber dass der Totensammler regelmäßig mit den Leichen durch ihr Dorf fahre, um diese tief im Eekhult zu verscharren.
Daraufhin habe Didrich ihr einen Handel vorgeschlagen, und … naja … Ruth habe angenommen. „Wieso auch nicht. Die sind schließlich tot und ich arm. Eins davon kann man noch ändern.“
Die Leichen habe sie also von da an immer in ein kleines Lagerhaus im Armenviertel Hammerbrook gebracht. Dort habe auch ihr Geld gelegen …
Ruth sagt ihnen genau, welches Lagerhaus das gewesen sei. Die Gruppe kann anschließend zum Armenviertel aufbrechen.

Option 2 – Ein großes Feuer
\red{\textbf{Szene}}:

Ereignis: Im Näherkommen sehen sie, dass vor der Hütte ein gewaltiges Feuer brennt.

Niemand ist zu sehen. Im Feuer erkennen sie allerdings die Überreste zweier Körper. Beides Frauen.

Die Tür der Hütte steht offen.

Interaktionen:

In der Hütte umsehen:

Sehen sie sich in der Hütte um, dann finden sie ein Blatt Pergament. Auf das Blatt wurde eine Art Wegbeschreibung gekritzelt – nur sehr grob. Der Weg führt zu einem Gebäude im Armenviertel Hammerbrook.
Genaueres Durchsuchen:

Suchen sie ausgiebiger, dann finden sie Geld. Etwas mehr Geld, als jemand, der hier lebt, besitzen sollte.
Verlassen der Hütte:

Beim Verlassen der Hütte trifft die Gruppe auf Wolfgang und Hermann. Diese werfen ihnen umgehend vor, mit der bösen Zauberin im Bunde zu sein.
Probe auf Beruhigen:

Erfolg: Beruhigen sie die beiden Bauern, erzählen diese der Gruppe, warum sie Ottildes Haus in Brand gesetzt haben.

Sie bringen folgende Argumente:
*Ruth soll wiederholt dabei gesehen worden sein, wie sie nachts tote Körper in die große Scheune gebracht habe.
*Ruth habe den Brief an die Kirche, welchen sie dem Totensammler mitgeben sollte, nie übergeben.
*Ruth habe die Pest gebracht. Sie war in Hamburg gewesen, um Korn zu kaufen.
*Kaum war sie zurück, brach die Krankheit aus! Zu diesem Zeitpunkt sei noch nirgends sonst in der Gegend jemand erkrankt! (Das ist drei Wochen her)

Misserfolg: Schaffen sie es nicht Wolfgang und Hermann zu beruhigen, so kommt es zum Kampf.
Wolfgang ist mit einer Mistforke bewaffnet. Außerdem sehen beide Bauern bereits von der Krankheit gezeichnet aus.

Kampf:

Wolfgang:

Leben: 70
Waffe: Mistgabel (70)
Schaden: 40
Parieren 30

Hermann:

Leben: 70
Waffe: Fäuste(70)
Schaden: 15
Parieren 5

Moral:

Die beiden Bauern brüsten sich mit der Tat. Wie reagiert unsere Gruppe darauf?


Ende des relevanten Eeksdurf-Plots.

Hammerbrook
Jeder, der das Armenviertel betritt, erhält Pestilenz +1.

Einleitung
\red{\textbf{Szene}}:

In Hamburgs Armenvierteln spürt man nichts vom neuen Reichtum. Dicht gedrängt leben hier Hafenarbeiter, einfache Leute und anderes Gesindel in ärmlichen Hütten und Häusern. Die Straßen sind gesäumt von Toten und Kranken, und in kaum einem Hause brennt Licht. Der Tod geht um und zeigt hier seine hässliche Fratze.

Auf den Straßen
\red{\textbf{Szene}}:

Ihr kommt also in Hammerbrook an, einem der Stadtviertel, in denen sich die Ärmsten der Stadt zusammenpferchen, in der Hoffnung, vom Reichtum zu profitieren, den der Handel in Hamburgs Kassen spült.

Ortsbeschreibung: Die Straßen sind verschneit und verlassen. Nur vereinzelt ziehen vermummte Gestalten umher und werfen unserer Gruppe argwöhnische Blicke zu.

Die Gruppe ist hier nicht willkommen …

Ereignis: Eine Gruppe Kinder kommt auf die Gruppe zu.

Probe auf Gassenwissen:
Alle würfeln einmal eine um 10 erschwerte Probe auf Gassenwissen.
Scheitert die Probe, wird dem Charakter etwas gestohlen, und die Kinder laufen damit davon. Es muss ein wichtiger Gegenstand sein, etwas, das die Person nicht missen möchte.
Anschließend können die Charaktere die Kinder verfolgen.
Probe auf Rennen:
Alle verfolgenden Charaktere würfeln eine Probe auf Rennen o.ä.

Erfolg: Holen sie die Kinder ein, geht es beim Lumpensammler weiter.

Misserfolg: Holen sie die Kinder nicht ein, ist der Gegenstand verloren.
Die Gruppe hat daraufhin eine Begegnung mit Sigrun.

Option 1 – Sie holen die Kinder ein und kommen zum Lumpensammler
\red{\textbf{Szene}}:

Gespräch mit dem Lumpensammler: Hier wird die Gruppe künstlich, freudig begrüßt.

Der alte Lumpensammler ist im Viertel gut bekannt und handelt mit allem, was Menschen so zu entbehren haben.

Woher seine Waren kommen, ist ihm dabei recht egal. Ist ja nicht sein Problem. Nun hat er den Gegenstand oder die Gegenstände der Gruppe „erworben“ und will ihn/sie auch nicht ohne Weiteres wieder herausgeben.

Allerdings lässt er sich auf eine kleine Wette ein. Denn so piekfeine Schnösel können doch bestimmt gut ...

Was können so piekfeine Schnösel bestimmt gut?
Ein Rätsel lösen (0 bis 49)
Ein „Problem“ lösen (50 bis 99)
Option 1 – Ein Rätsel
Ereignis: Voller Hochmut stellt der Lumpensammler unserer Gruppe nun ein Rätsel, das es zu lösen gilt.


Das Rätsel:
Einst wurde ein Pirat gefasst und sollte hängen. Es war aber üblich, den zum Tode verurteilten Dieben eine letzte Chance zu geben.

Sie mussten aus einem Säckchen einen Stein ziehen. Im Säckchen befanden sich ein weißer und ein schwarzer Stein. Zog der Dieb den weißen Stein, wurde ihm die Freiheit geschenkt. Zog er hingegen den schwarzen Stein, so baumelte er. Eines Tages kam nun jener Pirat an die Reihe, der einst das Gold des Königs geraubt hatte.

Der König wollte also sichergehen, dass dieser Halunke hängt. Er befahl dem Henker heimlich, zwei schwarze Steine ins Säckchen zu legen.

Am nächsten Tag ging der König zuversichtlich und voller Rachelust zum Galgen. Dort lagen überall schwarze und weiße Steine.

Der Henker nahm zwei von ihnen auf, aber der Verurteilte konnte sehen, dass der Henker zwei schwarze Steine in das Säckchen legte.

Der Dieb hatte den Strick schon um den Hals, als ihm die rettende Idee kam.

Er zog und musste freigelassen werden.

Was war es, dass dem Piraten das Leben rettete?


Lösung 1:

Er nimmt beide Steine aus dem Beutel und zeigt so, dass beide schwarz sind und der Henker ihn betrügen wollte.
Lösung 2:

Er zieht einen der Steine und wirft ihn weg. Da der verbleibende Stein im Sack schwarz ist, muss der gezogene Stein scheinbar weiß gewesen sein.
Erfolg: Beantwortet die Gruppe das Rätsel richtig, bekommen sie ihren Gegenstand zurück.

Misserfolg: Liegen sie falsch, will der Lumpensammler nicht verkaufen. Er genießt den Triumph viel zu sehr! Die entwendeten Gegenstände sind verloren.
Erfolgreiches Abschliessen der Aufgabe: Lösen sie die Aufgabe, berichtet der Lumpensammler ihnen, dass sie nicht die ersten Schnösel seien, die hier waren. Einer mit ganz komischem Aufzug sei dagewesen. Der habe nach abstrusen Pflanzen gefragt. Chillies und Zitronen seien darunter gewesen. Als ob man sowas hier bekomme. Außerdem habe er noch einen Blasebalg gewollt und etwas grobes Leinen, Lumpen und Teer. Komischer Typ.


Option 2 – Ein Problem
Ereignis: Der Lumpensammler führt die Gruppe in ein Hinterzimmer.

Als er die Tür öffnet, drückt sich ihnen der Gestank des Todes entgegen. In dem Raum liegen zwischen allerlei Unrat zwei Menschen, dem Tod bereits nahe.

Aufgabe: Der Lumpensammler will sie loswerden, aber denkt nicht daran, sie anzufassen. Er will sich schließlich nicht anstecken. Die Gruppe soll die beiden „entsorgen“, egal wie.

Jeder der die Sterbenden anfasst: Pestilenz +3.

Ereignis: Allerdings kommt in diesem Moment der Totensammler vorbei und nimmt ihnen die Leichen nur zu gerne ab. Er klagt, dass die Entsorgung nicht so einfach sei.

Interaktionen:

Nachfragen beim Totensammler: Fragen sie genauer nach, erzählt er ihnen von Eeksdurf, wo jemand die Leichen kaufe, aber nur Pestkranke.

Hilfe verweigern: Helfen sie nicht, gibt der Lumpensammler ihnen den Gegenstand bzw. die Gegenstände nicht wieder, egal was sie tun.


Erfolgreiches Abschliessen der Aufgabe: Lösen sie die Aufgabe, berichtet der Lumpensammler ihnen, dass sie nicht die ersten Schnösel seien, die hier waren. Einer mit ganz komischem Aufzug sei dagewesen. Der habe nach abstrusen Pflanzen gefragt. Chillies und Zitronen seien darunter gewesen. Als ob man sowas hier bekomme. Außerdem habe er noch einen Blasebalg gewollt und etwas grobes Leinen, Lumpen und Teer. Komischer Typ.

Option 2 – Das Treffen mit Sigrun
\red{\textbf{Szene}}:

Nachdem die Gruppe es nicht schafft die Kinder einzuholen begegnen sie auf der Straße einer Frau namens Sigrun.

Gespräch mit Sigrun: Sie ist gerade auf dem Weg zur St. Petri-Kirche. Dort werde niemand krank, erzählt sie. Es wären unzählige Menschen dort, die dem Ruf von Pater Salus gefolgt seien. Sie sei dort sicher. Allerdings müsse sie erst in Erfahrung bringen, ob auch ihre Kinder dort willkommen seien. Die lägen zu Hause … krank. Sigrun bittet die Gruppe, bei ihr zu Hause vorbeizuschauen und nach den Kindern zu sehen, während sie weg ist.

Moral:

Wie wird die Gruppe mit der Bitte von Sigrun umgehen?

Entscheidung: Die Gruppe muss sich entscheiden.

Sie helfen den Kindern nicht
Sie helfen den Kindern
Option 1 – Sie helfen den Kindern nicht
Ereignis: Sigrun zieht traurig davon, und die Gruppe bleibt auf sich gestellt.


Option 2 – Sie helfen den Kindern
\red{\textbf{Szene}}:

Die Gruppe kommt am Haus an. Es stinkt nach Tod.

Sie alle erhalten sofort +2 Pestilenz.

Die Kinder weinen und klagen und sind kaum noch bei Verstand.

Kümmern sie sich weiter um die Kinder, erhalten sie nochmals Pestilenz +2.

Ereignis: Plötzlich klopft es an der Tür.

Gespräch mit Hagen: Hagen, der Nachbar, steht vor der Tür und fragt die Gruppe, was sie hier suchten und wo Sigrun sei. Nach einer Erklärung lädt er sie zu sich auf Tee und einen Plausch ein. Er habe ihnen etwas Spannendes zu erzählen!

Bei Hagen vom Fleet
\red{\textbf{Szene}}:

Hagen begrüßt die Gruppe zunächst sehr freundlich.

\textbf{Raumbeschreibung}: Sein Haus ist aufgeräumt. Nirgendwo liegen Waren oder ähnliches herum.

Gespräch mit Hagen:

Anmerkung: Hat Hagen die Gruppe nicht zu sich eingeladen (falls die Gruppe ihn von selbst aufsucht), dann verlangt er im Gegenzug für Informationen einen Aufseherposten innerhalb der Hanse. Die Gruppe kann zustimmen oder ablehnen. Hauptsache sie kommen an die Infos.

Kommen sie von der Nachbarin und werden von ihm eingeladen, dann erzählt er von sich aus.
Hagen berichtet, dass er seit geraumer Zeit heimlich Waren abzweige. Er sei nicht stolz darauf, aber man müsse eben sehen, wo man bleibt. Jedenfalls komme er nachts am Nikolaifleet an diese Waren. Dort arbeite er. Als immer mehr seiner Kollegen krank wurden – das startete bereits vor vier Wochen, also vor allen anderen Ausbrüchen – wurde das sogar noch leichter. Eines Nachts jedenfalls schlich er sich wieder ins Lager, als ihm ein eigenartiger Mann begegnete. Er trug eine lange Maske, die beinahe wie der Schnabel eines Vogels aussah, ein weites Gewand und einen Stock bei sich. Außerdem roch es nach … Parfüm. Der Mann floh, als er Hagen sah, und ließ nichts außer einer Rattenfalle zurück. Diese war aber leer.

Anmerkung: Hagen will sie auf keinen Fall begleiten. Das ist ihm zu gefährlich. Auch mit Gewalt oder Drohungen kann man ihn nicht überzeugen!

Das Lagerhaus von Didrich
\red{\textbf{Szene}}:

Das Lagerhaus, welches Ruth beschrieben hatte, findet die Gruppe recht nahe der Stadtmauer, gerade vor den Toren der wohlhabenden Stadt.

Dennoch sieht es hier erbärmlich aus. Als sie am Lagerhaus ankommen ist dieses verschlossen. Drinnen brennt jedoch Licht.

Interaktionen:

Einbruch:

Es gibt verschiedene Möglichkeiten einzudringen.
Ereignis: Wenn sie einbrechen, tauchen aus der Dunkelheit zwei kräftige Männer auf, die sehr skeptisch sind. Ablenkung ist gefragt!


Im Inneren
\red{\textbf{Szene}}:

\textbf{Raumbeschreibung}: Im Lagerhaus ist es eiskalt. Allerdings brennt eine riesige Öllampe, die auf einem Tisch steht, auf dem allerlei Dinge liegen.

Darunter liegt eine tote Ratte, eine Zitronenschale und etwas von einem roten, scharfschmeckenden Pulver (Chili).

Interaktionen:

Türen öffnen:

Es gibt drei Türen zu kleineren Räumen. Bei allen sind jegliche Ritze und Schlitze säuberlich mit Lumpen und Lappen verstopft und verteert. Eine Ritze ist offen.
Öffnen sie diese Tür, entdecken sie schlimm zugerichtete Leichen von Pestkranken.

Alle im Haus bekommen Pestilenz +2.

Eine Tür ist offen. Dahinter finden sie üppige Kornvorräte. Auf den Säcken steht die Adresse eines Lagerhauses am Nikolaifleet geschrieben.
Öffnen sie die dritte Tür, strömen Ratten heraus. Sie sind ausgehungert und aggressiv. Sie greifen die Gruppe an!

Kampf:

Rattenschwarm:
Leben: 10 Ratten mit jeweils 10 Lebenspunkten
Waffe: Biss (40)
Schaden: Jeder Angriff wird mit einem W10 geworfen. Das Ergebnis sind die Ratten, die erfolgreich angreifen. Jeder Biss macht dabei 4 Schaden.
Ausweichen 10

Sind die Ratten besiegt, finden sie auch hier Säcke mit der Adresse eines Lagerhauses am Nikolaifleet.

Ende des relevanten Hammerbrook-Plots.

Die Kirche St. Petri
Einleitung
\red{\textbf{Szene}}:

Die Kirche St. Petri steht ganz im Zeichen des neuen Hamburger Wohlstands. Noch immer wird gebaut, aber man sieht ihr schon jetzt an, dass sie eines der zukünftigen Wahrzeichen der Handelsmetropole sein wird.

Viele Menschen pilgern angesichts ihrer Machtlosigkeit gegenüber der Pest hierher, um Schutz und Trost zu suchen. Kranke und Verzweifelte säumen die Gänge, und der Geruch von Tod liegt in der Luft.


Ankunft an der Kirche
\red{\textbf{Szene}}:

Ortsbeschreibung:Sie stehen vor der gewaltigen Pforte einer der schönsten Kirchen Hamburgs. In neuem Glanz erstrahlt St. Petri und ist aufgrund der Lage auf einem Geestrücken bereits von Weitem sichtbar.

Ereignis: Sobald jemand an die Tür klopft werden sie von Helfern begrüßt und hereingelassen.

Interaktionen:

Probe auf Wahrnehmung, Aufmerksamkeit, Kirchenwissen oder ähnliches:

Sie tragen zwar Mönchsgewänder, scheinen aber keinem Orden zugehörig zu sein.
Zumindest nicht erkennbar.
Nach Pater Salus fragen:

Wird einer der Helfer nach Pater Salus gefragt, so wird ihnen berichtet, dass dieser aufgebrochen sei, um Gerüchten auf den Grund zu gehen.
Nun ist von schwarzer Magie die Rede, böse Mächte seien in der Stadt am Werk. Die Gruppe ist aber herzlich eingeladen zu bleiben, sofern sie gesund sind und dies beschwören.
\red{\textbf{Szene}}:

Ereignis: Dann lässt man sie in den Hauptraum der Kirche. Hier sitzen überall Menschen mit besorgten Gesichtern herum. Andere wiederum blicken voller Zuversicht und lauschen gebannt Gesprächen, Gesängen und Predigten, welche Mönche überall im Raum zum Besten geben. Außerdem gehen verdächtig viele Klingelbeutel herum. Es führen einige Türen aus dem Hauptraum. Alle sind verschlossen.

\textbf{Raumbeschreibung}: Das innere einer großen Kirche. Man sieht, dass die Bauarbeiten noch im Gange sind, doch im Hauptraum kann man sich bereits ohne größere Probleme aufhalten. Trotz Bau lässt sich bereits eine prunkvolle Kirche erahnen.

Interaktionen:

Probe auf Wahrnehmung,Aufmerksamkeit, Lauschen oder ähnliches:

Hört die Gruppe bei den Mönchen genauer hin, ist die Rede vom Jüngsten Gericht,
Sünde und der Strafe des Herren, die alle Sünder trifft.
Man könne sich aber von seinen Sünden befreien.
Nur etwas weltliches Gut müsse man opfern, um hier,
bei den Frommen, bleiben zu dürfen.
Gespräch mit Anwesenden:

Fragen sie die Leute, so berichten manche davon, dass sie mit Angehörigen hier waren.
Die wurden dann aber irgendwann nach hinten gebeten, wohl weil sie besonders fromm waren. Während sie so lauschen, bemerken sie, dass immer wieder einzelne Menschen aus dem Saal durch die Seitentüren gebracht werden.
Man will sie hier aber nicht durchlassen. Das seien „private Gemächer“ der Priester und Mönche. Die Gruppe kann versuchen, sich Zugang zu verschaffen.
Im Hinterzimmer
\red{\textbf{Szene}}:

Schaffen sie es irgendwie hier hinein, geht es erstmal durch ein paar Gänge.

Es stinkt nach Tod. Alle erhalten Pestilenz +1.

Nach einer Weile kommen sie in eine Art Gewölbe. Hier liegen unzählige Kranke, von gerade so infiziert bis zu bereits verstorben, hinter einem Gitter eingesperrt wie in einer Zelle.

Gehen sie näher an die Menschen heran, erhalten sie Pestilenz +3.

Sie alle sind überzeugt davon, dass sie das hier als Sünder verdient haben.

Am hinteren Ende des Raumes gibt es eine Tür. Sie scheint ins Freie zu führen. Davor am Gitter steht eine junge Frau, Gundel. Sie ist bereits von der Pest gezeichnet, winkt die Gruppe aber zu sich.

Gespräch mit Gundel: Sie bittet die Gruppe, ihr dabei zu helfen zu entkommen. Sie müsse zu diesem Arzt, diesem Didrich. Der wisse, was zu tun ist. Da ist sie sich sicher! Gundel arbeitet am Nikolaifleet. Dort kocht sie für die Männer. Nach und nach wurden sie alle krank.

Plötzlich tauchte der Arzt auf, Didrich von Sinnfeld. Er versprach ihr, wenn sie ihm ein paar Ratten aus der Küche fange, würde er sie fürstlich bezahlen. Sie lehnte ab. Das schien ihr doch sehr merkwürdig, und die Bezahlung sei nicht üppig gewesen. Als er aber sagte, dass er an etwas arbeite, um den Schwarzen Tod zu besiegen, lenkte sie ein. Sie gab ihm die Ratten. Doch dann wurden immer mehr Leute krank. Sie bekam es mit der Angst zu tun und kam hierher. Doch es war zu spät. Man erleichterte sie um ihr Geld und sperrte sie dann hier zum Sterben hinein.


Moral:

Lassen sie Gundel und die anderen raus und gefährden damit die Stadt? Oder gehen sie und überlassen die Kranken ihrem unumgänglichen Schicksal?

Vorm Ausgang
\red{\textbf{Szene}}:

Ereignis: Hier treffen sie noch einmal den Totensammler. Er schaut sie vorwurfsvoll an und zieht dann weiter klingelnd seine Runden.


Ende des relevanten St.Petri-Plots.

Der Nikolaifleet
Einleitung
Erst seit wenigen Jahren wächst am Nikolaifleet ein gewaltiges Lagerzentrum für die Waren aus aller Welt.

Der neue Reichtum trägt hier Früchte, und in direkter Nähe zu Tee, Gewürzen und Tulpen siedeln sich die betuchteren Bürger Hamburgs in prächtigen Villen an. Alles scheint makellos und vom Chaos der übrigen Stadt unberührt. Fast schon zu makellos.

Ankunft am Fleet
\red{\textbf{Szene}}:

Ortsbeschreibung: Der Nikolaifleet ist ordentlich und sauber. Weißer Schnee fällt auf frisch gepflasterte Straßen. Niemand ist zu sehen, obwohl hier reges Treiben herrschen sollte. Die Gruppe findet recht unkompliziert das Warenlager.


Lagerhaus Grote Buur
\red{\textbf{Szene}}:

„Grote Buur“ steht in eisernen Lettern überm Eingang. Zur Überraschung aller ist die Tür nicht verschlossen.

\textbf{Raumbeschreibung}: Ein typisches Lagerhaus. Auf der Rückseite sind die Tore gut sichtbar, die zum Löschen der Boote genutzt werden, welche die Waren an den Fleet bringen. Das Lagerhaus hat gleich mehrere Etagen, und es gibt diverse Kräne und Seilzüge, um Waren auf den Ebenen zu transportieren. Bei näherer Betrachtung erkennt man, dass sich hier einige Ratten niedergelassen haben.

Ereignis: Niemand scheint hier zu sein. Doch dann hören sie eine Art Jammern aus den Untergeschossen. Sie finden eine Luke. Darunter sitzt ein gefesselter Mann. Er hat überall am Körper Wunden, die nicht nur von der Pest verursacht wurden.

Interaktionen:

Begegnung mit dem Mann:

Dem Mann fällt das Sprechen sehr schwer, aber er kann noch „Sinnfeld“ und „gegenüber“ stammeln.

Die Gruppe kann ihn retten, erhält dann aber Pestilenz +3.

Andernfalls stirbt er vor ihren Augen.
Körper untersuchen: Die Wunden an seinem Körper sind unter anderem Rattenbisse. Außerdem ist er übersäht mit Flöhen.

Das Handelsregister
\red{\textbf{Szene}}:

Das Handelsregister ist der Ort, an dem alles Wissen der Stadt um jeden Geschäftsmann und seine Machenschaften zusammenkommt.

Ortsbeschreibung: Das Gebäude ist modern, fein verziert und kann sich sehen lassen.

Die Gruppe kann hierhin gehen, um Auskunft über den Eigentümer des „Grote Buur“ und Didrich von Sinnfeld einzuholen.

Beim Eintreten sehen sie eine Beamtin, die offensichtlich das Sagen hat, und ein paar um sie herum wuselnde Mitarbeiter.

Interaktionen:

Auf Anfrage kann die Gruppe hier ein Dokument erhalten, das belegt, dass Didrich von Sinnfeld der Eigentümer des „Grote Buur“ ist und gegenüber wohnt.

Das Haus gegenüber
\red{\textbf{Szene}}:

Gleich gegenüber des „Grote Buur“ steht ein ansehnliches Bürgerhaus.

Ortsbeschreibung: Es ist gepflegt, sauber und von außen in hervorragendem Zustand.

Ereignis: Die Gruppe kann an die Tür und anklopfen. Als sie klopfen öffnet ihnen...

Wurf: Wer öffnet?
Ein Mann (0 bis 49)
Eine Frau (50 bis 99)
Option 1 – Ein Mann
\red{\textbf{Szene}}:

Ereignis: Die Tür öffnet sich einen Spalt, und eine Stimme ertönt.

Man sieht aber niemanden. „Wer da?“

Die Person reagiert erschrocken auf die Stimmen unserer Charaktere und stürmt nach hinten ins Haus davon. Verfolgen sie den Mann, sehen sie diesen um eine Ecke rennen und hören dann ein gewaltiges Klirren und Krachen.

Sie kommen in einen düsteren Raum voller medizinischer Instrumente und anderer abstruser Gerätschaften. Überall im Raum sind Notizen und Tagebücher verteilt. Auf dem Tisch in der Mitte des Raumes liegt ein kleiner Junge. Er ist tot und gezeichnet von der Pest. Sein Brustkorb und sein Kopf sind geöffnet. Seine Organe liegen auf Tellern neben seinem Körper. Sein Gesicht ist in Panik und Schmerz erstarrt. Vor dem Tisch auf dem Boden liegt der Körper eines bewusstlosen Mannes, der eben noch vor ihnen geflohen war. Er hat das Blut des Jungen an den Händen.

Interaktionen:

Dursuchen oder Lesen der Notizen und Tagebücher:

Die Notizen im Raum verraten, dass der Mann an etwas rund um die Pest geforscht hat und das Ganze etwas mit Ratten zu tun gehabt haben dürfte. Jedenfalls tauchen diese überall auf. Auch Beschreibungen zum Bau von Geräten (Masken und Kleidung), die eine Ansteckung verhindern, gibt es.

Probe auf Medizin:

Außerdem finden sie unzählige Zeichnungen sezierter Körper sowie Berichte von Obduktionen lebender Patienten und Krankheitsverläufen bei Gefangenen. Harter Toback. Aber auch wichtige Erkenntnisse. Diesen Unterlagen zufolge hat die Krankheit ihren Ursprung in der Rattenpopulation, und hält man sich von Kranken und Ratten fern, so kann man eine Ansteckung verhindern.
Moral:

Was macht die Gruppe mit Didrich und den Informationen?

Option 2 – Eine Frau
\red{\textbf{Szene}}:

Ereignis: Klopfen sie an, öffnet die Magd Traudel.

Gespräch mit Traudel: Im Haus wohne ein Herr von Sinnfeld, Didrich von Sinnfeld. Ja, er sei zu Hause. Die Gruppe möge eintreten!

Die Magd führt sie in den Salon. Sie werden gebeten, zu warten.

Ereignis: Nach einer Weile hören sie einen Schrei. Dann rennt das Hausmädchen an ihnen vorbei zur Tür hinaus.

Geht die Gruppe dem nach, finden sie Didrich in seinem Arbeitszimmer. Dieser beugt sich gerade über den leblosen Körper eines kleinen Jungen und ist im Begriff, ihn zu sezieren.

Gespräch mit Didrich:

„Ein Kleingeist, die gute Traudel. Viel zu leicht zu erschrecken. Und nun?! Habe ich nichts als Ärger am Hals.

Denn die werten Herrschaften, so nehme ich an, stehen meinem Treiben hier ebenso wenig wohlgesonnen gegenüber wie andere Vertreter ihrer Stände.

Sehe ich das richtig?”

Didrich schildert ihnen, dass er an der Krankheit forsche. Er sei kurz vor einem Durchbruch. Es wisse nun mit Sicherheit, dass es etwas mit Ratten zu tun habe. Nur der genaue Ablauf der Ansteckung sei ihm noch schleierhaft. Aber ein Fehlen von Ratten in einer Stadt ginge unmittelbar mit einem Abhandensein der Seuche einher, obwohl durch Kontakt zu Kranken ebenfalls eine Seuche zustande kommen könne. Seine Forschungen seien nicht immer ganz lupenrein gewesen. Er brauchte tote Körper, später lebendige, das gebe er zu. Zum Glück gäbe es genug Kranke direkt hier am Fleet. Gleich gegenüber fielen die Arbeiter reihenweise um. Aber dann blieben sie zu Haus. Ein Besuch auf Hammerbrook sei also unumgänglich geworden. Dann noch einer. Und noch einer. Jedenfalls könne man solche Forschungen in einer kleingeistigen Stadt wie Hamburg nicht ohne Weiteres betreiben. Daher die Heimlichtuerei. Was nun? Soll er seine Forschungen fortsetzen?

Ihm ist durchaus bewusst, weshalb die Gruppe hier ist.


Moral:

Was tun sie mit Didrich und seinen Forschungen?

Ende des relevanten Nikolaifleet-Plots.

Der Abschlussbericht
\red{\textbf{Szene}}:

Gespräch mit dem Rat: Am letzten Tage, oder wenn sie sich eher im Stande sieht, muss die Gruppe vor den Hamburger Rat treten. Sie werden aufgefordert, ihre Ergebnisse zu präsentieren. Dabei stellt der Rat durchaus Fragen und ist kritisch. Außerdem wurden dem Rat Informationen über das Handeln der Gruppe zugetragen, gerade was die moralischen Fragen betrifft:

Wie ist die Gruppe mit Gorichs Familie umgegangen?
Haben sie die Kranken in St. Petri befreit?
Wie sind sie mit den Bauern Wolfgang und Hermann umgegangen?
Was machten sie mit Didrich?
Haben sie Sigruns Kindern geholfen?
Hat die Gruppe alle Fragen beantwortet, will sich der Rat beraten.

Was hält der Spielleiter vom Ergebnis der Recherche, und was soll nun geschehen?

1. Die Gruppe war sorgfältig. Setzen wir ihren Vorschlag um!
2. Die Gruppe war nicht überzeugend genug. Warten wir ab!
3. Die Gruppe war nachlässig. Sie sollen bestraft werden!
Die Entscheidung des Spielleiters beendet das Abenteuer.
