%!TEX root = ../main.tex

\red{\textbf{Szene}}:

Nachdem ihr aus dem Rat entlassen wurdet macht ihr euch also auf den Weg zum Sitz des Beirats. In der unmittelbaren Umgebung des Ratshauses spürt man den Aufstieg Hamburgs als Handelsmetropole am deutlichsten. Ihr schlendert durch breite Straßen die von hohen Häusern gesäumt werden. Dienstboten eilen über den Pflasterstein und auch sonst herrscht geschäftiges treiben, als ihr unweit der St. Michaelis Kirche vor eine Kapelle tretet, die euch der Rat als eure Operationszentrale genannt hat

\textbf{Raumbeschreibung}: Als ihr an der kleinen Kapelle ankommt, erkennt ihr, dass es sich dabei um ein durchaus anschauliches Gebäude handelt, dass erst kürzlich einen neuen Anstrich mit weißer Farbe erhalten hat. Im Inneren stehen allerhand Tische und Stühle herum. Außerdem gibt es Schlafmöglichkeiten und so ziemlich alles, was man zum Leben so gebrauchen kann. Selbst ausgewählte Speisen stehen bereits zum Verzehr bereit.

\textbf{Ereignis}: Die Gruppe kann sich erst mal unterhalten. Ihr Gespräch wird allerdings von einem Klopfen unterbrochen ...

\begin{tcolorbox}
  Wurf: Wer oder was unterbricht das Gespräch unserer Gruppe?
  Das Militär (1 bis 33) (gehe zu \ref{militär}) \\
  Der Tod (34 bis 66) (gehe zu \ref{tot}) \\
  Ein Kind (67 bis 99) \ref{kind} \\
  Bei 100: würfle erneut.
\end{tcolorbox}

\section{Der militärische Besuch}
\label{militär}

\red{\textbf{Szene}}:

General zur Brügge steht davor und bittet um Einlass.

Gespräch mit Brügge: Dieser berichtet ihnen in vehementem Ton, dass die ganze Krise ein Werk der Dithmarscher sei. Diese hätten sich jahrelang an Hamburgs Handelsschiffen gütlich getan. Nun, da es einen Vertrag gibt, der das verhindert, versuchen einige von ihnen die Stadt zu schwächen, um davon zu profitieren oder sie gar ganz an sich zu reißen. Die Dithmarscher operieren von ihrem Versteck aus, das sich in einer Hafenkaschemme namens „Beim Gelockten Hund“ befinden soll. Ein gewisser Gorich leite das Ganze. Dort sollten sie mit ihren Recherchen beginnen.

Interaktionen:

Probe auf Menschenkenntnis:

General zur Brügge sagt schon die Wahrheit, aber seine Perspektive könnte durchaus verzerrt beziehungsweise einseitig sein.
Die Charaktere können sich aber sicher sein, dass er nicht lügt, zumindest seiner Auffassung nach nicht.

Ab hier können die Spieler frei entscheiden, wohin sie gehen wollen! Nach Ablauf der Frist von vier Tagen müssen sie beim Hamburger Rat vorsprechen. Bis dahin müssen sie sich auf eine Handlungsempfehlung festgelegt haben!


Option 2 – Der Tod
\red{\textbf{Szene}}:

Ereignis: Während sich die Gruppe noch unterhält, hören die Charaktere plötzlich ein lautes Klopfen an der Tür.

Davor steht der Totensammler Hanno. Er fragt, ob es Tote gäbe, die abzuholen seien, und ob im Haus bereits die Pest wüte.

Gespräch mit Hanno: Im Armenviertel sei es am Schlimmsten. Die Leichen könne er kaum mehr entsorgen. Man müsse kreativ werden.

Gegen Bestechung verrät er, dass er jemandem Leichen verkaufe. Dazu müsse er sie allerdings recht weit fortbringen, nämlich in einen kleinen Ort namens Eeksdurf vor den Toren der Stadt. Dort hinterlege er die leblosen Körper in einem Lagerhaus, wo bereits seine Bezahlung auf ihn warte. Die Absprache habe er dereinst mit einer jungen rothaarigen Frau getroffen.

Sie habe ihn angesprochen, nachdem sie ihn beim Abholen von Leichen im Dorf sah...


Ab hier können die Spieler frei entscheiden, wohin sie gehen wollen! Nach Ablauf der Frist von vier Tagen müssen sie beim Hamburger Rat vorsprechen. Bis dahin müssen sie sich auf eine Handlungsempfehlung festgelegt haben!


Option 3 – Ein Kind
\red{\textbf{Szene}}:

Ereignis: Es klopft plötzlich an der Tür und davor steht ein Kind zusammen mit seiner stark vermummten Mutter.

Gespräch mit der Mutter und dem Kind: Sie kämen aus dem Armenviertel Hammerbrook und seien auf der Suche nach der St. Petri-Kirche. Sie soll ein Zufluchtsort für gesunde und sündenfreie Menschen sein. Niemand werde dort krank! Für sie sei es zu spät, hustet die Frau, aber ihr kleines Kind, das sei noch zu retten. Sie wissen das alles von einem Mann, der im Armenviertel nach den Leuten sehe. Er werde nicht krank, egal was er tue ... Er habe sie losgeschickt. Sie wüssten gern den Weg.

Die Gruppe kann eine Beschreibung von Didrich von Sinnfeld erhalten. Außerdem geben ihnen die beiden auf Nachfrage den Tipp, einmal beim Lumpensammler im Armenviertel vorbeizuschauen.


Ab hier können die Spieler frei entscheiden, wohin sie gehen wollen! Nach Ablauf der Frist von vier Tagen müssen sie beim Hamburger Rat vorsprechen. Bis dahin müssen sie sich auf eine Handlungsempfehlung festgelegt haben!
