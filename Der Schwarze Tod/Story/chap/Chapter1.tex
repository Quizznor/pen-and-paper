%!TEX root = ../main.tex

\subsection{Das Wartezimmer}

\red{\textbf{Szene}}: Im Wartezimmer des Ratshauses

So begibt es sich, dass ihr geduldig vor einer gewaltigen Holztür im Ratshaus an der Troßtbrücke sitzt. Am gestrigen Abend wurdet ihr von Boten aufgesucht, die euch baten am folgenden Tag vor dem Hamburger Rat zu erscheinen, zu den man euch nun jeden Moment rufen wird.

Was tut ihr?

\red{\textbf{Interaktionen}}:

Die Gruppe hat noch etwas Zeit sich zu unterhalten und etwas umzusehen, bevor sie vor den Hamburger Rat gerufen werden. Sie können frei entscheiden, ob sie andere Spielcharaktere aus ihrem alltäglichen Leben bereits kennen.

\textbf{Raumbeschreibung}: Die Gruppe sitzt in einer Art Warteraum, der für damalige Verhältnisse sehr üppig eingerichtet ist. Es hängen mehrere Bilder von großen Hamburger Persönlichkeiten an den Wänden. Außerdem stehen kleine exotische Leckereien bereit, und auch Getränke werden angeboten. In einer Ecke des Raumes steht ein Ratsdiener.

An der Wand hängt eine Karte (siehe Abbildung \ref{fig:Karte}) der Stadt, welche die Charaktere sich ansehen können. Diese wird den Spielern im weiteren Verlauf des Abenteuers auch zur Verfügung stehen.

\begin{figure}[t]
	\begin{center}
		\includegraphics[scale=0.7]{./images/Karte.png}
		\caption{Stadkarte von Hamburg, anno 1350}
    \label{fig:Karte}
	\end{center}
\end{figure}


\subsection{Das Gespräch mit dem Rat}

Also Ihr euch also unterhaltet öffnet sich plötzlich die Holztür und ein junger Ratsdiener bittet euch vor den Rat zu treten.

\red{\textbf{Szene}}: Vor dem Hamburger Rat

\textbf{Raumbeschreibung}: Ihr tretet in einen großen Raum ein, der von einer U-förmigen Tischreihe dominiert werden, an dem 18 Männer in erhöhter Position sitzen. Auch wenn der Raum sonst nur spärlich eingerichtet ist, ist offensichtlich dass sich hier sonst niemand aufhält, der wenig Geld hat. Aus den Wänden sind kunstvoll Löwenköpfe und andere Muster herausgearbeitet. Eine Wand ist von hohen Fenstern gesäumt, die den Raum hell erleuchten.

Im Hintergrund huschen Ratsdiener mit Papieren umher oder gehen anderen Aufgaben nach.

\red{\textbf{Info}: Die 18 Männer setzen sich aus neun Rechtskundigen, sieben Kaufleuten und zwei Vertretern der Kirche zusammen.}

\red{\textbf{Interaktion}}:

test \ref{Frieder}

Vorsitzender des Rates:
%\gqm{\textit{}}
