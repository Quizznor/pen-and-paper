%!TEX root = ../main.tex

\purple{\textbf{Pestilenz}: Jeder, der das Armenviertel betritt, erhält +1 Pestilenz.}

\red{\textbf{Szene}}:

Ihr kommt also in Hammerbrook an, einem der Stadtviertel in denen sich die Ärmsten der Stadt zusammenpferchen in der Hoffnung vom Reichtum zu profitieren den der Handel in Hamburgs Kassen spült. Dicht gedrängt leben hier Hafenarbeiter, einfache Leute und anderes Gesindel in ärmlichen Hütten und Häusern. Man merkt, dass hier andere Regeln zu gelten scheinen als im Rest der Stadt, die Straßen sind gesäumt von Toten und Kranken, und in kaum einem Haus brennt Licht. Der Tod geht um, und tagtäglich rechnen hier viele mit dem schlimmsten.

\section{Auf den Straßen}
\label{straße}

\red{\textbf{Szene}}:

Die Straßen sind verschneit und verlassen. Nur vereinzelt ziehen vermummte Gestalten umher und werfen unserer Gruppe argwöhnische Blicke zu. Ihr scheint hier nicht willkommen zu sein.

\red{\textbf{Information für den Spielleiter}: Sollte einer der Spieler in auffallend feinen Kleidern durch das Armenviertel wandern wird die Gruppe später überfallen. (gehe zu} \blue{ref{kampf2}}\red{).}

Plötlich kommen mehrere verhärmte Kinder unter lautem Geschrei auf euch zu. Sie haben einen unförmigen Lederklumpen dabei und scheinen ganz offensichtlich Fußball zu spielen. Als sie auf eurer Höhe sind rempeln einige der größeren Kinder euch an und murmeln eine Entschuldigung, bevor sie wieder in einer Seitengasse verschwinden.

\red{\textbf{Probe auf Gassenwissen o.ä.}: Alle würfeln eine Probe. Wer diese erfolgreich besteht bemerkt, dass ihm ein Gegenstand geklaut wurde.}

Während die Spieler das bemerken (oder auch nicht!) biegen die Kinder gerade aus ihrem Sichtfeld in eine kleine Seitengasse ein. Die Spieler können die Kinder jedoch noch versuchen mit einer Probe auf Rennen (um 10 erschwert) die Kinder zu verfolgen.

\begin{itemize}
  \item Erfolg: Holen sie die Kinder ein (gehe zu \blue{\ref{eingeholt}}).
  \item Misserfolg: Die Gegenstände sind verloren (gehe zu \blue{\ref{neingeholt}})
\end{itemize}

\subsection{Option 1 - Beim Lumpensammler}
\label{eingeholt}

\red{\textbf{Szene}}:

Ihr verfolgt also die Kinder bis diese die Tür zu einem Laden aufstoßen und darin verschwinden. Wenige Momente später betretet auch ihr außer Atem den Laden. Von den Kindern ist keine Spur zu sehen. Jedoch begrüßt der Lumpensammler Luis (\blue{\ref{Lumpensammler}}) überschwänglich die neue \gqm{Kundschaft}.

Auf Nachfrage stellt sich heraus, dass der Lumpensammler die Gegenstände der Spieler hat, aber nicht ohne weiteres herausgeben will. Nun hat die Gruppe folgende Optionen:

\begin{itemize}
  \item Probe auf Einschüchtern, um 30 erschwert (gehe zu \blue{\ref{fertig}})
  \item Die Gruppe geht wieder (gehe zu \blue{\ref{neingeholt}})
  \item Der Lumpensammler möchte mit den Spielern wetten (gehe zu \blue{\ref{wette}})
\end{itemize}



\subsubsection{Das Rätsel}
\label{wette}

Lumpensammler: \gqm{\textit{Ich hab es! Wir schließen eine Wette ab, ich stelle euch ein Rätsel das es zu lösen gilt. Wenn ihr es löst gebe ich euch eure Sachen wieder, schließlich können so piekfeine Schnösel wie ihr es seid bestimmt gut ein Rätsel lösen?}}

Voller Hochmut und Vorfreude trägt der Lumpensammler das Rätsel vor:

Lumpensammler: \gqm{\textit{Einst wurde ein Pirat gefasst der dem König sein Gold gestohlen hatte. Der König tobte und verlangte, dass der Pirat unverzüglich am Galgen aufgeknüpft werde. Es war aber üblich, den zum Tode verurteilten Dieben eine letzte Chance zu geben und Gott über sie richten zu lassen. Sie mussten aus einem schwarzen Säckchen einen Stein ziehen. Im Säckchen befanden sich immer genau ein weißer und ein schwarzer Stein. Zog der Dieb den weißen Stein, wurde ihm die Freiheit geschenkt. Zog er hingegen den schwarzen Stein, so baumelte er.  Eines Tages kam nun jener Pirat der einst das Gold des Königs geraubt hatte vor den Scharfrichter und wartete auf sein Gottesurteil.

Der König aber wollte sichergehen, dass der Halunke hängt und hat dem Henker am Abend zuvor im Heimlichen befohlen zwei schwarze Steine in das Säckchen zu legen.
So ging der König am nächsten Tage also voller Zuversicht also zum Richtplatz, wo überall weiße und schwarze Steine herumlagen. Als es Zeit wurde für den Verurteilten sein Urteil zu erhalten bückte sich der Henker und nahm zwei schwarze Steine vom Boden auf, die er im Säckchen ablegte. Der Pirat sah dies jedoch und wusste somit, dass er kein gerechtes Urteil erhalten würde. Er glaubte die Schlinge schon um seinen Hals als ihm die rettende Idee kam. \\
Er zog einmal und musste freigelassen werden. \\
Was war es, dass dem Piraten das Leben rettete?}}

Lösung 2:

Er zieht einen der Steine und wirft ihn weg. Da der verbleibende Stein im Sack schwarz ist, muss der gezogene Stein scheinbar weiß gewesen sein.
Erfolg: Beantwortet die Gruppe das Rätsel richtig, bekommen sie ihren Gegenstand zurück.

Misserfolg: Liegen sie falsch, will der Lumpensammler nicht verkaufen. Er genießt den Triumph viel zu sehr! Die entwendeten Gegenstände sind verloren.
Erfolgreiches

Abschliessen der Aufgabe: Lösen sie die Aufgabe, berichtet der Lumpensammler ihnen, dass sie nicht die ersten Schnösel seien, die hier waren. Einer mit ganz komischem Aufzug sei dagewesen. Der habe nach abstrusen Pflanzen gefragt. Chillies und Zitronen seien darunter gewesen. Als ob man sowas hier bekomme. Außerdem habe er noch einen Blasebalg gewollt und etwas grobes Leinen, Lumpen und Teer. Komischer Typ.




Erfolgreiches Abschliessen der Aufgabe: Lösen sie die Aufgabe, berichtet der Lumpensammler ihnen, dass sie nicht die ersten Schnösel seien, die hier waren. Einer mit ganz komischem Aufzug sei dagewesen. Der habe nach abstrusen Pflanzen gefragt. Chillies und Zitronen seien darunter gewesen. Als ob man sowas hier bekomme. Außerdem habe er noch einen Blasebalg gewollt und etwas grobes Leinen, Lumpen und Teer. Komischer Typ.









\subsection{Das Treffen mit Sigrun}
\label{\neingeholt}


Misserfolg: Holen sie die Kinder nicht ein, ist der Gegenstand verloren.
Die Gruppe hat daraufhin eine Begegnung mit Sigrun.


Option 2 – Das Treffen mit Sigrun
\red{\textbf{Szene}}:

Nachdem die Gruppe es nicht schafft die Kinder einzuholen begegnen sie auf der Straße einer Frau namens Sigrun.

Gespräch mit Sigrun: Sie ist gerade auf dem Weg zur St. Petri-Kirche. Dort werde niemand krank, erzählt sie. Es wären unzählige Menschen dort, die dem Ruf von Pater Salus gefolgt seien. Sie sei dort sicher. Allerdings müsse sie erst in Erfahrung bringen, ob auch ihre Kinder dort willkommen seien. Die lägen zu Hause … krank. Sigrun bittet die Gruppe, bei ihr zu Hause vorbeizuschauen und nach den Kindern zu sehen, während sie weg ist.

Moral:

Wie wird die Gruppe mit der Bitte von Sigrun umgehen?

Entscheidung: Die Gruppe muss sich entscheiden.

Sie helfen den Kindern nicht
Sie helfen den Kindern
Option 1 – Sie helfen den Kindern nicht
Ereignis: Sigrun zieht traurig davon, und die Gruppe bleibt auf sich gestellt.


Option 2 – Sie helfen den Kindern
\red{\textbf{Szene}}:

Die Gruppe kommt am Haus an. Es stinkt nach Tod.

Sie alle erhalten sofort +2 Pestilenz.

Die Kinder weinen und klagen und sind kaum noch bei Verstand.

Kümmern sie sich weiter um die Kinder, erhalten sie nochmals Pestilenz +2.

Ereignis: Plötzlich klopft es an der Tür.

Gespräch mit Hagen: Hagen, der Nachbar, steht vor der Tür und fragt die Gruppe, was sie hier suchten und wo Sigrun sei. Nach einer Erklärung lädt er sie zu sich auf Tee und einen Plausch ein. Er habe ihnen etwas Spannendes zu erzählen!

Bei Hagen vom Fleet
\red{\textbf{Szene}}:

Hagen begrüßt die Gruppe zunächst sehr freundlich.

\textbf{Raumbeschreibung}: Sein Haus ist aufgeräumt. Nirgendwo liegen Waren oder ähnliches herum.

Gespräch mit Hagen:

Anmerkung: Hat Hagen die Gruppe nicht zu sich eingeladen (falls die Gruppe ihn von selbst aufsucht), dann verlangt er im Gegenzug für Informationen einen Aufseherposten innerhalb der Hanse. Die Gruppe kann zustimmen oder ablehnen. Hauptsache sie kommen an die Infos.

Kommen sie von der Nachbarin und werden von ihm eingeladen, dann erzählt er von sich aus.
Hagen berichtet, dass er seit geraumer Zeit heimlich Waren abzweige. Er sei nicht stolz darauf, aber man müsse eben sehen, wo man bleibt. Jedenfalls komme er nachts am Nikolaifleet an diese Waren. Dort arbeite er. Als immer mehr seiner Kollegen krank wurden – das startete bereits vor vier Wochen, also vor allen anderen Ausbrüchen – wurde das sogar noch leichter. Eines Nachts jedenfalls schlich er sich wieder ins Lager, als ihm ein eigenartiger Mann begegnete. Er trug eine lange Maske, die beinahe wie der Schnabel eines Vogels aussah, ein weites Gewand und einen Stock bei sich. Außerdem roch es nach … Parfüm. Der Mann floh, als er Hagen sah, und ließ nichts außer einer Rattenfalle zurück. Diese war aber leer.

Anmerkung: Hagen will sie auf keinen Fall begleiten. Das ist ihm zu gefährlich. Auch mit Gewalt oder Drohungen kann man ihn nicht überzeugen!

Das Lagerhaus von Didrich
\red{\textbf{Szene}}:

Das Lagerhaus, welches Ruth beschrieben hatte, findet die Gruppe recht nahe der Stadtmauer, gerade vor den Toren der wohlhabenden Stadt.

Dennoch sieht es hier erbärmlich aus. Als sie am Lagerhaus ankommen ist dieses verschlossen. Drinnen brennt jedoch Licht.

Interaktionen:

Einbruch:

Es gibt verschiedene Möglichkeiten einzudringen.
Ereignis: Wenn sie einbrechen, tauchen aus der Dunkelheit zwei kräftige Männer auf, die sehr skeptisch sind. Ablenkung ist gefragt!


Im Inneren
\red{\textbf{Szene}}:

\textbf{Raumbeschreibung}: Im Lagerhaus ist es eiskalt. Allerdings brennt eine riesige Öllampe, die auf einem Tisch steht, auf dem allerlei Dinge liegen.

Darunter liegt eine tote Ratte, eine Zitronenschale und etwas von einem roten, scharfschmeckenden Pulver (Chili).

Interaktionen:

Türen öffnen:

Es gibt drei Türen zu kleineren Räumen. Bei allen sind jegliche Ritze und Schlitze säuberlich mit Lumpen und Lappen verstopft und verteert. Eine Ritze ist offen.
Öffnen sie diese Tür, entdecken sie schlimm zugerichtete Leichen von Pestkranken.

Alle im Haus bekommen Pestilenz +2.

Eine Tür ist offen. Dahinter finden sie üppige Kornvorräte. Auf den Säcken steht die Adresse eines Lagerhauses am Nikolaifleet geschrieben.
Öffnen sie die dritte Tür, strömen Ratten heraus. Sie sind ausgehungert und aggressiv. Sie greifen die Gruppe an!

Kampf:

Rattenschwarm:
Leben: 10 Ratten mit jeweils 10 Lebenspunkten
Waffe: Biss (40)
Schaden: Jeder Angriff wird mit einem W10 geworfen. Das Ergebnis sind die Ratten, die erfolgreich angreifen. Jeder Biss macht dabei 4 Schaden.
Ausweichen 10

Sind die Ratten besiegt, finden sie auch hier Säcke mit der Adresse eines Lagerhauses am Nikolaifleet.

Ende des relevanten Hammerbrook-Plots.
