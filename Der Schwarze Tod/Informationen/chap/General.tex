%!TEX root = ../main.tex

\textbf{Wo?}            -  mittelalterliches Hamburg \\
\textbf{Wann?}          -  1350 n. Chr. \\
\textbf{Spielerzahl?}   -  3 - 5 \\
\textbf{Schwierigkeit?} -  einfach - mittel \\
\textbf{Spieldauer?}    -  3-4 Stunden

\section{Anmerkungen für Spielleiter}

Die Formatierung ist nicht zufällig gewählt. Im Verlauf des Abenteuers werden spezielle Informationen wie folgt verdeutlicht:

\begin{itemize}
  \item \textit{Kursive Texte}
  Alles, was \textit{kursiv} geschrieben ist, kann wörtlich vorgetragen werden. Dabei handelt es sich meistens um die Einleitungen der einzelnen Abschnitten oder um wörtliche Rede in Gesprächen.

  \item \textbf{Raumbeschreibungen/Ortsbeschreibungen}
  Diese Beschreibungen verweisen auf die Einrichtung eines Raums oder die Beschaffenheit eines Ortes und sind \textbf{fett} gedruckt. Raumbeschreibungen beschreiben meistens alles, was innerhalb eines Gebäudes zu sehen ist, Ortsbeschreibungen hingegen beschreiben, was draußen ist.

  \item \red{\textbf{Szenen und Interaktionen}}
  Die Abschnitte sind in Szenen und Interaktionen unterteilt. Damit es einfacher ist, dorthin zu navigieren, sind diese \red{rot} markiert. Szenen geben eine Handlung vor, die sich den Spielern offenbart, wenn sie sich in einem Abschnitt befinden. Interaktionen ermöglichen optionale Handlungsstränge, die den Spielern entgehen können, wenn sie nicht die entsprechenden Aktionen durchführen oder sich für die entsprechende Option entscheiden.

  \item \green{\textbf{Moral}}
  Alle Stellen, an denen die Spieler mit moralischen Fragen konfrontiert werden, sind \green{grün} gekennzeichnet. Diese markieren Entscheidungen, die sich auf den Ausgang der Geschichte auswirken können.

  \item \purple{\textbf{Pestilenz}}
  Alle Stellen, an denen die Spieler in Kontakt mit der Pest kommen und gegebenenfalls Pestilenz anhäufen können, sind \purple{lila} markiert.
\end{itemize}
