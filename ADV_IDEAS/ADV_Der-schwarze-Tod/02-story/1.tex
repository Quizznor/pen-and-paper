%!TEX root = ../adventure.tex

\chapter{Prolog: Das Gespräch mit dem Rat}
\label{adventure}

\begin{advquote}
\large Wir schreiben das Jahr 1350. Hamburg wächst dank des erstarkenden Seehandels
stetig und die Hanse trägt ihren Teil dazu bei. Täglich gehen am Rheinhafen Schiffe
aus aller Herren Länder vor Anker, während vom Rathaus an der Troßtbrücke der Rat die
Geschicke der Stadt lenkt. Es ist eine Zeit des Aufbruchs, aber auch eine Zeit der
Angst. Denn neben Piraten und anderen finsteren Gestalten, die zunehmend in den
Gassen der Stadt umherstreifen, greift etwas noch viel gefährlicheres um sich. Die
Leute hatten bereits von einem „Schwarzen Tod" gehört, der in anderen Städten gewütet
haben soll, und nun scheint es so, als sei die Plage auch in Hamburgs Gassen
eingekehrt. Der düstere Geruch des Todes zieht durch die Docks und Armenviertel,
während der hohe Rat in der Hansestadt darüber entscheidet, was zu tun ist.

Und was auch immer das sein mag, es muss schnell geschehen.
\end{advquote}

\section{Das Wartezimmer}
\label{sec:wartezimmer}

Ihr seit also aus ebendiesem Grund vom Hamburger Rat zu sich gerufen worden. Was
genau von euch erwartet wird um dem schwarzen Tod vorzubeugen wird sich alsbald
zeigen. So begibt es sich, dass ihr geduldig vor einer gewaltigen Holztür im Ratshaus
an der Troßtbrücke sitzt. Hinter der Tür tagt im Moment der Hamburger Rat, zu den
man euch nun in wenigen Augenblicken rufen wird. Ob ihr euch bereits kennt dürft ihr
entscheiden und ausspielen wie ihr möchtet. Noch besteht jedoch ein wenig Zeit sich
zu unterhalten und umzuschauen. Was tut ihr also?

\begin{place-box}{Wartezimmer}

Ihr befindet euch in einer Art Warteraum, der für damalige Verhältnisse sehr üppig
eingerichtet ist. Man erkennt deutlich, dass Hamburgs Reichtum hier sein Übriges
getan hat. An den Holzwänden hängen mehrere Bilder von großen Hamburger
Persönlichkeiten. Außerdem stehen kleine Teller mit allerlei Obst und Brot bereit,
und auch Getränke werden angeboten. In den Ecken des Raumes verteilt stehen
Ratsdiener die für die Erfrischungen sorgen. An der Wand hängt, und das ist für euch
vermutlich am interessantesten, eine Karte der Stadt (\then \page{fig:map})

\end{place-box}

\section{Das Gespräch mit dem Rat}
\label{sec:ratsgespräch}

Als Ihr euch also unterhaltet öffnet sich plötzlich die Holztür und ein junger
Ratsdiener bittet euch vor den Rat zu treten.

\begin{place-box}{Hamburger Rat}

Ihr tretet in einen großen Raum ein, der von einer U-förmigen Tischreihe dominiert
wird an dem 18 Männer in erhöhter Position sitzen. Auch wenn der Raum sonst nur
spärlich eingerichtet ist, ist offensichtlich dass sich hier sonst niemand aufhält,
der wenig Geld hat. Aus den Wänden sind kunstvoll Löwenköpfe und andere Muster
herausgearbeitet. Eine Wand ist von hohen Fenstern gesäumt, die den Raum hell
erleuchten. Im Hintergrund huschen Ratsdiener mit Papieren umher oder gehen anderen
Aufgaben nach.

\end{place-box}

\begin{info-box}

Die 18 Männer setzen sich aus neun Rechtskundigen, sieben Kaufleuten und zwei
Vertretern der Kirche zusammen. Je nach Gesprächsverlauf können diese zu Wort kommen.

\end{info-box}

Nachdem ihr von den Ratsherren nähergewunken werdet spricht der Vorsitzende.

\begin{say-box}{Ratsvorsitzender}

Der Hamburger Rat dankt euch für euer Erscheinen. Bevor wir uns jedoch mit den euch
betrauten Aufträgen befassen werden, sagt, was wisst ihr über die Pestilenz?

\end{say-box}

\begin{probe-box}{Medizin}{10}

Menschen die von der Pestilenz befallen sind leiden unter Schwindelgefühlen und
Schüttelfrost. Auch klagen manche über starke Kopfschmerzen und erbrechen sich
häufig. Ein hohes Fieber sowie schwarze Pestbeulen an den Lymphknoten treten im
Verlaufe der Krankheit auf und sind für den Patienten meist ein sicheres
Todeszeichen.

\end{probe-box}

\begin{say-box}{Ratsvorsitzender}

Nun ist dem Rat zu Ohren gekommen, dass auch vor den Mauern Hamburgs die Pestilenz
ihr Unwesen treibt. Diese Krankheit muss unter allen Umständen aus der Stadt
verdrängt werden. Zu diesem Zwecke werdet ihr in einen Beirat gerufen. Der Rat
fordert euch auf in den nächsten vier Tagen die Fälle von Pestilenz zu untersuchen
und dem Rat am fünften Tag eine Empfehlung auszusprechen, wie diese Pest zu
bewältigen ist. Euch wird während eurer Zeit im Beirat eine Kapelle nahe der St.
Michaelis Kirche zur Verfügung gestellt, die fortan als eure Operationszentrale
dienen wird, alle weiteren Informationen wird man dort für euch bereitlegen.

\end{say-box}

Anschließend wendet sich der Vorsitzende des Rates seinen Aufzechnungen zu. Unter
den anderen Ratsmitgliedern keimen Gespräche auf, an euch besteht also für den Rat
keine weitere Interesse. Von der Seite her eilt ein Ratsdienern herbei der euch
bittet ihm zu folgen.

Ihr folg also dem Bediensteten aus dem Ratshaus und macht euch auf zum Sitz des
Beirats. Ihr spürt auf dem Weg dorthin deutlich, dass in der unmittelbaren Umgebung
des Ratshauses der Aufstieg Hamburgs als Handelsmetropole seine Folgen hinterlassen
hat. Ihr schlendert durch breite Straßen die von hohen Häusern gesäumt werden.
Mittlerweile dämmert auch schon ein bisschen der Abend herein, als ihr unweit der
St. Michaelis Kirche vor eine Kapelle tretet, die euch der Ratsdiener als eure
Operationszentrale vorstellt. Er drückt euch einen Türschlüssel in die Hand und
wendet sich dann zum gehen um.

\section{Die Kapelle}
\label{sec:kapelle}

\begin{place-box}{Kapelle}

Es handelt sich um ein durchaus anschauliches Gebäude, dass erst kürzlich einen
neuen Anstrich mit weißer Farbe erhalten hat. Im Inneren stehen allerhand Tische
und Stühle herum, einige Kommoden und Schränke hier und da. Außerdem gibt es
Schlafmöglichkeiten und so ziemlich alles, was man sonst so zum Leben gebrauchen
kann. Selbst vorbereitetes Abendessen steht für euch zum Verzehr bereit.

\end{place-box}

\begin{info-box}

Es ist am naheliegendsten für die Spieler ihr Abendmahl einzunehmen, sollte sich
einer der Spieler umsehen wollen könnten sie durchaus Essensrationen für den
folgenden Tag oder das ein oder andere Messer finden.

\end{info-box}

Als ihr euch gerade zum Abendessen setzen wollt stört ein lautes Klopfen an der Türe
die Ruhe.

\begin{ref-box}{Wer klopft an die Türe?}

Würfel 01 bis 33: Das Militär \then Gehe zu \page{ssec:militär} \\
Würfel 34 bis 66: Der Tod \then Gehe zu \page{ssec:tod} \\
Würfel 67 bis 99: Die Familie \then Gehe zu \page{ssec:familie}
\end{ref-box}

\subsection{Das Militär}
\label{ssec:militär}

Durch eine Luke in der Tür könnt ihr einen älteren Mann erspähen. Er ist in einen
militärischen Mantel gehüllt, der vor lauter Orden schlaff von seiner breiten
Brust herunter hängt. Seine breite Nase ist von einer Narbe entstellt. Generell
macht eher einen sehr befehlsgewohnten und grobschlächtigen Eindruck. Außerdem ist
er in Begleitung zweier Soldaten und guckt euch finster durch die Luke an. Mit
lauter Stimme richtet er das Wort an euch.

\begin{say-box}{General Brügge}

Nun öffnet mir die Tür in Namen des Herrgott! Wisst Ihr denn nicht wer ich bin?

\end{say-box}

Nach dem der Mann eintritt stellt er sich den Spielern als \charakter{Bruegge} vor.
Er berichtet ihnen in vehementem Ton, dass die ganze Krise ein Werk der Dithmarscher
sei. Diese hätten sich jahrelang an Hamburgs Handelsschiffen gütlich getan. Nun, da
es einen Vertrag gibt, der das verhindert, versuchen einige von ihnen die Stadt zu
schwächen, um davon zu profitieren oder sie gar ganz an sich zu reißen. Die
Dithmarscher operieren von ihrem Versteck aus, das sich in einer Hafenkaschemme
namens „Beim Gelockten Hund“ befinden soll. Ein gewisser \charakter{Gorich}
leite das Ganze. Dort sollten sie mit ihren Recherchen beginnen.

\begin{probe-box}{Menschenkenntnis}{}

\charakter{Bruegge} sagt die Wahrheit oder zumindest glaubt er das, aber seine
Sichtweise ist durch den langjährigen Militärdienst möglicherweise verzerrt.

\end{probe-box}

\subsection{Der Tod}
\label{ssec:tod}

Durch eine Luke in der Tür ist eine gebeugte Gestalt zu sehen. Ihr Gesicht ist
von einer Kapuze verhüllt. Mit schnarrender Stimme fragt die Person euch

\begin{say-box}{Hanno}

Habt ihr Tote im Hause?

\end{say-box}

Durch entsprechende Fragen lässt sich sehr leicht herausfinden, dass es im
Armenviertel am Schlimmsten sei. Die Leichen könne er kaum mehr entsorgen. Man
müsse kreativ werden...

\begin{info-box}

Gegen Bestechung (oder eine passende Probe) verrät \charakter{Hanno}, dass er
jemandem Leichen verkaufe. Dazu müsse er sie allerdings recht weit fortbringen,
nämlich in einen kleinen Ort namens Eeksdurf vor den Toren der Stadt. Dort hinterlege
er die leblosen Körper in einem Lagerhaus, wo bereits seine Bezahlung auf ihn
warte. Die Absprache habe er dereinst mit einer jungen rothaarigen Frau getroffen,
die ihn angesprochen habe, nachdem sie ihn beim Abholen von Leichen im Dorf sah...

\end{info-box}

\subsection{Die Familie}
\label{ssec:familie}

Eine vermummte Frau steht vor der Tür, ein kleines Kind schaut schüchtern aus ihrem
Rock zu euch empor. Die Frau erzählt, dass sie aus dem Armenviertel Hammerbrook
kämen und auf der Suche nach der St. Petri-Kirche seien. Dort befindet sich ein
Zufluchtsort für gesunde und sündenfreie Menschen. Niemand werde dort krank! Für sie
ist es zu spät, hustet die Frau, aber ihr kleines Kind, das sei noch zu retten.

Auf Nachfrage erzählt die Frau, dass sie das alles von einem Mann wisse, der im
Armenviertel nach den Leuten sehe. Er werde nicht krank, egal was er tue...
Er habe sie losgeschickt. Sie wüssten gern den Weg.

\begin{info-box}

Öffnen die Spieler der Frau die Türe erhält derjenige der sie geöffnet hat
\Pestilenz{+1}

Die Gruppe kann eine Beschreibung von \charakter{Didrich} erhalten. Außerdem gibt
die Frau Ihnen auf Nachfrage den Tipp, einmal beim Lumpensammler im Armenviertel
vorbeizuschauen.

\end{info-box}

\subsection{Wie gehts weiter?}
\label{ssec:kapelle-ende}

Nachdem der unerwartete Besuch wieder gegangen ist setzt ihr euch an den Tisch
und esst euer Abendessen. Das geht recht unspektakulär über die Bühne. Anschließend
legt ihr euch schlafen um am folgenden Tag frisch ans Werk zu gehen.

\begin{info-box}

Ab hier können die Spieler frei entscheiden, wohin sie gehen wollen! Nach Ablauf
der Frist von vier Tagen müssen sie beim Hamburger Rat vorsprechen. Bis dahin
müssen sie sich auf eine Handlungsempfehlung festgelegt haben!

\end{info-box}

\begin{ref-box}{Wie gehts weiter?}

Die Gruppe geht zum Hafen \then Gehe zu \page{chap:hafen} \\
Eine Reise nach Eeksdurf \then Gehe zu \page{chap:eeksdurf} \\
Besuch im Armenviertel \then Gehe zu \page{chap:hammerbrook} \\
Die Kirche St. Petri solls sein \then Gehe zu \page{chap:stpetri} \\
Abstecher zum Nikolaifleet \then Gehe zu \page{chap:nikolaifleet}

\end{ref-box}
