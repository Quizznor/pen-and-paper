%!TEX root = adventure.tex

% Don't forget to add the corresponding glossaries at adventure_config.tex !

\longnewglossaryentry{Beispielcharakter}{%
type={Beispiel},
name={Max Mustermann},}{%
ist ein Beispielcharakter. Er hat eine Musterfrau die er sehr liebt und drei
Musterkinder. Er lebt mit seiner Musterfamilie in einem Musterhaus in Musterstadt.

\begingroup
\renewcommand{\arraystretch}{1.3}
\begin{char-box}{Max Mustermann}


\centering
\begin{tabular*}{\textwidth}{@{\extracolsep{\fill}} p{0.19\textwidth}c|p{0.19\textwidth}c|p{0.19\textwidth}c}

\textbf{Handeln} & \raisebox{-1pt}{\includegraphics[width=0.08\textwidth, height=6mm]{01-img/strength.png}} &
\textbf{Wissen} & \raisebox{-1pt}{\includegraphics[width=0.08\textwidth, height=6mm]{01-img/knowledge.png}} &
\textbf{Soziales} & \raisebox{-1pt}{\includegraphics[width=0.08\textwidth, height=6mm]{01-img/social.png}} \\

% Handeln        & P   & Wissen   & P  & Soziales   & P
Spitze anstiften & 100 & Deutsch  & 80 & Langweilen & 1000 \\
Pflaster machen  & 20  & Englisch & 40 &            & \\
& & Excel & 30 & & \\

\hline
Talentwert & \textbf{12} & Talentwert & \textbf{15} & Talentwert & \textbf{100}

\end{tabular*}

\end{char-box}
\endgroup
}

\longnewglossaryentry{Bruegge}{%
type={Personen},
name={General zur Brügge},}{%
, ein um die 50 Jahre alter, dickbäuchiger und hochdekorierter Mann mit grauem und
gut gepflegtem Bart. Der General nimmt alles sehr genau, nicht nur den perfekten
Sitz seiner Uniform. Er mag hitzig sein, ist aber keineswegs dumm. Seine Karriere
begann in der Schlacht um die Handelsrouten Hamburgs mit den Piraten der Region.
Er konnte trotz der neuen Verträge den Frieden nie akzeptieren.

Mit einer gelungenen Probe auf Gesellschaft kann herausgefunden werden, dass sein
Beiname "Der Kürzermacher" lautet. Er wurde ihm aufgrund seines Einsatzes gegen
die Piraten und seiner Vorliebe, diesen als Bestrafung Körperteile zu „entfernen“
gegeben.}

\longnewglossaryentry{Gorich}{%
type={Personen},
name={Gorich, der Dithmarscher},}{%
unterscheidet sich äußerlich nicht von \charakter{Gert}. Gorich weiß sehr genau,
wann er sich bedeckt halten muss und wann er zur Tat schreiten kann, ohne Gefahr zu
laufen gefangen genommen zu werden. Gesucht und gefürchtet von der Hamburger Hanse,
ist Gorich gerissen genug, um noch im Untergrund seinen Profit herauszuschlagen.
Dennoch ist Profit nicht sein einziger Antrieb. Seine Familie liegt ihm am Herzen
und für seine Kinder würde Gorich fast alles tun.}

\longnewglossaryentry{Gert}{%
type={Personen},
name={Gert der Wirt},}{%
Ein kräftiger Mann, der keine Scherereien in seiner Schenke duldet, außer er
verursacht sie selbst. Meist hält er sich im Hintergrund und gibt eine Suppe aus,
deren Zutaten niemand kennt – oder kennen will. Gert ist schweigsam und redet nur,
wenn er etwas Wichtiges zu sagen hat. Seine Stammgäste scheinen eine Ehrfurcht vor
ihm zu haben, die sich der sporadische Besucher nicht recht erklären kann. vielleicht
liegt es daran, dass Gert den „Gelockten Hund“ führt wie ein Kapitän sein Schiff.}

\longnewglossaryentry{Hanno}{%
type={Personen},
name={Hanno},}{%
ist ein buckliger, zynischer aber freundlicher alter Mann. Zumindest sollte man
vermuten, dass er alt ist. So richtig sicher kann man das nicht sagen. Seine Kleidung
besteht größtenteils aus Flicken. Er ist gebückt vielleicht 1,40 Meter groß. Da er
immer gebückt und unterwürfig läuft, weiß niemand, wie groß er eigentlich wäre. Er hat
eine kratzige Stimme und weiß seine Informationen zu Geld zu machen.}

\longnewglossaryentry{Didrich}{%
type={Personen},
name={Didrich von Sinnfeld},}{%
strahlt eine unangenehme Aura aus. Seine dünne Nase ist spitz zulaufend. Wache,
umtriebige Augen schauen aus tiefen Höhlen. Er ist spindeldürr und trägt sein
braunes Haar kraus in Locken auf dem Kopf. Er ist gebildet und nimmt nichts für
gegeben, er überprüft Theorien lieber selbst. Er wirkt herablassend und gekünstelt,
weil er sich seiner Intelligenz bewusst ist. Bei seiner Forschung wandelt er nicht
nur auf dem Grat zur Morallosigkeit, er überschreitet diesen schmalen Grat auch
bewusst, wo es seiner Meinung nach nötig ist. Sein Stand und Vermögen ermöglichen
ihm einen Lebensstil, in dem er sich mit Rätseln der Natur und den Belangen der Welt
auseinandersetzen kann. Auf seinen Reisen, aber auch durch sein Geschäft lernt er oft
allerlei Neues kennen, das man so in Hamburg noch nicht kennt.}
