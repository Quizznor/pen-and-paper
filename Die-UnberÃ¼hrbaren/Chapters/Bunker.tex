%!TEX root = ../main.tex

Entschließen sich die Spieler Trekov Richtung Osten zu verlassen treffen sie nach einem längeren Fußmarsch durch den Wald (unabhängig von der in Trekov verbrachten Zeit beginnt es zu dämmern) auf einen 3 Meter hohen Stacheldrahtzaun.
In der Mitte des Geländes steht ein leeres Wärterhäuschen und daneben eine in den Zaun eingelassene Türe. Das Häuschen ist abgeschlossen, im Inneren befinden sich Batterien die auf einem Schreibtisch liegen.
In einiger Entfernung ragt hinter dem Zaun eine massive Felswand hervor. Es scheint eine Betonröhre in die Felswand hineinzuführen.

Der Zaun steht unter Strom. Versuchen die Spieler den Zaun zu berühren erleiden sie einen Stromschlag und Schaden von -10 HP, der Stromschlag kann durch Gummihandschuhe verhindert werden.
Um Zutritt zu dem Gelände zu erhalten können die Spieler die Scheibe des Wärterhäuschens mit einer Kraftprobe (Malus -30) zerstören oder mit einem Bolzenschneider ein Loch in den Zaun schneiden, jedoch können dann im weiteren Verlauf des Abenteuers Untote auf das Gelände gelangen.
\\Alternativ kann die Gruppe an nahegelegenen Bäumen hinaufklettern und über das Dach des Wärterhäuschens den Zaun passieren.

\section{Hof}

Abgesehen von einem gepflasterten Weg, der von der Zauntüre in den Tunnel führt ist der Hof leer. Auf dem Weg in den Tunnel tritt einer der Spieler auf dem Schlüssel zum Wärterhäuschen. Des weiteren wird offensichtlich, dass in der Tunnelöffnung ein Falltor ist, das momentan offensteht.

Im Inneren der Betonröhre ist es dunkel, vor ihnen ertönt ein merkwürdiges Kratzgeräusch. Haben die Spieler eine Lichtquelle so erkennen sie, dass ein Untoter an einer massiven Stahltür kratzt.
Der Zombie wird im Verlauf des Kampfes probieren zu kreischen, gelingt ihm dies werden 1W6/2 Zombies angelockt, die je nach früheren Entscheidungen auf das Gelände gelangen und der Gruppe in den Rücken fallen könnten.


%\begin{center}
%  \input{./tab/Tunnel_Zombie.tex}
%\end{center}

Neben der Stahltür ist in den Stein ein Terminal eingelassen. Die Tastaturbeleuchtung verrät, dass es über Strom verfügt und scheinbar noch funktionstüchtig ist.
Eine erfolgreiche Probe auf Auffassungsgabe o.ä offenbart, dass oberhalb der Türe ein rotes Licht in regelmäßigen Abständen aus der Dunkelheit blinkt.
Im Tunnel hat die Gruppe folgende Optionen:

\begin{itemize}
  \item Wahllos Codes eingeben
  \\Nach einer falschen Eingabe leuchtet das Display rot auf und es passiert nichts. Sollten drei flasche Codes in Folge eingegeben werden schließt sich das vordere Tor.
  Wenn Spieler nicht augenblicklich nach draußen stürmen (Probe auf Rennen, um -20 erschwert) werden sie im Tunnel eingeschlossen.
  Hat sich das Tor geschlossen kehrt kurz Stille ein gefolgt von einem Zischen, das nicht weiter ausfindig zu machen ist.
  Hierbei handelt es sich um Lachgas, dass die Spieler ausknockt. Am nächsten Morgen wachen sie im Reinraum des Bunkers auf.

  \item richtiger Code wird eingegeben
  \\Aus purem Zufall kann die Kombination 1701 erraten werden. Daraufhin öffnet sich die Türe und die Spieler können eine Treppe hinuntersteigen um auf einem langen Flur anzukommen. Hier wird ihnen ein Wissenschaftler über den Weg laufen, der augenblicklich fliehen wird. Kurze Zeit später taucht ein Trupp aus drei schwerbefaffneten Söldnern auf.

  \item Vor der Kamera gestikulieren / Den Tunnel verlassen
  \\Plötlich ertönt eine Frauenstimme, die von den Felswänden widerhallt und nach der Intention der Abenteurer fragt. Überzeugen die Spieler die Stimme wird sich die Tür öffnen und Söldner sie auffordern ihre Waffen abzugeben. Daraufhin werden sie zum Büro der Direktorin geleitet.
\end{itemize}
