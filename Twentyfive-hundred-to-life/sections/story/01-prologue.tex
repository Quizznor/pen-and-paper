%!TEX root = ../../adventure.tex

\chapter[Prolog]{Eine kurze Geschichte der Menschheit}

\begin{say}
Komprimiert man die Geschichte der Erde - insgesamt knapp 5 Millarden Jahre - auf nur einen einzigen Tag, so entwickeln sich Mikroben und Einzeller gegen
18 Uhr. Die Gattung Homo tritt etwa um 23:59 Uhr auf den Tagesplan. Der anatomisch moderne Mensch etwickelt sich vier Sekunden vor Mitternacht. In diesen
umgerechnet dreihunderttausend Jahren kann der Homo sapiens sapiens auf eine beachtliche Entwicklung zurückblicken, die ihresgleichen sucht. Vom Affen
abstammend entdeckte der Homo Sapiens das Feuer, Werkzeuge und die Vorteile der Kommunikation und Kooperation. \todo{continue}

Wir schreiben das Jahr 2500 n.Chr.

Fortschritte in der Medizin strecken die natürliche Lebensspanne eines jeden, der es sich leisten kann bis deutlich über 100 Jahre. Über 30 Millarden
Menschen leben mittlerweile auf der Erde. Die genaue Zahl kennt eigentlich niemand. Höchstens die weltumspannenden Ultrakonzerne haben noch einen groben
Überblick darüber, was vor sich geht. Fest steht jedenfalls, die Erde ist überbevölkert. Zwar gibt es florierende Kolonien auf dem Mond und Mars, doch auch
diese stoßen mittlerweile an ihre Populationsgrenzen. Die einzige Lösung scheint das Erschließen neuer Sternensysteme.

Zu diesem Zweck hat \charakter{EnVi} Euch rekrutiert. Als Pioniere der Menschheit sollt Ihr potentiell bewohnbare Planeten außerhalb des Sonnensystems
erkunden und neue Lebensräume erschließen. Nach einem mehrjährigen Bewerbungsprozess habt ihr euch also für den Dienst qualifiziert und befindet euch an
Bord eines Shuttles, dass euch in eine niedrige Umlaufbahn und zu eurem Raumschiff, der ISV New Horizons bringen wird.
\end{say}
