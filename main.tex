\documentclass{include/protokollclass}

\usepackage{pdfpages}
%\usepackage{graphicx}
\usepackage{float}
\usepackage{amsmath}
\usepackage{booktabs}
\usepackage{siunitx}
\usepackage[ngerman]{babel}

%% ---------------------------------
%% |    Commands for neat stuff    |
%% ---------------------------------

\newcommand{\gqm}[1]{\glqq#1\grqq}

\title{Die Unberührbaren}
\author{Shorty\thanks{Verantwortlich für komplette Rahmenhandlung}, Quizznor\thanks{hauptsächlich Feinschliff}}

%% --------------------
%% |    Main Story    |
%% --------------------

\begin{document}

\maketitle
\tableofcontents

\chapter{Informationen}

\section{Allgemeines}
\textbf{Wo?}            - Russland, in der Nähe des südlichen Urals \\
\textbf{Wann?}          - Jetztzeit, ein Jahr nach einer Zombieapokalypse \\
\textbf{Spielerzahl?}   - 3-5 Spieler \\
\textbf{Schwierigkeit?} - Anspruchsvoll \\
\textbf{Spieldauer?}    - 3/4 Stunden

\section{Fraktionen}
\begin{itemize}
  \item Die Sentinels

Nur wenig ist über die Sentinels bekannt da sie sehr isoliert leben.\\Angeblich besteht die Gruppe aus etwa 50 Wissenschaftlern und Wissenschaftlerinnen die in einer alten Bunkeranlage bei Trekov ein neues Zuhause gefunden haben. Sie sehen sich selbst als die Bewahrer des Wissens und forschen energisch an einem Heilmittel gegen das Virus.

  \item Die Unberührbaren

Die Unberührbaren sind eine Gruppe von 5 Männern und Frauen, die angeblich immun gegen das Virus sind. Fast schon wie Götter werden sie von einigen verehrt.
\\Sie interagieren kaum mit der Außenwelt, nicht zuletzt deswegen, weil sie \gqm{Berührbare} für Abschaum halten. Ihr Anführer ist Bruder Matthäus Benedict, der erste Unberührbare, der den Biss eines Infizierten überlebte.
\\Noch immer ziert eine Narbe seine Schulter die davon zeugt. Schenkt man seinem Wort Glauben sorgen Antikörper in seinem Blut für die Immunität.
\\In unregelmäßigen Abständen gibt es die Prüfung der Offenbarung. Ein Auserwählter erhält eine Bluttransfusion von Bruder Benedict. Nach der Zeit des Erwachens wird er dem Biss eines Untoten ausgesetzt. Überlebt er diesen gilt er als einer der Unberührbahren und wird in deren Kreise aufgenommen.

  \item Pilger

Pilger sind fanatische Anhänger der Unberührbaren. Sie leben in der Nähe der Zitadelle oder pilgern aus den umliegenden Siedlungen regelmäßig herbei um Events wie z.B. der Prüfung der Offenbarung beizuwohnen.

  \item Juggernauts

Angeführt von \gqm{Jacob} bestehen die Juggernaut aus Exmilitär und ehemaligen Söldnern. Jeder von ihnen hat auf den ein oder anderen Weg eine tüchtige Kampfausbildung genossen. Die Juggernaut finanzieren ihre durchaus ansehnliche Ausrüstung durch Schutzgelder der umliegenden Siedlungen.

  \item Untote

Bei Untoten handelt es sich um Menschen, die von einem nicht näher bekannten Virus infiziert wurden. Sie sind ausgesprochen aggresiv gegenüber anderen Menschen und werden versuchen diese bei Sichtkontakt anzugreifen. Ihre taktische Kampfexpertise beschränkt sich auf \gqm{Beißen} und \gqm{Schlagen}.

\begin{itemize}
  \item Beißen
  \\Der Schadenbasiswert (SBW) von "Beißen" beträgt 20. Wird ein Charakter gebissen würfle 2W10. Beträgt die Summe der Würfel mindestens 18 so verwandelt sich der Charakter innerhalb eines Tages in einen Untoten.
  \\Symptome der Verwandlung sind Blässe, Kopfschmerzen, Übelkeit, Schweißausbrüche, Verwirrtheit, Bewusstlosigkeit und Wutausbrüche.
  \item Schlagen
  \\Ganz normale Fausattacke. SBW 10

  Wenn nicht näher spezifiziert besitzen Untote 80 HP.

\end{itemize}
\end{itemize}
\newpage
\section{Charaktere}

% \xtable{h}{test}{}{}

\chapter{Geschichte}

\section{Prolog}

Niemand kann genau sagen wie es passierte.
\\Die die es vielleicht einst vermochten sind längst tot.
\\Gab es einen Patient Zero? War es eine menschengemachte Katastrophe?
\\Oder, wie andere behaupteten Gottes Zorn der dank Drogen, Gewalt, Korruption und Blasphemie über uns kam?
\\Prediger und andere Fanatiker schrien ja bekanntlich seit Jahrhunderten das Ende der Welt herbei, doch selbst sie ahnten nicht, dass das Weihnachtsfest 2018 alles verändern sollte.
\\In den Krankenhäusern fing es an.
Patienten klagten über starke Kopfschmerzen und Übelkeit, starben in der Nacht, und wurden am folgenden Morgen dabei gesehen, wie sie wehrloses Krankenhauspersonal in dunkle Gänge zerrten.
\\Schnell waren die Innenstädte übersät mit Totgeglaubten.
\\Überrascht im weihnachtlichen Einkaufsbummel leisteten Einwohner kaum Gegenwehr und fielen den wachsenden Horden zum Opfer.
\\Viele Städte wurden evakuiert und der Notstand ausgerufen. Das Militär trat auf den Plan um Kraftwerke, Quarantänezonen und Evakuierungsrouten zu schützen.
\\Erst mehrere Tage nach Ausbruch der Epidemie sollte das eigentliche Außmaß der Katastrophe offensichtlich werden. Viele Quarantänezonen wurden überrannt von Untoten. Selbst das Militär zog sich von Knotenpunkten zurück, und so brach die Kommunikation endgültig zusammen. Letzte Fetzen von Berichterstattungen erzählten von Infizierten die in den Regierungsbezirken ihr Unwesen trieben.
\\Die ersten schwer befestigten Siedlungen gründeten sich um die einstigen Metropolen. Angeführt von dubiosen Gestalten galt hier das Recht des Stärkeren.
\\Optimistische Schätzungen gingen davon aus, dass etwa 3\% der europäischen Bevölkerung Schutz in diesen Refugien gefunden hatten. Noch weniger lebten fernab jeglicher Zivilisation als Einsiedler in den Bergen.


\section{Einführung}

Während sich viele dem Alltag des Siedlungslebens fügen suchen vereinzelt Gruppen von Abenteurern nach der Ursache oder gar einer Heilung von dem Virus.
\\Ihr seid als solche Gruppe auf der Suche nach den Sentinels, es heißt sie leben zurückgezogen in einem Atomschutzbunker nahe der ehemaligen Kleinstadt Trekov. Einst hochrangige Wissenschaftler in ihren jeweiligen Disziplinen haben sie Unmengen an Aufzeichnungen über nun verlorene Technologien in Sicherheit gebracht. Wenn jemand eurem Quest helfen kann, dann sie.

Ihr befindet euch zu Beginn eurer Reise bereits am Ortseingang von Trekov. Hinter dem verrosteten Ortseingangsschild macht ihr eine Ansammlung von Häusern aus, denen man ansieht, dass sie schon lange nicht gepflegt wurden.
\\Durch die Häuserfassaden ziehen sich Risse, viele Gartenzäune sind umgerissen worden und auf den Straßen stehen in unregelmäßigen Abständen ausgebrannte Autos.
\\Es liegt eine unheimliche Stille in der Luft an diesem klaren Nachmittag im Spätherbst. In der Ferne ist der Schrei eines Eichelhähers zu höhren.

Was tut ihr?

\section{Trekov}

Die Spieler können sich nun in Trekov frei bewegen. Sie können die Autowracks als auch die Häuser durchsuchen. Beim Durchsuchen des ersten Autos finden sie einen Bolzenschneider. Beim weiteren durchsuchen wird ihnen auffallen, dass das komplette Dorf schon mehr als einmal geplündert wurde. Beim Durchsuchen der Häuser Würfel 1W10. Zeigt der Würfel eine 1-6 ist das Haus geplündert worden.
\\Zeigt der Würfel eine 4-10 kommen sie an das Haus der Familie \TODO{Namen ausdenken}. Nachdem das Event abgeschlossen ist finden die Spieler nur noch bereits geplünderte Häuser. Bewegen sich die Spieler weiter in Richtung des Gebirges kommen sie am Bunker an.

Verlassen die Spieler Trekov in Richtung Westen lies weiter bei Novabarsk. Entscheiden sie sich nach Norden zu gehen treffen sie auf Juggernauts. \TODO{Events kreieren}

\subsection{Haus}

Ihr steht vor einem in Efeuranken gehüllten, einstöckigem Haus. Zwischen dem hochgewachsenen Gras im Vorgarten macht ihr einen Pfad aus, der hinter das Haus führt. Die Eingangstür ist aus massiven Holz gefertigt, in deren Mitte ein eiserner Türklopfer prankt, sie scheint noch intakt zu sein. Rechts neben der Türe könnt ihr auf einem verrostetem Namensschild \TODO{Namen einfügen} entziffern. Auf der anderen Seite der Türe befindet sich ein Briefkasten

\textbf{Im Schlafzimmer des Hauses befindet sich ein Zombie. Der Zombie kann mit einem -20 erschwerten Wurf auf Wahrnehmung o.ä. ausgemacht werden. Der Spieler hört dann ein leises Wimmern. Der Schwierigkeitsgrad kann auf +/- 0 gesenkt werden wenn die Spieler einmal um das Haus herum laufen und sich nach Gefahren umsehen bzw. hören}

\begin{figure}[ht]
	\begin{center}
		\includegraphics[scale=0.25]{./fig/Trekov_Haus.png}
		\caption{Grundriss des Hauses}
	\end{center}
\end{figure}

\subsubsection{Vorgarten}

Die Eingangstür ist abgeschlossen. Sollten die Spieler den Türklopfer betätigen oder sich gewaltsam Zutritt zu dem Haus verschaffen wird der Zombie auf die Spieler aufmerksam und wird versuchen diese anzugreifen nachdem sie das Haus betreten. Durch den Schlitz im Briefkasten können die Spieler einen Zettel erkennen, um ihn allerdings an sich zu nehmen benötigen die Spieler entweder einen Schlüssel um den Briefkasten zu öffnen oder sie versuchen es über eine Fähigkeit wie Fingerfertig, Schlangenmensch o.ä. (mit einem Malus von -30) den Zettel aus dem Briefkasten zu holen. Die Spieler können den Briefkasten auch zerstören allerdings wird der Untote dann auf sie aufmerksam.

Auf dem Zettel steht in einer fein säuberlichen Handschrift:

\begin{itshape}
Liebste Mutter,
\\ich hoffe es geht dir gut. Die Nachrichten sind furchtbar. Ich bin zum Bunker berufen worden. Irgendein biologischer Notfall. Ich hoffe diese Krankheit breitet sich nicht auch noch hier aus. Es wurden wirklich viele Wissenschaftler gerufen. Biochemiker, Genetikexperten und Ärzte. Ich hoffe sie erlauben es, dass wir unsere Familien nachholen dürfen.

Haltet durch! Ich liebe euch.
\\Natascha
\end{itshape}

\subsubsection{Garten}

Folgen die Spieler dem Pfad der hinter das Haus führt kommen sie in den Garten. Dort können sie eine Schaufel, alte Gummihandschuhe und eine Gartenhacke (zerbricht mit einer Chance von 1/1w6 bei jedem Gebrauch, SBW 25) finden. Der Hintereingang zum Haus scheint offen zu sein, da die Tür nur angelehnt ist. Durch die Hintertür gelangen die Spieler in das Wohnzimmer.

\subsubsection{Wohnzimmer}

Das Wohnzimmer ist ärmlich eingerichtet. Hinter einem Esstisch erstrecken sich Sofas und Sessel die einem alten Röhrenfernseher zugewandt sind. In der äußeren Ecke des Raumes befindet sich ein geschmückter Weihnachtsbaum, dessen verweklte Nadeln noch immer auf dem Boden liegen.
\\Geradeaus führt ein Flur zur vorderen Haustür, links daneben  befindet sich ein Durchgang zur Küche.

\subsubsection{Küche}

Die Küche wirkt ungewöhnlich gut in Schuss bis auf die Tatsache, dass neben der Spüle ein toter Hund in seinem Korb liegt. Auf dem Herd stehen leere Töpfe, in den Schränken befinden sich hauptsächlich vergammelte Lebensmittel.
\\Durchsuchen die Spieler den Schrank neben dem Kühlschrank finden sie 1-3 Küchenmesser (SBW 15,20,15). Sind sie hierbei auffällig laut wird der Untote versuchen an sie heranzuschleichen.

\subsubsection{Flur}

Im Flur liegen unter einem Fenster zwei extrem verweste Leichen. Es handelt sich um ein Kind, dass in den Armen einer älteren Frau liegt. Untersuchen die Spieler die Leichen näher erkennen sie, dass beide Einschusslöcher im Kopf vorweisen.
\\Will ein Spieler die Leichen auf Wertsachen untersuchen, so findet er einen Revolver mit 4 Schuss, erhält aufgrund der verwerflichen Tat jedoch einen Malus von -10 auf die restlichen Würfe des Tages.
\\Unterhalten sich die Spieler in der Nähe der Leichen wird der Untote im Schlafzimmer auf sie aufmerksam und wird versuchen durch die Tür zu brechen (Schlagen).

Im anderen Seitenarm des Flures finden sich allerlei Familienportraits, dass abwechselnd zwei Kinder und ihre Eltern zeigt. Neben der Haustür hängt der Briefkastenschlüssel.

\subsubsection{Schlafzimmer}

Sobald die Spieler die Tür öffnen führt der Zombie einen Angriff aus, töten sie den Untoten, so können sie in seinen zerfetzten Jeanshosen ein Taschenmesser sowie Zigaretten und ein Feuerzeug finden.
Unter dem Ehebett befindet sich des weiteren eine Taschenlampe ohne Batterien.

Der Untote hat 70 Lebenspunkte. Weitere Stats finden sich in der unteren Tabelle

\begin{center}
  \input{./tab/zombie_trekov_house.tex}
\end{center}


\subsection{Marktplatz}

Gehen die Spieler weiter in das Dorf hinein gelangen sie an eine offene, gepflasterte Fläche, auf der verstreut Unrat herumliegt.
In der Mitte des Platzes steht ein mittlerweile vertrockneter Brunnen. Davor befindet sich an einer Anschlagtafel eine Karte von Trekov und Umgebung.
Besitzt einer der Spieler eine besondere Auffassungsgabe oder vergleichbare Talente (Oder Spieler würfeln eine um +10 erleichterte Probe) kann er erkennen , dass jemand einen Pfeil und \gqm{HILFE} nahe der Straße zum Uralgebirge in die Tafel geritzt hat.

\begin{figure}[ht]
	\begin{center}
		\includegraphics[scale=0.4]{./fig/Trekov_Karte.pdf}
		\caption{Karte von Trekov und Umgebung}
	\end{center}
\end{figure}

\subsection{Kirche}

Im Inneren der Kirche ist es dunkel. Vereinzelt fällt Licht durch zerbrochene Ornamente. Die wenigsten Kirchenbänke stehen noch an ihren angedachten Plätzen, die meisten sind umgekippt oder gleich völlig zerstört. Im Raum verteilt liegen hin und wieder reglose Körper von Untoten und Menschen gleichermaßen.
Eine Probe auf Sinneswahrnehmung o.ä. kann offenbaren, dass aus einer dunklen Ecke ein gequältes Röcheln hervordringt.

Hierbei handelt es sich um Petrov, einem Schwerverletztem, der sie um Wasser bitten wird. Folgt die Gruppe seiner Aufforderung wird er ihnen von einer missglückten Suche nach den Sentinel erzählen.
\\Seine Gruppe habe die ganze Umgebung um Trekov abgesucht, doch keinen geheimen Unterschlupf gefunden, in dem sich Wissenschaftler verschanzt hätten. Der einzige Ort an dem sie noch nicht gesucht hätten sei das Gelände im Osten (zum Ural hin).

Verwehrt die Gruppe das Trinken wird Petrov ihnen nicht von seiner Expedition erzählen, kann jedoch mit einem Wurf auf überreden umgestimmt werden. Bevor die Gruppe weitergeht wird Petrov sie bitten ihn zu töten, damit er nicht als Zombie verendet.
Durchsuchen die Spieler daraufhin sein Hab und Gut werden sie einige Bandagen und Schmerzmittel in einem Rucksack finden.

\subsection{Friedhof}
Ironischerweise ist auf dem Friedhof die Abwesenheit des Menschen am deutlichsten. Hohes Gras bedeckt die meisten Gräber, auf denen noch immer welke Pflanzen in Tontöpfen stehen. Auffallend ist, dass neben den älteren Gräbern die meist über einen Grabstein samt Namen verfügen auch viele Erdlöcher mit schlichten Holzkreuzen drapiert sind.

Ansonsten findet sich hier - abgesehen von evtl. Grabkerzen - nichts an Wert.

\section{Bunker}

Entschließen sich die Spieler Trekov Richtung Osten zu verlassen treffen sie nach einem längeren Fußmarsch durch den Wald (unabhängig von der in Trekov verbrachten Zeit beginnt es zu dämmern) auf einen 3 Meter hohen Stacheldrahtzaun.
In der Mitte des Geländes steht ein leeres Wärterhäuschen und daneben eine in den Zaun eingelassene Türe. Das Häuschen ist abgeschlossen, im Inneren befinden sich Batterien die auf einem Schreibtisch liegen.
In einiger Entfernung ragt hinter dem Zaun eine massive Felswand hervor. Es scheint eine Betonröhre in die Felswand hineinzuführen.

Der Zaun steht unter Strom. Versuchen die Spieler den Zaun zu berühren erleiden sie einen Stromschlag und Schaden von -10 HP, der Stromschlag kann durch Gummihandschuhe verhindert werden.
Um Zutritt zu dem Gelände zu erhalten können die Spieler die Scheibe des Wärterhäuschens mit einer Kraftprobe (Malus -30) zerstören oder mit einem Bolzenschneider ein Loch in den Zaun schneiden, jedoch können dann im weiteren Verlauf des Abenteuers Untote auf das Gelände gelangen.
\\Alternativ kann die Gruppe an nahegelegenen Bäumen hinaufklettern und über das Dach des Wärterhäuschens den Zaun passieren.

\subsection{Hof}

Abgesehen von einem gepflasterten Weg, der von der Zauntüre in den Tunnel führt ist der Hof leer. Auf dem Weg in den Tunnel tritt einer der Spieler auf dem Schlüssel zum Wärterhäuschen. Des weiteren wird offensichtlich, dass in der Tunnelöffnung ein Falltor ist, das momentan offensteht.

Im Inneren der Betonröhre ist es dunkel, vor ihnen ertönt ein merkwürdiges Kratzgeräusch. Haben die Spieler eine Lichtquelle so erkennen sie, dass ein Untoter an einer massiven Stahltür kratzt.
Der Zombie wird im Verlauf des Kampfes probieren zu kreischen, gelingt ihm dies werden 1W6/2 Zombies angelockt, die je nach früheren Entscheidungen auf das Gelände gelangen und der Gruppe in den Rücken fallen könnten.

\begin{center}
  \input{./tab/Tunnel_Zombie.tex}
\end{center}

Neben der Stahltür ist in den Stein ein Terminal eingelassen. Die Tastaturbeleuchtung verrät, dass es über Strom verfügt und scheinbar noch funktionstüchtig ist.
Eine erfolgreiche Probe auf Auffassungsgabe o.ä offenbart, dass oberhalb der Türe ein rotes Licht in regelmäßigen Abständen aus der Dunkelheit blinkt.
Im Tunnel hat die Gruppe folgende Optionen:

\begin{itemize}
  \item Wahllos Codes eingeben
  \\Nach einer falschen Eingabe leuchtet das Display rot auf und es passiert nichts. Sollten drei flasche Codes in Folge eingegeben werden schließt sich das vordere Tor.
  Wenn Spieler nicht augenblicklich nach draußen stürmen (Probe auf Rennen, um -20 erschwert) werden sie im Tunnel eingeschlossen.
  Hat sich das Tor geschlossen kehrt kurz Stille ein gefolgt von einem Zischen, das nicht weiter ausfindig zu machen ist.
  Hierbei handelt es sich um Lachgas, dass die Spieler ausknockt. Am nächsten Morgen wachen sie im Reinraum des Bunkers auf.

  \item richtiger Code wird eingegeben
  \\Aus purem Zufall kann die Kombination 1701 erraten werden. Daraufhin öffnet sich die Türe und die Spieler können eine Treppe hinuntersteigen um auf einem langen Flur anzukommen. Hier wird ihnen ein Wissenschaftler über den Weg laufen, der augenblicklich fliehen wird. Kurze Zeit später taucht ein Trupp aus drei schwerbefaffneten Söldnern auf.

  \item Vor der Kamera gestikulieren / Den Tunnel verlassen
  \\Plötlich ertönt eine Frauenstimme, die von den Felswänden widerhallt und nach der Intention der Abenteurer fragt. Überzeugen die Spieler die Stimme wird sich die Tür öffnen und Söldner sie auffordern ihre Waffen abzugeben. Daraufhin werden sie zum Büro der Direktorin geleitet.
\end{itemize}

\section{Novabarsk}



\end{document}
